\newpage
\section{Discussion}
\label{sec:discussion}

%---------------------------------------------------------------------------------------
\subsection{Limitations}
\label{sec:limitations }

\paragraph{Multiple Load Profiles.}
Although the load profiles used in the test case are static (their buses is pre-defined) by platform design it 
is often the case in real life. However, we did not handle multiple loads at once, and the platform is not yet 
designed to handle them. This is a limitation that should be addressed in future work. It suggests is yet designed 
as a single demand per bus-time pair.

\paragraph{Limited Benchmarking.}
The main limitation of the platform is that it lacks comprehensive benchmarking against established 
standards or similar systems. While the platform demonstrates functionality, there is no systematic comparison 
(e.g. IEEE test cases). This makes it difficult to objectively assess the platform's effectiveness compared 
to existing solutions.


\subsection{Methodological Insights}
\label{sec:methodological_insights}

\paragraph{Technical-Economic Integration.}
The platform's key methodological contribution lies in establishing a direct link between technical system design
and economic valuation. By integrating technical dispatch optimization with comprehensive financial analysis, it
enables stakeholders to evaluate both the operational feasibility and economic viability simultaneously during
the design phase. This represents a significant advancement over traditional approaches that often treat technical
and economic assessments as separate processes.

\paragraph{Renewable Asset Valuation.}
The optimization framework enables assessment of renewable energy assets by quantifying the costs of 
conventional "dirty" generation within the entire system. This cost metric, when combined with investment 
parameters, provides a clear basis for comparing different scenarios and determining the profitability of 
renewable alternatives. Rather than evaluating green assets in isolation, this approach reveals their true 
economic value by measuring their impact on overall system costs.

\paragraph{Storage Optimization.}
The platform's analysis of battery state-of-charge patterns reveals opportunities for optimizing 
storage deployment timing and sizing. By examining partial-year installation scenarios, as shown in the 
state-of-charge comparison figures, we can better determine not just the optimal storage capacity but also 
when during the year new storage should be commissioned. This temporal dimension of storage deployment 
represents an important area for future framework development.

\paragraph{Maintenance Management.}
The platform could be enhanced by incorporating a maintenance re-routing system that creates temporary 
alternative paths during asset downtime. This would allow for seamless transitions during maintenance periods 
by automatically redirecting power flows through equivalent backup systems. Such functionality would improve 
system reliability and provide more realistic operational scenarios that account for planned and unplanned 
maintenance events.


%---------------------------------------------------------------------------------------
\subsection{Future Directions}
\label{sec:future_directions}

\paragraph{Benchmarking and Validation.}
A critical next step is establishing comprehensive benchmarking 
against industry standards and IEEE test cases. This would validate the platform's performance and provide 
objective comparisons with existing solutions, building confidence in its results and highlighting areas 
for improvement.

\paragraph{Real-Time Price Integration.}
Incorporating real-time electricity pricing mechanisms would significantly enhance the platform's practical utility. This could enable:

\begin{itemize}
    \item Analysis of arbitrage opportunities between peak and off-peak periods
    \item Evaluation of green energy resale potential, particularly for private owners 
    \item More accurate modeling of revenue streams from grid services
    \item Dynamic optimization of storage charging/discharging based on market conditions
\end{itemize}

\paragraph{Code Optimization.}
While computational efficiency is not an immediate concern, for any serious 
use of the platform the code reliability and maintainability should be improved. Same goes for the interactions 
between the different components.

\paragraph{Enhanced Prompt Engineering.}
Further prompt engineering work could improve report readability, 
aiming at facilitating stakeholder engagement. This would ensure key metrics and findings are presented in clear, 
actionable formats aligned with industry standards.


\subsection{Concluding Remarks}
This platform successfully bridges technical dispatch simulations with investment analysis in energy systems. 
The test cases validate the core functionality while demonstrating how sensitivity analyses, scenario comparisons, 
and AI-enhanced reporting can generate actionable insights. Future development priorities should be chosen based 
on specific use cases, whether utility-scale planning, microgrid optimization, or renewable integration. However,
benchmarking against industry standards and IEEE test cases is a must.