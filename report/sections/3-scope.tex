% ---
% 3  Problem Definition and Scope
% ---
\newpage
\section{Problem Definition and Scope}

This project extends the first-year LP framework in two directions:
\begin{itemize}
    \item \textbf{MILP investment-dispatch model} – replaces the single-year LP with a multi-year 
    mixed-integer formulation that chooses both what to build and how to run it, so CAPEX and 
    OPEX are minimised in one pass.
    \item \textbf{Independent ML forecasting module} – delivers day-ahead predictions of the 
    exogenous time series (load and variable renewables). It is stand-alone for now but provides 
    the data that a rolling short-term optimisation would need.
\end{itemize}

This section outlines what is inside the study, what stays outside, and which indicators we will track.

\subsection{Planning Horizon and System Boundaries}

The scope of the study is defined by several key elements:

\textbf{Strategic horizon:} The model considers a user-defined list of years, normally between 1 and 10. 
This allows the optimisation to capture the timing of asset builds and retirements.

\textbf{Operational resolution:} Each year is represented by three typical weeks (winter, summer, 
spring-autumn). This approach keeps the MILP size modest while preserving essential seasonal detail.

\textbf{Forecast horizon:} The main forecasting target is 24 hours ahead, with additional checks at 48, 72, and 168 hours. This aligns with day-ahead operational needs and allows us to observe how forecast quality degrades with longer horizons.

\textbf{Network footprint:} The optimisation is run on a fixed test grid, including buses, lines, and assets. This lets us attribute changes in results to the model itself, not to changes in the underlying data.

Everything behind the connection point, as well as retail tariffs and ancillary services, is excluded from the study.

\subsection{Decision Variables and Constraints}

\begin{center}
\begin{tabular}{lll}
\textbf{Variable} & \textbf{Type} & \textbf{Status} \\
\hline
$b_{g,y},\;b_{s,y}$ & binary & new -- build decisions for generators $g$ and storage $s$ \\
$u_{g,y,t}$ & binary & new -- on / off for thermal generators \\
$p_{g,y,s,t}$ & continuous & carried over -- dispatch per generator \\
$p^{\text{ch}}_{s,y,s,t},\;p^{\text{dis}}_{s,y,s,t}$ & continuous & carried over -- storage charge, discharge \\
$e_{s,y,s,t}$ & continuous & carried over -- state of charge \\
$f_{l,y,s,t}$ & continuous & carried over -- DC line flow \\
\end{tabular}
\end{center}

New or changed constraint families include:
\begin{itemize}
    \item \textbf{Build--lifetime link:} Sums of $b_{g,y}$ or $b_{s,y}$ set the installed flag for each year and forbid overlapping rebuilds.
    \item \textbf{Unit commitment:} Minimum up/down times, start costs, and ramp limits are tied to $u_{g,y,t}$.
    \item \textbf{CAPEX annuity:} Annual cost terms are based on build binaries and a capital recovery factor.
    \item \textbf{Storage relaxed final SoC:} The end-of-week state of charge may float within 10\% of capacity to speed up the solve.
\end{itemize}

All other LP constraints—such as power balance, line limits, storage energy balance, and renewable profiles—remain in place, but now reference the new binary variables where needed.
These elements turn the LP into a MILP, giving the solver the freedom to co-optimise when to invest and how to operate.

\subsection{Forecast Horizon and Accuracy Targets}

The forecasting module generates:
\begin{itemize}
    \item \textbf{Primary target} -- 24 hourly values for the next day for each time series.
    \item \textbf{Additional checks} -- 48 h, 72 h and 7-day horizons to stress-test model degradation.
    \item \textbf{Features} -- calendar flags, recent lags, rolling means, simple weather proxies.
    \item \textbf{Models} -- linear baseline, random forest, gradient boosting, LSTM; all treated with identical split and scaling rules.
\end{itemize}

Accuracy targets (set after an initial back-test):

\begin{center}
\begin{tabular}{ll}
\textbf{Metric} & \textbf{Day-ahead target} \\
\hline
MAE & $< 5\%$ of mean load \\
RMSE & $< 7\%$ of mean load \\
\end{tabular}
\end{center}

Longer horizons have looser bands, logged but not optimised against.

\subsection{Key Performance Indicators}

The following indicators are tracked to assess the value of the MILP and the forecasting module:
\begin{itemize}
    \item \textbf{Cost:} Total discounted system cost, including CAPEX annuities and weighted OPEX, is the main optimisation objective.
    \item \textbf{Solver:} MILP wall-clock time and optimality gap are reported for each run.
    \item \textbf{Forecast:} MAE and RMSE are tracked for each horizon and time series; lower values are better.
    \item \textbf{Robustness:} The number of hours with unmet demand or line overload is monitored and should be zero.
    \item \textbf{Transparency:} The share of total cost by asset class is reported to support sensitivity analysis.
\end{itemize}
% ---

\newpage
\section{Scope, Model Structure, and Key Performance Indicators}
\label{sec:scope_model}

\subsection{Scope and Boundaries}
The study focuses on grid investment and operation under high renewable shares, explicitly modeling both long-term asset decisions and short-term operational constraints. The boundaries are as follows:
\begin{itemize}
    \item \textbf{Strategic planning horizon:} Multi-year, typically 1–10 years, capturing asset builds, retirements, and their timing.
    \item \textbf{Operational resolution:} Each year is represented by three typical weeks (winter, summer, spring/autumn), balancing MILP tractability and seasonal fidelity.
    \item \textbf{Forecasting:} Day-ahead (24h) renewable and demand forecasting, with additional checks at 48, 72, and 168 hours to assess degradation.
    \item \textbf{System boundaries:} The model runs on a fixed test grid (buses, lines, assets), with retail tariffs and ancillary services excluded. Everything behind the grid connection point is out of scope.
\end{itemize}

\subsection{Model Structure and Variables}
The MILP jointly optimizes investment (CAPEX) and operational (OPEX) decisions:
\begin{itemize}
    \item \textbf{Investment variables:} Binary decisions for whether to build generators ($b_{g,y}$) or storage ($b_{s,y}$) in each year.
    \item \textbf{Unit commitment:} Binary on/off states for thermal units ($u_{g,y,t}$), supporting future extensions (min up/down, start costs).
    \item \textbf{Operational dispatch:} Continuous variables for generator output, storage charging/discharging, state-of-charge, and DC line flow.
\end{itemize}
Key new/changed constraints:
\begin{itemize}
    \item \textbf{Asset lifetime:} Enforces non-overlapping rebuilds and tracks active years.
    \item \textbf{Unit commitment:} Enforces commitment logic, linking minimum up/down and future start costs.
    \item \textbf{CAPEX annuity:} Investment cost spread using a capital recovery factor.
    \item \textbf{Storage:} Final state-of-charge relaxed within 10\% of nominal, improving solver speed.
\end{itemize}
All core physical and economic constraints (power balance, line limits, storage, renewables) remain, now referencing new binary variables as needed.

\subsection{Forecasting Module}
A stand-alone ML forecasting module predicts exogenous series (load, wind, PV) with the following features:
\begin{itemize}
    \item \textbf{Targets:} 24 hourly values for the next day, plus 48h, 72h, and 7-day checks.
    \item \textbf{Features:} Calendar indicators, recent lags, rolling means, weather proxies.
    \item \textbf{Models:} Linear, random forest, gradient boosting, LSTM (all use common splits/scaling).
\end{itemize}
Accuracy is quantified via MAE and RMSE, with day-ahead targets set to $<$5\% (MAE) and $<$7\% (RMSE) of mean load.

\subsection{Key Performance Indicators (KPIs) and Assessment}
The following metrics are tracked to evaluate the planning and forecasting framework:
\begin{itemize}
    \item \textbf{System cost:} Total discounted cost (CAPEX annuity + OPEX) as main optimization objective.
    \item \textbf{Solver performance:} MILP wall-clock time and optimality gap for each scenario.
    \item \textbf{Forecast accuracy:} MAE and RMSE for each horizon and time series.
    \item \textbf{Robustness:} Number of hours with unmet demand or line overloads (should be zero).
    \item \textbf{Transparency:} Share of total system cost by asset class for scenario analysis.
\end{itemize}
These indicators validate the model’s ability to recommend robust, cost-effective investment and operational strategies under uncertainty.
