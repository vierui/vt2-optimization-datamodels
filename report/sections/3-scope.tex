\newpage
\section{Methodology and Scope}
\label{sec:method_scope}

This project addresses investment planning and operational scheduling for a power grid under a mix of 
asset types and growing demand. We justify below the core modeling choices and set boundaries 
for both optimization and forecasting, explaining the rationale behind the modeling of our grid setup. 
Technical formulations and implementation details are deferred to Sections~\ref{sec:MILP_transition} 
and~\ref{sec:forecasting}.

\subsection{Optimization model -- Long-Term Planning}
The scope is deliberately focused: all modeling and analysis are performed on a fixed transmission 
grid with a representative asset mix (generators, renewables, storage). The grid configuration spans
from the grid size to the number of buses, lines, and different asset types/locations. Same goes for 
the loads and generation profiles.

Multi-year power system planning requires both a long-term investment decisions and short-term 
operational modeling based ( which is here, a power flow problem optimzation). Linear Programming can only 
handle continuous decision variables. We previously used it to solve the power flow problem, modelin it with 
linear constraints of the form:
\[
Ax \leq b, \quad x \in \mathbb{R}^n
\]
A LP is unsuitable for modeling decisions involving logical conditions such as “if investment is made, 
then operation is allowed”. Hence no ability to represent discrete (on/off) logic or in our case, 
unit-commitment decisions. 

Mixed-Integer Linear Programming (MILP) extends LP by introducing binary variables, enabling explicit 
modeling of build/replace decisions and operational constraints. The integrated MILP framework used 
here allows capital (CAPEX) and operating (OPEX) costs to be co-optimized, simulating when and how 
assets are built, retired, and optimally dispatched~\cite{andersson2004power, wood2013power}.


\subsection{Machine Learning model -- Short-Term Forecasting}

Short-term forecasting is essential for effective system operation under high shares of solar and 
wind. Approaches range from statistical models (e.g., ARIMA/SARIMA) to machine learning 
methods such as Gradient Boosted Decision Trees (GBDT/XGBoost). Statistical 
models provide interpretable baselines and exploit seasonality and autocorrelation, while ML models 
capture nonlinear patterns in larger, more complex datasets. Hybrid or ensemble approaches can 
combine strengths of both~\cite{grzebyk2021xgboost, zhong2020xgboost}.

For this project, both families are benchmarked: Gradient Boost as the primary forecasting model and 
SARIMA as a baseline. The forecasts target day-ahead horizons, aligning with demand for markets operations
or grid stability to mitigate risks.

\subsection{Solver and Toolchain Selection}

Energy system MILPs are computationally challenging. Open-source solvers (CBC or GLPK) are widely 
used in research but can struggle with large models, particularly those involving many binaries and 
long planning horizons. Commercial solvers (CPLEX, Gurobi) offer better performance, robustness, and 
advanced features such as parallelization, which can be critical for realistic 
studies~\cite{mittelmann2023benchmarks}. This project uses CPLEX via its Python API 
(\texttt{docplex})~\cite{ibm2022cplexpython}.

Python is the de facto standard language for research and development in optimization and ML due to 
its broad library ecosystem. Key packages in this project include:
\begin{itemize}
    \item \texttt{docplex} for optimization and CPLEX solver integration~\cite{ibm2022cplexpython}
    \item \texttt{scikit-learn}, \texttt{XGBoost} for ML/forecasting ~\cite{grzebyk2021xgboost, zhong2020xgboost, scikit-learn}
    \item \texttt{statsmodels} for SARIMA/statistical modeling~\cite{seabold2010statsmodels}
    \item \texttt{Optuna} and \texttt{skopt} for hyperparameter tuning ~\cite{head2018scikitoptimize}
    \item \texttt{pvlib} for solar/weather features calculations~\cite{holmgren2018pvlib}
    \item standard data-handling libraries (\texttt{pandas}, \texttt{numpy}, \texttt{matplotlib})~\cite{mckinney-proc-scipy-2010},~\cite{harris2020array}
\end{itemize}

\subsection{Scope and Boundaries of the Study}
The project’s scope covers:
\begin{itemize}
    \item \textbf{Strategic Horizon} Multi-year (typically 1--30 years but could be adapted), with three
    representative weeks per year (winter, summer, spring/autumn) to keep the MILP tractable but seasonally realistic.
    \item \textbf{System Boundaries} A fixed test grid with predefined buses, lines, and asset 
    candidates; retail tariffs and ancillary services are excluded. The grid is fixed during 
    the optimization but configured before.
    \item \textbf{Decisions} Investment timing and size (build binaries), operational scheduling 
    (dispatch, unit commitment), and storage operation.
    \item \textbf{Forecasting} Independent module producing day-ahead (visually up to 2 days and 
    metrics up to 1 week) forecasts for Photovoltaic (PV) availability with accuracy assessed by 
    MAE, RMSE and R\textsuperscript{2}.
    \item \textbf{Performance indicators} Total system cost (objective function), solver performance 
    (solve time, optimality gap), and forecast error (MAE, RMSE).
\end{itemize}

Elements outside scope include demand-side management, real-time balancing, grid expansion, and 
non-economic reliability constraints. All implementation and testing are performed in a reproducible 
Python workflow using \texttt{Poetry} for project and dependency management. Project metadata and 
dependencies are in the \texttt{pyproject.toml} file ; similar to VT1.

In summary, the methodology rests on (i) an integrated MILP model for joint investment and operation 
planning for a multi-year horizon, (ii) a day-ahead, Python-based ML/statistical forecasting pipeline, and (iii)
robust solver and workflow choices. Technical details and results for each major block are presented in 
subsequent sections.