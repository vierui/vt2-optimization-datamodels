\newpage
\section{Introduction}
\label{sec:intro_lit}

\subsection{Context and Motivation}
Electricity powers everything: homes, factories, and even the internet. 
Global energy demand is rising, and the shift to renewables like solar and wind is accelerating. But, 
this transition brings new challenges. Renewables are unpredictable, and increasing digitalization means 
more volatile and less controllable demand. Recent blackouts have exposed how dependent we are on the grid, 
and how easily disruptions cascade to other critical infrastructure, including the telecom network. 
This makes planning both short- and long- term demand or renewable generation more urgent than ever.

Building on last semester's linear programming (LP), we now introduce a mixed-integer linear programming (MILP) framework 
that optimizes both long-term investments and operations in an integrated way. In parallel, we developed a machine 
learning forecasting module to predict short-term (day-ahead) availability of renewable generation.

\subsection{Long-Term Planning with Mixed Integer Optimization}
The initial model, developed during the HS24 semester, addressed power flow and investment planning 
with a linear programming (LP) approach: fixed demand, hand-crafted scenarios, and only continuous 
variables for decision-making. While this enabled proof-of-concept studies, it had significant 
limitations and some lack of 'mathematical elegance'. The reliance on manually defined scenarios meant the true 
best investment strategy might never be evaluated. As a result, the model could not efficiently or systematically 
explore the full solution space for integrated planning.

The new version (VT2) adopts Mixed-Integer Linear Programming (MILP), where discrete (binary) variables 
enable us to model on/off states, asset retirements, maintenance cycles, and investment decisions in a unified, 
realistic manner~\cite{andersson2004power, wood2013power}. MILP co-optimizes capital and operational expenditure 
(CAPEX and OPEX), preventing suboptimal investment timing or dispatch caused by treating planning and operation 
separately. A commercial solver (IBM CPLEX) is now used and previous open-source solver (GLPK via PuLP) who 
is dramatically outperformed with large MILP problems and lack of parallelization.

\subsection{Short-Term Forecasting with Machine Learning}
Short-term forecasting of PV and wind is critical for grid operation. Methods range from simple persistence 
baselines to advanced ML models. We benchmarked several, ultimately using gradient boosting decision trees for primary 
forecasting and SARIMA as a baseline~\cite{grzebyk2021xgboost, zhong2020xgboost}. This enables to create a baseline 
model to predict electricity availability hence predict electricity spot prices based on single- imporant assets

Python’s mature ecosystem makes it the dominant platform for optimization and ML in energy systems. We leverage 
libraries such as scikit-learn, and statsmodels. Goal is to provide a baseline model to understand which factors drive
the electricity availability in the short-term.

%--------------------------------
% \section{Introduction}

% \subsection{Context \& Motivation}
% Electricity powers everything: homes, factories, even the servers running the latest 
% AI tools. The world's hunger for energy keeps rising, and the shift to renewables 
% like solar and wind is only speeding up. But that switch brings headaches: solar and 
% wind aren't predictable, and as more things go digital, unpredictable spikes can shake 
% the whole grid. Recent blackouts have made us realize how dependent we are on 
% the grid and often forget that internet is for now no existing without electricity.
% It's clear: we need to get better at planning for both the next hour and the next decade, 
% while also pushing for the integration of renewables sources of energy into the grid.

% \subsection{VT1 Bottlenecks}
% In the first version (VT1), we built a model that handled power flow and investment, 
% testing it on a handful of scenarios we set up by hand. The basics worked:
% \begin{itemize}
%     \item Demand was fixed and didn’t adapt.
%     \item Each scenario had to be loaded and managed manually.
%     \item The math (LP) let us solve problems, but skipped things like startup costs or 
%     on/off states for power plants.
%     \item We had to glue things together with custom scripts.
% \end{itemize}
% This setup proved the idea, but it missed the mark when it came to real-world details—
% especially picking the best grid setup based on cost and performance.

% \subsection{Goals and Scope of VT2}
% VT2 takes a major step forward, tackling the main gaps from the first version 
% with a more advanced and realistic approach:

% \begin{itemize}
%     \item \textbf{Realistic operational modeling:} Instead of only using 
%     continuous variables, VT2 upgrades to mixed-integer linear programming (MILP). 
%     This lets us explicitly represent unit commitment—so power plants can 
%     switch on or off, model asset retirements and simulate maintenance.
%     Once the logic implemented, we are a few code lines away of also introducing minimum up/down times, 
%     or start-up costs models but these were not implemented in this version.
%     \item \textbf{Data-driven renewable integration:} the new version now incorporates 
%     historical and reanalysis weather data to generate realistic wind and 
%     solar production forecasts. This means we’re not just simulating “average” 
%     days, but capturing the variability and uncertainty that challenge the grid.
%     making it possible to stress-test decisions against extreme events, 
%     seasonal lows, and sudden surges in demand.
%     \item \textbf{Modular, robust codebase:} The entire model has been 
%     refactored into a modular Python package using \texttt{Poetry}. Automated 
%     testing and continuous integration ensure that every change is checked for consistency and reliability. This not only speeds up development but makes it easier to extend—whether that’s adding new assets, market mechanisms, or scenario types.
% \end{itemize}

% With these improvements, VT2 shifts from a simple prototype to a more robust 
% planning and decision-support tool. It can now assess investment and operational 
% strategies under high renewable shares, tight operational constraints, and 
% uncertain future conditions—laying the groundwork for future features like 
% demand response and market-clearing simulation.