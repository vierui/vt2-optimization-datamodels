\newpage
\section{Introduction}
\label{sec:intro_lit}

\subsection{Context and Motivation}
Electricity powers everything: homes, factories, and even the internet. 
Global energy demand is rising, and the shift to renewables like solar and wind is accelerating. But, 
this transition brings new challenges. Renewables are unpredictable, and increasing digitalization means 
more volatile and less controllable demand. This makes planning both short- and long- term demand or 
renewable generation more urgent than ever.

Building on last semester's investment analysis with linear programming (LP) for energy assets, we now 
introduce a mixed-integer linear programming (MILP) framework that optimizes both long-term investments
and operational states in an integrated way. In parallel, we developed a machine learning forecasting 
module to predict short-term (day-ahead) availability of renewable generation.

\subsection{Long-Term Planning with Mixed Integer Optimization}
The initial model, developed during the HS24 semester, addressed power-flow problem optimization and 
investment planning with a linear programming (LP) approach: fixed demand, hand-crafted scenarios, and only 
continuous variables for decision-making. While this enabled proof-of-concept studies, it had significant 
limitations and some lack of 'mathematical elegance'. The reliance on manually defined scenarios meant the true 
best investment strategy might never be evaluated. As a result, the model could not efficiently or systematically 
explore the full solution space for integrated planning.

The new version (VT2) adopts Mixed-Integer Linear Programming (MILP), where discrete (binary) variables 
enable us to model on/off states, asset retirements, maintenance cycles, and investment decisions in a unified, 
realistic manner~\cite{andersson2004power, wood2013power}. MILP co-optimizes capital and operational expenditure 
(CAPEX and OPEX), preventing suboptimal investment timing or dispatch caused by treating planning and operation 
separately. A commercial solver (IBM CPLEX) is now used and outperforms the open-source solver (GLPK via PuLP) 
for large MILP problems. It lacks of parallelization too.

\subsection{Short-Term Forecasting with Machine Learning}
Short-term forecasting of PV and wind is critical for grid operation. Methods range from simple persistence 
baselines to advanced ML models. We benchmarked several, ultimately using gradient boosting decision trees for primary 
forecasting and a statistical model as a baseline~\cite{grzebyk2021xgboost, zhong2020xgboost}.

Python’s mature ecosystem makes it the dominant platform for optimization and ML in energy systems. We leverage 
libraries such as scikit-learn, and statsmodels. Goal is to provide a baseline model to understand which factors drive
the electricity availability in the short-term.