% --- Literature and Toolchain Review ---
\section{Literature and Toolchain Review}

\subsection{Mixed-Integer Programming in Power-System Planning}
Mixed-Integer Programming (MILP) is an improved version of Linear Programming (LP) 
that allows to solve problems with discrete variables for example whether a unit is on or off
(unit commitment). Such binary decisions can not be modelled with pure linear programming.
This allows to integrate investment (capital expenditure) and operational (dispatch, or operating cost)
decisions into one optimization framework. One can co-optimize the true least-cost solution 
that considers both capex and opex together. \cite{andersson2004power, wood2013power}.

Formulating generation expansion planning (GEP) as an MILP captures the binary nature of building 
decisions (build vs. not build) and can include operational details like unit commitment. Despite being possible to include operation 
details such as minimum on/off time, rampe rates, etc. in the model, it was not done in this project.
Not only is it more realistic but it also avoids suboptimal decisions that could arise from 
treating planning and operation separately.

\subsection{Short-term PV/Wind Forecasting Methods}
Short-term forecasting of photovoltaic (PV) or wind power is vital for grid operations. 
Approaches include:
\begin{itemize}
    \item \textbf{Persistence/Empirical}: Simple baselines, e.g., assuming tomorrow equals today.
    \item \textbf{Physical/NWP}: Use weather forecasts and physical models for power prediction.
    \item \textbf{Statistical}: ARIMA/SARIMA and ARIMAX models learn from historical data and 
    exogenous variables \cite{predictive_modeling_notes}. They are interpretable and 
    data-efficient but limited for nonlinearities.
    \item \textbf{Machine Learning}: Methods like neural networks, SVR, and especially gradient 
    boosting (e.g., XGBoost) capture complex patterns and often outperform statistical models 
    when sufficient data is available \cite{grzebyk2021xgboost, zhong2020xgboost, scikit-learn}. 
    Gradient boosting is noted for its accuracy and speed in PV/wind forecasting.
    \item \textbf{Hybrid/Ensemble}: Combine models (e.g., ARIMA+ANN) for improved robustness, 
    though added complexity may not always yield better results.
\end{itemize}
Given these findings, we used gradient boosting (XGBoost) for forecasting, with SARIMA as a 
baseline.

\subsection{Solver Landscape and Selection}
Solving large MILPs requires robust solvers. Commercial options (CPLEX, Gurobi, Xpress) are 
state-of-the-art, offering fast solve times and reliability \cite{mitchell2011pulp, forrest2018cbc}. 
Open-source solvers (CBC, GLPK, SCIP, HiGHS) are free but generally slower and less robust. 
Benchmarks show Gurobi and CPLEX are typically 12--100x faster than CBC, and commercial solvers 
solve more instances to optimality \cite{mittelmann2023benchmarks}.

For this project, CPLEX was chosen for the sake of understanding the solver and for having a 
academic license. Additionally, CPLEX offers a Python API (docplex), which made integration with 
our Python-based workflow straightforward

\subsection{Python Ecosystem for Optimisation and ML}
This interoperability and abundance of libraries is a major reason Python is so dominant in these 
fields today. Our literature review also confirmed that many recent research works in energy systems 
adopt Python for similar tasks, citing its balance of user-friendliness and powerful capabilities

Below is a list of the most relevant libraries for optimization and machine learning:

\begin{itemize}
    \item For optimization, libraries like PuLP and Pyomo allow flexible model formulation and 
    solver switching \cite{mitchell2011pulp}. We used the CPLEX Python API (docplex) for direct 
    integration. 
    \item For ML, scikit-learn and XGBoost provide powerful tools for data processing and 
    forecasting. Others librairies such as PyTorch, TensorFlow, and Keras are the reference ones 
    for deep learning and neural networks \cite{scikit-learn}. 
    \item Statsmodels was used for time-series (SARIMA) modeling as a baseline model.
\end{itemize}

\newpage
