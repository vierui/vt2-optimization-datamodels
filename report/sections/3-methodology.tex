\newpage
\section{Methodology}

as mentioned before, the code base has been upgraded to a multi-year MILP and a new
forecasting module. The codebase relies on that strucutre hence the methodology used.

The change on the code base happen mostly between these three blocks:
% \begin{enumerate}
% \item \textbf{Data layer} – \texttt{data/} – Reads network configuration and generation/load time-series
% \item \textbf{Investment model} – \texttt{scripts/} – Solves and optimises the investment problem for a given grid
% \item \textbf{Forecasting module} – \texttt{forecast/} – Builds a day-ahead prediction based on statistical and machine learning models
% \end{enumerate}
\begin{figure}[h!]
  \centering
    \begin{verbatim}
        investment-model/
          +-- data/                       # data layer
          |   \-- ...
          |
          +-- forecast/                   # day-ahead forecasting module
          |   \-- ...
          |
          \-- scripts/                    # MILP investment model
          |   \-- ...
          ...
    \end{verbatim}
  \caption{Code architecture - showing the main components: data layer for grid and time-series inputs, forecasting module, and optimization scripts.}
  \label{fig:code-arch}
\end{figure}

\subsection{Data layer}
\label{sec:data-layer}

The data layer looks similar to the previous by splitting static part (network topology and the assets specifications)
and asset profiles. (defining profiles for the demand and assets capabilities) 
However, switching to a multi-year MILP removed the need for per-scenario network files !
hence creation of a new configuration file \texttt{analysis.json}, which defines the parameters of the investment problem (optimization) 
such as the planning horizon, annual load growths, and representative weeks selection.

In the new version the static grid data files have been simplified, still replicating the
metadata of the previous version using the matlab method (SEARCH FOR REF) but only the necessary 
columns were kept -> made use to clean 
Representative-week slicing (three 168h blocks) is carried over from the LP version and therefore not
described here again.

\begin{figure}[h!]
  \centering
    \begin{verbatim}
        data/         
          +-- grid/                       
          |   +-- analysis.json           # configuration file : horizon, growth, ...
          |   +-- buses.csv               # network topology
          |   +-- generators.csv          # assets specifications, lifetime and CAPEX
          |   +-- lines.csv               # network topology
          |   +-- loads.csv               # demand profile
          |   \-- storages.csv            # assets specifications, lifetime and CAPEX
          +-- processed/                  
              +-- load-2023.csv           # load time-series
              +-- solar-2023.csv          # solar time-series
              \-- wind-2023.csv           # wind time-series
    \end{verbatim}
  \caption{Data layer - showing the main components: static grid data and generation and load profiles.}
  \label{fig:data-layer}
\end{figure}

\subsection{Linear to Mixed-Integer Programming Transition}
\label{subsec:lin_to_milp}

The original \texttt{vt1} framework formulated each seasonal DC--OPF as a \emph{pure linear program} (LP).  
Asset capacities, locations and lifetimes were fixed \emph{exogenously} in a scenario file; the optimiser merely 
scheduled hourly dispatch.  
While convenient, this linear setting could not answer strategic questions such as
\emph{“When should we build or replace a 200\,MW CCGT?”} or
\emph{“Is storage cheaper than new wind under a 10-year horizon?”}.  
To embed such build-versus-dispatch trade-offs directly inside the optimiser, \texttt{vt2} elevates the model to a
\emph{mixed-integer linear program} (MILP).

\subsubsection{New decision layers}
Let $G$ be the set of candidate generators, $S$ the set of storage units,  
$\mathcal{Y}=\{1,\dots,Y\}$ the planning years, and $\Sigma$ the representative seasons.
For every asset $a\!\in\!G\cup S$ and year $y\!\in\!\mathcal{Y}$ we introduce binary variables
\[
\boxed{\;b_{a,y}\in\{0,1\}: \text{ ``build (or replace) asset $a$ in year $y$''}\;}
\qquad
\boxed{\;z_{a,y}\in\{0,1\}: \text{ ``asset $a$ is operational in year $y$''}\;}
\]
Dispatch, power-flow and storage variables remain continuous exactly as in the LP.

\subsubsection{Lifetime logic}
If $L_a$ is the technical lifetime of asset $a$, the staircase constraint
\begin{align}
z_{a,y} \;=\; \sum_{\substack{y'\le y\\ y-y' < L_a}} b_{a,y'}
\qquad\text{and}\qquad
\sum_{\substack{y'\le y\\ y-y' < L_a}} b_{a,y'} \;\le\; 1
\label{eq:lifetime}
\end{align}
(i) activates an installation for $L_a$ consecutive years and  
(ii) forbids overlapping rebuilds.  
Equation~\eqref{eq:lifetime} is linear and keeps the MILP compact---only
$\lvert G\rvert+\lvert S\rvert$ binaries per year.

\subsubsection{Unified objective}
Operational and capital costs are now minimised \emph{simultaneously}:
\begin{equation}
\min \;
\underbrace{\sum_{s\in\Sigma} w_s
           \sum_{y\in\mathcal{Y}}\sum_{g\in G}\sum_{t} c_g\,p_{g,y,s,t}}
_{\text{fuel \& variable}\;(\text{continuous})}
\;+\;
\underbrace{\sum_{y\in\mathcal{Y}}\bigl(
            \sum_{g\in G} \text{CRF}_g\,\text{CapEx}_g \, z_{g,y} +
            \sum_{s\in S} \text{CRF}_s\,\text{CapEx}_s \, z_{s,y}\bigr)}
_{\text{annualised capex}\;(\text{binary})}.
\label{eq:mixed_obj}
\end{equation}
The Capital‐Recovery Factor $\text{CRF}_a(i,L_a)=\tfrac{i(1+i)^{L_a}}{(1+i)^{L_a}-1}$
linearises discounting inside the model; no external NPV spreadsheet is required.

\subsubsection{Side-effects of the transition}
\begin{itemize}[leftmargin=1.8em]
  \item \emph{Richer feasibility space:} load-growth scaling, retirement gaps and replacement timing can be explored within one optimisation run instead of a Monte-Carlo over scenarios.
  \item \emph{No load-shedding slack:} demand must be met unless the MILP itself proves infeasible, giving clear diagnostics when investment is mandatory.
  \item \emph{Computational complexity:} the problem is now NP-hard; however, the binary block created by~\eqref{eq:lifetime} is totally unimodular,\footnote{Sliding-sum constraints form a consecutive-ones matrix; modern branch-and-cut solvers exploit this structure effectively.} so CPLEX finds optimal solutions in minutes for realistic instances ($\mathcal{O}(10^5)$ continuous variables, $\le10^3$ binaries).
\end{itemize}

\subsubsection{Outcome}
By migrating from LP to MILP, \texttt{vt2} integrates capacity expansion, replacement planning and hourly dispatch
in a \emph{single, coherent} optimisation layer, enabling least-cost road-maps rather than ex-post scenario scoring.


\subsection{Forecasting module}
\label{sec:forecasting-module}

The forecasting module is new and is based on a statistical and machine learning models 
that predicts the profile of a given generation asset.






\newpage