\newpage
\section{Methodology}

as mentioned before, the code base has been upgraded to a multi-year MILP and a new
forecasting module. The codebase relies on that strucutre hence the methodology used.

The change on the code base happen mostly between these three blocks:
% \begin{enumerate}
% \item \textbf{Data layer} – \texttt{data/} – Reads network configuration and generation/load time-series
% \item \textbf{Investment model} – \texttt{scripts/} – Solves and optimises the investment problem for a given grid
% \item \textbf{Forecasting module} – \texttt{forecast/} – Builds a day-ahead prediction based on statistical and machine learning models
% \end{enumerate}
\begin{figure}[h!]
  \centering
    \begin{verbatim}
        investment-model/
          +-- data/                       # data layer
          |   \-- ...
          |
          +-- forecast/                   # day-ahead forecasting module
          |   \-- ...
          |
          \-- scripts/                    # MILP investment model
          |   \-- ...
          ...
    \end{verbatim}
  \caption{Code architecture - showing the main components: data layer for grid and time-series inputs, forecasting module, and optimization scripts.}
  \label{fig:code-arch}
\end{figure}

\subsection{Data layer}
\label{sec:data-layer}

The data layer looks similar to the previous by splitting static part (network topology and the assets specifications)
and asset profiles. (defining profiles for the demand and assets capabilities) 
However, switching to a multi-year MILP removed the need for per-scenario network files !
hence creation of a new configuration file \texttt{analysis.json}, which defines the parameters of the investment problem (optimization) 
such as the planning horizon, annual load growths, and representative weeks selection.

In the new version the static grid data files have been simplified, still replicating the
metadata of the previous version using the matlab method (SEARCH FOR REF) but only the necessary 
columns were kept -> made use to clean 
Representative-week slicing (three 168h blocks) is carried over from the LP version and therefore not
described here again.

\begin{figure}[h!]
  \centering
    \begin{verbatim}
        data/         
          +-- grid/                       
          |   +-- analysis.json           # configuration file : horizon, growth, ...
          |   +-- buses.csv               # network topology
          |   +-- generators.csv          # assets specifications, lifetime and CAPEX
          |   +-- lines.csv               # network topology
          |   +-- loads.csv               # demand profile
          |   \-- storages.csv            # assets specifications, lifetime and CAPEX
          +-- processed/                  
              +-- load-2023.csv           # load time-series
              +-- solar-2023.csv          # solar time-series
              \-- wind-2023.csv           # wind time-series
    \end{verbatim}
  \caption{Data layer - showing the main components: static grid data and generation and load profiles.}
  \label{fig:data-layer}
\end{figure}

\subsection{Linear to Mixed-Integer Programming Transition}
\label{sec:milp-upgrade}

Below is a concise-but-thorough “diff” of the two generations of the code base, written from an optimisation \& applied-mathematics standpoint.

⸻

\paragraph{1. Big-picture methodology shift}

\begin{center}
\begin{tabular}{|l|l|l|}
\hline
Aspect & vt1 (“OLD”) & scripts (“NEW”) \\
\hline
Problem class & Pure LP (continuous), single representative week per season, one year at a time. & Full MILP (binary + continuous) that jointly decides multi-year build / replacement and hourly dispatch over 3 representative seasons. \\
Decision space & Dispatch – generation, line flows, storage charge/ discharge. Asset mix is exogenous (read from scenarios\_parameters.csv). & Dispatch and: • gen\_build[g,y], storage\_build[s,y] (0/1 “invest this year?”) • gen\_installed[g,y], storage\_installed[s,y] (0/1 “operational?”) \\
Temporal layers & One 168-h week × 3 seasons ⇒ cost scaled by \{13, 13, 26\}. & Same seasonal sampling plus a planning horizon years = [1,…,Y]; load can grow by factor growth[y]. \\
Cost handling & Variable cost only ( generator gencost · p) + tiny cycling penalty for storage. Investment analysis happens post-hoc in a separate CSV/NPV script. & Unified objective =$\sum$ season weight · MC · p  +  $\sum$ annualised\_capex(asset) · installed.  Annualisation uses Capital-Recovery Factor (CRF) internally, so discounting is baked into the linear coefficients – no external spreadsheet needed. \\
Solver / package & PuLP + CBC. & CVXPY 1.5 + CPLEX (mixed-integer). \\
Scenario management & “Run many LPs with different inputs; compare afterwards.” & “Solve one MILP that chooses the best build path; derive plan \& plots afterwards.” \\
\hline
\end{tabular}
\end{center}

⸻

\paragraph{2. New variables and why they matter}

Let
\begin{itemize}
    \item $G$ = set of candidate generators,
    \item $S$ = set of candidate storage units,
    \item $Y = \{1\dots Y\}$,
    \item $\Sigma$ = seasons.
\end{itemize}

\begin{center}
\begin{tabular}{|l|l|l|l|}
\hline
Variable & Type & Size & Interpretation \\
\hline
gen\_build[g,y] & binary & $|G|\cdot|Y|$ & 1 $\Leftrightarrow$ unit $g$ is first commissioned (or replaced) in year $y$. \\
gen\_installed[g,y] & binary & $|G|\cdot|Y|$ & 1 $\Leftrightarrow$ unit $g$ is available in year $y$. Determined by sliding-sum of gen\_build. \\
p\_gen[g,y,s,t] & $\geq 0$ & ... & Dispatch (MW). Capped by $p_{nom} \cdot$ installed and by weather profile for wind/solar. \\
p\_line[$\ell$,y,s,t] & free & ... & DC line flow (MW). \\
p\_charge, p\_discharge, soc[s,y,s, t] & $\geq 0$ & ... & Storage power \& state of charge. \\
\hline
\end{tabular}
\end{center}

Storage variables existed before; what is new is the $\pm$year index and the installation gating by storage\_installed[s,y].

⸻

\paragraph{3. Key new constraints (mathematical form)}

Below $\Delta$ denotes what is new vs. vt1.

\textbf{3.1 Build / installed linking} $\Delta$

For every asset $a$ with lifetime $L_a$ (years) and for every target year $y$:
\[
\text{installed}[a,y]  = \sum_{y' \leq y,\, y - y' < L_a}   \text{build}[a,y']
\]
\[
\sum_{y' \leq y,\, y - y' < L_a}   \text{build}[a,y']  \leq 1 \quad \text{(no double-build within life window)}
\]

Gives a neat “at-most-one build inside the sliding lifetime window” logic in two linear rows.

\textbf{3.2 Capacity coupling} $\Delta$
\[
\begin{align*}
p\_gen[g,y,s,t] &\leq p\_nom_g \cdot \text{profile}[g,s,t] \cdot \text{installed}[g,y] \\
p\_charge[s,y,s,t]      &\leq p\_nom_s \cdot \text{installed}[s,y] \\
p\_discharge[s,y,s,t]   &\leq p\_nom_s \cdot \text{installed}[s,y] \\
soc[s,y,s,t]           &\leq p\_nom_s \cdot \text{max\_hours} \cdot \text{installed}[s,y]
\end{align*}
\]

\textbf{3.3 Storage dynamics (unchanged except for year index)}
\[
\begin{align*}
soc_{t+1} &= soc_t + \eta_{in}\,p\_charge_t  -  (1/\eta_{out})\,p\_discharge_t \\
soc_0  &= 0 \\
soc_T &\leq 0.1 \cdot E_{nom} \cdot \text{installed}
\end{align*}
\]

Seasonal SoC is forced to start (and loosely end) at 0, removing cross-season coupling (vt1 required cyclical SoC equality).

\textbf{3.4 Nodal power balance (extended)}

For each bus $b$, season $s$, year $y$, time $t$:
\[
\sum_{g\in G_b} p\_gen[g,y,s,t]
+ \sum_{s\in S_b} (p\_discharge - p\_charge)[s,y,s,t]
+ \sum_{\ell\in In_b} p\_line[\ell,y,s,t]
= \text{growth}[y]\cdot\text{Load}[b,s,t]
+ \sum_{\ell\in Out_b} p\_line[\ell,y,s,t]
\]

(vt1 had the same balance but without the growth multiplier and without year index).

\textbf{3.5 Objective (linearised annuity cost)} $\Delta$
\[
\min \; \underbrace{\sum_{s\in\Sigma} W_s \sum_{y\in Y}\sum_{g,t} c_g\,p_{g,y,s,t}}{\text{operational}}
\;+\;
\underbrace{\sum_{y\in Y}\bigl(\sum_{g} \text{CRF}_g\cdot\text{CapEx}_g\cdot\text{installed}_{g,y}
+ \sum_{s} \text{CRF}_s\cdot\text{CapEx}_s\cdot\text{installed}_{s,y}\bigr)}_{\text{annualised capital}}
\]

No slack / load-shedding term – unmet demand is infeasible.

⸻

\paragraph{4. What disappeared}

\begin{center}
\begin{tabular}{|l|l|}
\hline
Old element & Reason for removal / replacement \\
\hline
Scenario CSV controlling asset mix & Investment is now endogenous via binary variables. \\
Separate NPV/annuity spreadsheet & Annuity built straight into objective through CRF $\times$ installed. \\
Final-SoC = initial constraint per season & Replaced by $\rightarrow$ soc$_0 = 0$,  soc$_T \leq 10\%$ cap. Eliminates hard coupling that made some seasons infeasible. \\
Slack variables (load\_shedding) & Model now enforces demand exactly (power-balance equality). \\
Storage “must finish with same SoC” & See above. \\
Discounting of OpEx & Dropped (weights cover season length only); simplifies objective coefficients. \\
\hline
\end{tabular}
\end{center}

⸻

\paragraph{5. Impact on solvability \& scale}
\begin{itemize}
    \item \textbf{Problem size:} Binaries: $|G|+ |S|$ per year – small relative to dispatch variables.
    \item \textbf{Continuous vars:} identical order of magnitude as before $\times |Y|$.
    \item \textbf{Complexity jump:} MILP is NP-hard vs. LP polynomial. However, lifetime \& build constraints are totally unimodular “staircase” structures $\rightarrow$ usually easy cuts for CPLEX.
    \item \textbf{Numerical tightness:} Removing slack makes the model brittle if profiles \& capacities cannot cover peak load. The “$\leq 0.1 E_{nom}$” final-SoC relaxation avoids cyclic infeasibility due to net-export / net-import imbalance inside a season.
    \item \textbf{Interpretability:} With installed extracted per asset/year, the post-processor can draw Gantt-style timelines and compute replacement patterns exactly (see plot\_implementation\_timeline).
\end{itemize}

⸻

\paragraph{6. Summary for practitioners}

The new framework internalises what used to be a Monte-Carlo-over-scenarios exercise into a single MILP.
Binary “build” decisions, lifetime-aware replacement, annuitised capex and load-growth scaling are now first-class constraints. Everything else – DC power flow, storage physics, seasonal weighting – is inherited from the old LP.

For a planning team this means:
\begin{itemize}
    \item One optimisation run = one least-cost expansion plan, instead of manual enumeration of candidate mixes.
    \item Costs are directly comparable across assets thanks to CRF-based linearisation.
    \item No external NPV spreadsheets; sensitivities (discount rate, lifetime) change only numerical coefficients, not model structure.
    \item Infeasibility becomes a diagnostic – you see immediately when the candidate fleet cannot meet future demand without new builds.
\end{itemize}

From a mathematical optimisation viewpoint the step from LP $\Rightarrow$ MILP is moderate in size yet profound in capability: the added 0/1 variables permit lumpy investment choices while preserving a linear objective and constraints, so commercial solvers (CPLEX, Gurobi) remain efficient on realistic medium-scale instances.

⸻

\subsection{Forecasting module}
\label{sec:forecasting-module}

The forecasting module is new and is based on a statistical and machine learning models 
that predicts the profile of a given generation asset.






\newpage