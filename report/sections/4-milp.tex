\newpage
\section{Linear to Mixed-Integer Programming Transition}
\label{sec:MILP_transition}

This section builds directly on the DC Optimal Power Flow (DC-OPF) formulation and storage constraints presented in the previous semester project (“\href{https://github.com/vierui/vt1-energy-investment-model/blob/master/report/main.pdf}
{[vt1]})”). While the core network and dispatch logic remain intact, we extend the model to include multi-year investment decisions using a Mixed-Integer Linear Programming (MILP) formulation. To fully appreciate this section, readers are encouraged to revisit the previous report, where the LP formulation is detailed.

The MILP formulation introduces binary variables to capture investment timing and operational availability of assets, enabling the model to answer:
> “When should we build or replace which generator or storage asset to meet future demand at minimal cost?”

The formulation maintains the original DC-OPF structure (Section~\ref{sec:dcopf}) and storage dynamics (Section~\ref{sec:storage}), but modifies and adds constraints to represent investment logic. The main changes are detailed below.

\subsection{Methodology logic}
\label{ssec:MILP_methodology}

\begin{enumerate}[label=\textbf{\roman*.}]
    \item \textbf{Binary build and availability variables}\\ 
          Two $0/1$ variables are introduced for every asset $a$ (generator or storage) and planning year $y$:
          \[
             \underbrace{b_{a,y}}_{\text{build}} = 
             \begin{cases}1 &\text{if the unit is \emph{commissioned} in }y,
                \\0&\text{otherwise,}\end{cases}
             \qquad
             \underbrace{i_{a,y}}_{\text{installed}} = 
             \begin{cases}1 &\text{if the unit is \emph{operational} in }y,
                \\0&\text{otherwise.}\end{cases}
          \]
          These variables replace the fixed scenario-based asset list from the vt1 model.

    \item \textbf{Lifetime-aware build–installed coupling}\\  
          Each asset lives exactly $L_a$ years.  
          A sliding-window rule (see Eq.~\eqref{eq:installed_def_new}) links the two binaries 
          while preventing “double builds” inside its lifetime window 
          (Eq.~\eqref{eq:one_build_new}).

    \item \textbf{Capacity activation via installation flag}\\  
          Generator dispatch $p^{\mathrm{gen}}$, storage power $(p^{\mathrm{ch}},
          p^{\mathrm{dis}})$ and energy $\mathrm{soc}$ are all \emph{multiplied} by the installation flag $i_{a,y}$; if a unit is not built, its capacity is mathematically zero.

    \item \textbf{Unified cost metric}\\  
          Capital expenditure (CAPEX) is annualized using the Capital Recovery Factor (CRF).  It is an asset-specific financial metric used to calculate the annualized cost of an investment over its lifetime.
          The goal is to make investment and operational costs comparable in a single linear objective. It is defined as:
          \begin{equation}
            \mathrm{CRF}_a = \frac{i\,(1+i)^{L_a}}{(1+i)^{L_a}-1},
          \end{equation}
          where $i$ is the interest rate and $L_a$ the lifetime of the asset.

    \item \textbf{Demand growth inside the model}\\  
          To mimic a time-varying load through the years, the loads are scaled within the nodal-balance by a defined 
          factor $\gamma_y$. This induces a growing demand in the model without increasing the generation. 
          This should induce a diversification of the generation mix through the years.

\end{enumerate}

\subsection{Mathematical Implementation}
\label{ssec:MILP_implementation}

For clarity the modifications are grouped into  
\textbf{(A)~MILP-specific constraints} and  
\textbf{(B)~code-level modelling improvements}.  
Classical DC-OPF balances and line limits remain unchanged. \cite{wood2013power}

\subsubsection*{A.   MILP integration new constraints}

\begin{itemize}
    \item \textbf{Build–Installed coupling} (NEW)\\
    Once built, a unit stays alive for $L_a$ years; build at most once during that life.
        
        \begin{equation}
            \setlength{\arraycolsep}{2em}
            \begin{array}{rlrl}
                i_{a,y} = \displaystyle \sum_{\substack{y' \le y \\ y - y' < L_a}} b_{a,y'} 
                & \displaystyle \sum_{\substack{y' \le y \\ y - y' < L_a}} b_{a,y'} \le 1
            \end{array}
            \tag{2a, 2b}
            \label{eq:build_install_new}
        \end{equation}
        \addtocounter{equation}{1}  % skips the unused (1) makes the next auto one (2)
        In plain terms:
        \begin{itemize}
            \item the first equation says “asset is operational in year $y$ if built in one of the last $L_a$ years.”  
            \item the second equation says “you can build or replace at most once per lifetime.”
        \end{itemize}


    \item \textbf{Generator Output Limits} (MODIF.)\\
    blablabla
    \begin{equation}
    0 \leq P^{\text{gen}}_{g,y,s,t} 
    \leq P^{\text{nom}}_{g} \cdot \cdot i_{g,y}
    \end{equation}

    \item \textbf{Storage Power Limits} (MODIF.)\\
    blablabla
    \begin{equation}
    \begin{aligned}
    0 &\leq P^{\text{ch}}_{s,y,s,t} \leq P^{\text{nom}}_{s} \cdot i_{s,y} \\
    0 &\leq P^{\text{dis}}_{s,y,s,t} \leq P^{\text{nom}}_{s} \cdot i_{s,y}
    \end{aligned}
    \end{equation}

    \item \textbf{Energy Capacity Limit} (MODIF.)\\
    blablabla
    \begin{equation}
    0 \leq E_{s,y,s,t} \leq E^{\text{nom}}_{s} \cdot i_{s,y}
    \end{equation}
    
    These replace the vt1 constraints where capacities were fixed and exogenously defined.
    
\end{itemize}

\subsubsection*{B.  Additional modelling changes}

\begin{itemize}
    \item \textbf{Seasonal state-of-charge boundary} (MODIF.)\\
    The cycle-equality used in vt1 is relaxed.
    \begin{equation}
        \mathrm{soc}_{s,y,s,0}=0,\qquad
        0\;\le\;\mathrm{soc}_{s,y,s,T}\;\le\;0.1\,E^{\mathrm{nom}}_{s}\; i_{s,y}.
    \end{equation}
    The battery starts empty and may end anywhere within 10 \% of its energy capacity, avoiding cross-season infeasibility.

    \item \textbf{Nodal balance with load scaling} (MODIF.)\\
    For each bus $b$, season $s$, year $y$, hour $t$:
    \begin{equation}
        \sum_{g \in \mathcal{G}_b} P^{\text{gen}}_{g,y,s,t} +
        \sum_{s \in \mathcal{S}_b} (P^{\text{dis}}_{s,y,s,t} - P^{\text{ch}}_{s,y,s,t}) +
        \sum_{\ell \in \text{in}(b)} F_{\ell,y,s,t}
        = 
        \gamma_y \cdot D_{b,s,t} +
        \sum_{\ell \in \text{out}(b)} F_{\ell,y,s,t}
        \label{eq:nodal_new}
        \end{equation}

    The only change is the growth factor $\gamma_y$ on the demand term.
\end{itemize}


\subsubsection*{C.  Objective function with annualised CapEx}
\label{ssec:MILP_objective}

\begin{adjustwidth}{1.5em}{0pt}  % adjust indent here to match other items
The new cost function minimises both dispatch cost and annualised investment:

\begin{equation}
    \min \quad
    \underbrace{\sum_{s \in \Sigma} W_s \sum_{y,t,g} c_g \cdot P^{\text{gen}}_{g,y,s,t}}_{\substack{\text{Operating Costs} \\ \text{OPEX}}}
    +
    \underbrace{\sum_{y}\Bigl(
    \sum_{a\in\mathcal{A}}
        A_a \;\cdot i_{a,y}\Bigr)}_{\substack{\text{Investment Costs} \\ \text{CAPEX}}}
\end{equation}

With:
\begin{itemize}
    \item $W_s$: Number of calendar weeks represented by each season $s$ (e.g., winter = 13)
    \item $A_a = \mathrm{CRF}_a \cdot C^{\text{cap}}_a$: Annualised CapEx per asset $a$
\end{itemize}

Hence the MILP simultaneously finds the least-cost \emph{dispatch} and the cheapest \emph{build / replace} schedule over the planning horizon. It eliminates the need for external spreadsheets computing NPV and manual scenarios comparison.
\end{adjustwidth}

\subsection{Code Implementation}
\label{ssec:MILP_code}

The new MILP layer is achieved with **minimal churn** to the original vt1 code:  
data ingestion and the DC-OPF core are reused almost verbatim, while a handful of
Python modules inject the binary-investment logic.  
Figure~\ref{fig:milp_pipeline} sketches the end-to-end data flow.

\begin{figure}[H]
\centering
%\includegraphics[width=.85\textwidth]{images/milp_pipeline.pdf}
\caption{From CSV to optimisation results: the vt2 MILP pipeline.}
\label{fig:milp_pipeline}
\end{figure}

\subsubsection*{A.  File roles and interactions}

\begin{center}\small
\begin{tabular}{p{3cm}p{9cm}}
\hline
\textbf{File / Module} & \textbf{What it does (plain English)} \\ 
\hline
\texttt{pre.py} & Reads raw CSVs, slices a \emph{typical} week per season, returns a Python \texttt{dict}. \\
\texttt{network.py} & Two “data-only” classes:  
\texttt{Network} (one season) and \texttt{IntegratedNetwork} (many seasons + years).  
They hold Pandas frames but no optimisation code. \\
\texttt{optimization.py} & Builds the MILP with CVXPY \(\rightarrow\) CPLEX, extracts all variable
values into a flat JSON-friendly dictionary. \\
\texttt{main.py} & Command-line front-end: calls \texttt{pre}\(\rightarrow\)\texttt{Network}\(\rightarrow\)\texttt{optimization},
then hands results to the plotting/post modules. All logging is wired here. \\
\texttt{post.py} & Turns the flat result dictionary into a Gantt chart, seasonal plots, and an implementation-plan JSON.\\
\hline
\end{tabular}
\end{center}

\subsubsection*{B.  “Smart” touches in the MILP code}

\begin{itemize}
    \item \textbf{Lifetime window in two rows.}  
          Listing~\ref{lst:lifetime} builds the \emph{entire} sliding window once; CPLEX then recognises the
          resulting matrix block as totally unimodular, so branch-and-bound cuts quickly prune binaries.
          \begin{lstlisting}[language=Python, caption={Lifetime coupling snippet}, label={lst:lifetime}]
window = [b[(a,yb)] for yb in years if 0 <= y - yb < L_a]
constraints += [i[(a,y)] == cp.sum(window), cp.sum(window) <= 1]
          \end{lstlisting}
    \item \textbf{Vector variables, not 8 760 scalars.}  
          One CVXPY vector per $(asset, year, season)$ packs all 168 hourly values.  
          Memory drops from \(\mathcal{O}(T\! \times\! \text{assets})\) scalars to a handful of numpy blocks.
    \item \textbf{Pre-computed bus dictionaries.}  
          Look-ups like “which generators sit on bus 5 in summer?” are done \emph{once} in Python, then stored as lists; inner power-balance loops just iterate these lists → no expensive Pandas access.
    \item \textbf{Relaxed seasonal SoC.}  
          Replacing the hard $E_{T}=E_{0}$ with a “$\leq 10 \%$” band avoids infeasibility when a median week has net import or export; yet the inequality is still linear, so the LP relaxation stays tight.
    \item \textbf{Capital-Recovery baked in.}  
          Annual cost $A_a = \mathrm{CRF}_a \cdot C^{\text{cap}}_a$ is computed \emph{before} it reaches the solver.
          Discount-rate sensitivity therefore changes only coefficients, not the constraint matrix.
    \item \textbf{Flat-key result map.}  
          Each optimisation variable value is exported under a self-describing string  
          \texttt{p\_gen\_\{season\}\_\{asset\}\_\{year\}\_\{hour\}},  
          making downstream analysis possible with nothing more than \texttt{dict.get()}.
\end{itemize}

\subsubsection*{C.  Solver migration: CBC \(\rightarrow\) CPLEX}

\begin{itemize}
    \item \emph{Why CPLEX?}  We now have \(\approx 700\) binary variables; CBC struggled beyond 1–2 years,
          whereas CPLEX’s presolve and cut families solve a 10-year horizon in \(< 30\) s on a laptop.
    \item \emph{Parameter hand-off.}  All tuning lives in one dictionary:  
          \texttt{\{'threads':10, 'timelimit':1080, 'mipgap':0.01\}} — users override via the CLI
          \texttt{--solver-options '{"timelimit":3600}'}, no code edits needed.
    \item \emph{Warm-start ready.}  Because binaries only gate capacities, a feasible LP point is trivial
          (\(b=i=0\)).  If ever needed, this could be passed to CPLEX to cut root-node time further.
\end{itemize}

\subsubsection*{D.  Possible next tweaks}

\begin{enumerate}[label=\alph*)]
    \item Replace the looped SoC update with a single sparse Toeplitz matrix product to
          accelerate CVXPY canonicalisation.  
    \item Add a lazy-constraint callback so that line thermal limits are added on
          demand, reducing root-LP size for large grids.  
    \item Provide an \texttt{xarray} wrapper around the flat dict for more ergonomic
          analysis in Python notebooks.
\end{enumerate}

\vspace{0.5em}
\noindent
\textbf{Take-away.}  
With ca.~600 lines of fresh code the project jumps from an LP “dispatch calculator” to a
multi-year investment MILP while preserving readability and run-time efficiency —
exactly the sort of incremental upgrade one expects in professional optimisation
software development.