\newpage
\section{Linear to Mixed-Integer Programming Transition}
\label{sec:MILP_transition}

This section builds directly on the DC Optimal Power Flow (DC-OPF) formulation and storage constraints presented in the previous semester project (“\href{https://github.com/vierui/vt1-energy-investment-model/blob/master/report/main.pdf}
{[vt1]})”). While the core network and dispatch logic remain intact, we extend the model to include multi-year investment decisions using a Mixed-Integer Linear Programming (MILP) formulation. To fully appreciate this section, readers are encouraged to revisit the previous report, where the LP formulation is detailed.

The MILP formulation introduces binary variables to capture investment timing and operational availability of assets, enabling the model to answer:
> “When should we build or replace which generator or storage asset to meet future demand at minimal cost?”

The formulation maintains the original DC-OPF structure (Section~\ref{sec:dcopf}) and storage dynamics (Section~\ref{sec:storage}), but modifies and adds constraints to represent investment logic. The main changes are detailed below.

\subsection{Methodology logic}
\label{ssec:MILP_methodology}

\begin{enumerate}[label=\textbf{\roman*.}]
    \item \textbf{Binary build and availability variables}\\ 
          Two $0/1$ variables are introduced for every asset $a$ (generator or storage) and planning year $y$:
          \[
             \underbrace{b_{a,y}}_{\text{build}} = 
             \begin{cases}1 &\text{if the unit is \emph{commissioned} in }y,
                \\0&\text{otherwise,}\end{cases}
             \qquad
             \underbrace{i_{a,y}}_{\text{installed}} = 
             \begin{cases}1 &\text{if the unit is \emph{operational} in }y,
                \\0&\text{otherwise.}\end{cases}
          \]
          These variables replace the fixed scenario-based asset list from the vt1 model.

    \item \textbf{Lifetime-aware build–installed coupling}\\  
          Each asset lives exactly $L_a$ years.  
          A sliding-window rule (see Eq.~\eqref{eq:installed_def_new}) links the two binaries 
          while preventing “double builds” inside its lifetime window 
          (Eq.~\eqref{eq:one_build_new}).

    \item \textbf{Capacity activation via installation flag}\\  
          Generator dispatch $p^{\mathrm{gen}}$, storage power $(p^{\mathrm{ch}},
          p^{\mathrm{dis}})$ and energy $\mathrm{soc}$ are all \emph{multiplied} by the installation flag $i_{a,y}$; if a unit is not built, its capacity is mathematically zero.

    \item \textbf{Unified cost metric}\\  
          Capital expenditure (CAPEX) is annualized using the Capital Recovery Factor (CRF).  It is an asset-specific financial metric used to calculate the annualized cost of an investment over its lifetime.
          The goal is to make investment and operational costs comparable in a single linear objective. It is defined as:
          \begin{equation}
            \mathrm{CRF}_a = \frac{i\,(1+i)^{L_a}}{(1+i)^{L_a}-1},
          \end{equation}
          where $i$ is the interest rate and $L_a$ the lifetime of the asset.

    \item \textbf{Demand growth inside the model}\\  
          To mimic a time-varying load through the years, the loads are scaled within the nodal-balance by a defined 
          factor $\gamma_y$. This induces a growing demand in the model without increasing the generation. 
          This should induce a diversification of the generation mix through the years.

\end{enumerate}

\subsection{Mathematical Implementation}
\label{ssec:MILP_implementation}

For clarity the modifications are grouped into  
\textbf{(A)~MILP-specific constraints} and  
\textbf{(B)~code-level modelling improvements}.  
Classical DC-OPF balances and line limits remain unchanged. \cite{wood2013power}

\subsubsection*{A.   MILP integration new constraints}

\begin{itemize}
    \item \textbf{Build–Installed coupling} (NEW)\\
    Once built, a unit stays alive for $L_a$ years; build at most once during that life.
        
        \begin{equation}
            \setlength{\arraycolsep}{2em}
            \begin{array}{rlrl}
                i_{a,y} = \displaystyle \sum_{\substack{y' \le y \\ y - y' < L_a}} b_{a,y'} 
                & \displaystyle \sum_{\substack{y' \le y \\ y - y' < L_a}} b_{a,y'} \le 1
            \end{array}
            \tag{2a, 2b}
            \label{eq:build_install_new}
        \end{equation}
        \addtocounter{equation}{1}  % skips the unused (1) makes the next auto one (2)
        In plain terms:
        \begin{itemize}
            \item the first equation says “asset is operational in year $y$ if built in one of the last $L_a$ years.”  
            \item the second equation says “you can build or replace at most once per lifetime.”
        \end{itemize}


    \item \textbf{Generator Output Limits} (MODIF.)\\
    blablabla
    \begin{equation}
    0 \leq P^{\text{gen}}_{g,y,s,t} 
    \leq P^{\text{nom}}_{g} \cdot \cdot i_{g,y}
    \end{equation}

    \item \textbf{Storage Power Limits} (MODIF.)\\
    blablabla
    \begin{equation}
    \begin{aligned}
    0 &\leq P^{\text{ch}}_{s,y,s,t} \leq P^{\text{nom}}_{s} \cdot i_{s,y} \\
    0 &\leq P^{\text{dis}}_{s,y,s,t} \leq P^{\text{nom}}_{s} \cdot i_{s,y}
    \end{aligned}
    \end{equation}

    \item \textbf{Energy Capacity Limit} (MODIF.)\\
    blablabla
    \begin{equation}
    0 \leq E_{s,y,s,t} \leq E^{\text{nom}}_{s} \cdot i_{s,y}
    \end{equation}
    
    These replace the vt1 constraints where capacities were fixed and exogenously defined.
    
\end{itemize}

\subsubsection*{B.  Additional modelling changes}

\begin{itemize}
    \item \textbf{Seasonal state-of-charge boundary} (MODIF.)\\
    The cycle-equality used in vt1 is relaxed.
    \begin{equation}
        \mathrm{soc}_{s,y,s,0}=0,\qquad
        0\;\le\;\mathrm{soc}_{s,y,s,T}\;\le\;0.1\,E^{\mathrm{nom}}_{s}\; i_{s,y}.
    \end{equation}
    The battery starts empty and may end anywhere within 10 \% of its energy capacity, avoiding cross-season infeasibility.

    \item \textbf{Nodal balance with load scaling} (MODIF.)\\
    For each bus $b$, season $s$, year $y$, hour $t$:
    \begin{equation}
        \sum_{g \in \mathcal{G}_b} P^{\text{gen}}_{g,y,s,t} +
        \sum_{s \in \mathcal{S}_b} (P^{\text{dis}}_{s,y,s,t} - P^{\text{ch}}_{s,y,s,t}) +
        \sum_{\ell \in \text{in}(b)} F_{\ell,y,s,t}
        = 
        \gamma_y \cdot D_{b,s,t} +
        \sum_{\ell \in \text{out}(b)} F_{\ell,y,s,t}
        \label{eq:nodal_new}
        \end{equation}

    The only change is the growth factor $\gamma_y$ on the demand term.
\end{itemize}


\subsubsection*{C.  Objective function with annualised CapEx}
\label{ssec:MILP_objective}

\begin{adjustwidth}{1.5em}{0pt}  % adjust indent here to match other items
The new cost function minimises both dispatch cost and annualised investment:

\begin{equation}
    \min \quad
    \underbrace{\sum_{s \in \Sigma} W_s \sum_{y,t,g} c_g \cdot P^{\text{gen}}_{g,y,s,t}}_{\substack{\text{Operating Costs} \\ \text{OPEX}}}
    +
    \underbrace{\sum_{y}\Bigl(
    \sum_{a\in\mathcal{A}}
        A_a \;\cdot i_{a,y}\Bigr)}_{\substack{\text{Investment Costs} \\ \text{CAPEX}}}
\end{equation}

With:
\begin{itemize}
    \item $W_s$: Number of calendar weeks represented by each season $s$ (e.g., winter = 13)
    \item $A_a = \mathrm{CRF}_a \cdot C^{\text{cap}}_a$: Annualised CapEx per asset $a$
\end{itemize}

Hence the MILP simultaneously finds the least-cost \emph{dispatch} and the cheapest \emph{build / replace} schedule over the planning horizon. It eliminates the need for external spreadsheets computing NPV and manual scenarios comparison.
\end{adjustwidth}

%----------------------------------------------------------
\subsection{Implementation Details (MILP pipeline)}
\label{ssec:milp_impl}
%----------------------------------------------------------
%
% This subsection mirrors the structure used for forecasting
% (\S\ref{ssec:impl_details}), but now focuses on the
% mixed-integer linear optimisation workflow that sizes,
% installs and dispatches power-system assets over a
% multi-year horizon.

\subsubsection*{A.\ Overview and roles}

\begin{figure}[h!]
      \centering
      \begin{tikzpicture}[
          node distance=1.2cm and 1.6cm,
          every node/.style={font=\small, rounded corners},
          io/.style={draw, fill=blue!20, minimum height=1cm, minimum width=2.1cm, align=center},
          main/.style={draw, fill=green!20, minimum height=1cm, minimum width=2.6cm, align=center},
          opti/.style={draw, fill=orange!20, minimum height=1cm, minimum width=2.6cm, align=center},
          output/.style={draw, fill=purple!20, minimum height=1cm, minimum width=2.6cm, align=center},
          data/.style={draw, fill=gray!10, minimum height=1cm, minimum width=2.6cm, align=center},
          arrow/.style={->, thick}
      ]
  
      % Nodes
      \node[io] (grid) {grid/\\\scriptsize grid config};
      \node[main, right=of grid] (main) {main.py\\\scriptsize orchestration};
      \node[opti, right=of main, text=red] (opti) {scripts/\\\scriptsize Optimization framework};
      \node[data, below=1.2cm of opti] (data) {data/\\\scriptsize generation and loads profiles};
      \node[output, right=of opti] (out) {outputs/\\\scriptsize plots, metrics, predictions, results};
  
      % Arrows
      \draw[arrow] (grid) -- (main);
      \draw[arrow] (data.north) -- (opti.south);
      \draw[arrow] (main) -- (opti);
      \draw[arrow] (opti) -- (out);
  
      \end{tikzpicture}
      \caption{Compact overview of the forecasting pipeline.}
      \label{fig:forecast_flow_compact}
  \end{figure}

\begin{itemize}
    \item \texttt{main.py} –\- **driver script** accepting CLI flags, steering the four logical stages:
          preprocessing, network assembly, optimisation, and post-processing/logging.  
    \item \texttt{pre.py} –\- slices three \mbox{168-h} representative weeks, matches
          profiles to assets, and attaches analysis meta-data (years, season weights, load-growth).
    \item \texttt{optimization.py} –\- builds the **DC-OPF MILP** with annualised CAPEX,
          solves it via \textsc{Cplex}, then serialises all variables back into the
          \texttt{IntegratedNetwork}.
    \item \texttt{post.py} –\- turns the raw decision variables into human-readable
          implementation plans, generation-mix graphics and asset timelines.
    \item \texttt{analysis/production\_costs.py} –\- converts dispatch into MWh,
          adds annuitised investment streams, and prints/plots per-asset
          cost breakdowns.
    \item \texttt{results/} –\- central drop-zone for logs, JSON artefacts and figures.
\end{itemize}

\subsubsection*{B.\ Implementation}

Figure~\ref{fig:scripts_block} shows the deeper call graph inside the
\texttt{scripts/} package, highlighting the \emph{three}
execution phases:

\begin{figure}[h!]
    \centering
    \begin{tikzpicture}[
        node distance=1.2cm and 1.2cm,
        every node/.style={font=\small, rounded corners},
        io/.style={draw, fill=green!20, minimum height=1cm, minimum width=2cm, align=center},
        phase/.style={draw, fill=orange!40, minimum height=1cm, minimum width=2.8cm, align=center},
        sub/.style={draw, fill=pink!10, minimum height=1cm, minimum width=3.0cm, align=center},
        helper/.style={draw, fill=orange!5,  minimum height=1cm, minimum width=3.0cm, align=center},
        output/.style={draw, fill=purple!20, minimum height=1cm, minimum width=2.6cm, align=center},
        data/.style={draw, fill=gray!10, minimum height=1cm, minimum width=2.6cm, align=center},
        arrow/.style={->, thick},
        looplabel/.style={font=\scriptsize\itshape}
    ]

    % Nodes -----------------------------------------------
    \node[phase] (prep)      {pre.py\\\scriptsize \#1 Pre-process};
    \node[io, below=of prep, yshift=-4.4cm] (raw) {main.py};
    \node[phase, right=of prep] (net)   {network.py\\\scriptsize \#2 Build objects};
    \node[phase, right=of net]  (milp)  {optimization.py\\\scriptsize \#3 Solve MILP};
    \node[phase, right=of milp] (post)  {post.py\\\scriptsize \#5 Post-proc};
    \node[sub, below=of milp] (integ) {IntegratedNetwork()};
    \node[phase, below=of integ] (analysis) {costs.py \\\scriptsize \#4 Costs analysis};
    \node[output, below=of analysis] (out) {results/};
    \node[data, above=of prep] (data) {data/};

      % data inputs
      \draw[arrow, dashed] (raw) -- (prep);
      \draw[arrow, dashed] (raw) |- (analysis.west);
      \draw[arrow] (data) -- (prep);

      % build & optimisation flow
      \draw[arrow] (prep) -- (net);
      \draw[arrow] (net)  -- (milp);
      \draw[arrow] (milp) -- (post);
      \draw[arrow] (milp) -- ++(0,-1.5) node[midway, left] {\scriptsize writes} -- (integ.north);
      \draw[arrow] (net.south) -- ++(0,-1.7) node[midway, left] {\scriptsize builds} -- (integ.west);
      \draw[arrow] (integ.south) -- (analysis.north);
      \draw[arrow] (post.south) |- ++(0,-6.14) -- (out.east);
      \draw[arrow] (analysis.south) -- (out.north);

    % Envelope
    \begin{pgfonlayer}{background}
        \node[draw=orange, thick, rounded corners, inner sep=0.4cm, 
              fit=(prep) (net) (milp) (post) (integ) (analysis), 
              label=above:{\textbf{\texttt{scripts/}}}] {};
    \end{pgfonlayer}
    \end{tikzpicture}
    \caption{Detailed flow inside \texttt{scripts/}: preprocessing \(\rightarrow\) object construction \(\rightarrow\) MILP solution \(\rightarrow\) reporting.}
    \label{fig:scripts_block}
\end{figure}



\begin{description}
    \item[Phase~1 – Pre-processing]  
          \texttt{pre.py} converts raw CSV/time-series into
          \texttt{grid\_data} \(+\) \texttt{seasons\_profiles}.  
          A light sanity-check now ensures the
          \texttt{representative\_weeks} sum to 52; otherwise a
          default \(13/13/26\) split is injected.

    \item[Phase 2 – Object assembly]  
          \texttt{network.py} wraps every season in a
          \texttt{Network} (data-only) and records them inside a
          global \texttt{IntegratedNetwork}.  Tweaks that proved
          essential:
          \begin{itemize}
              \item robust bus-ID matching (string vs.\ integer);
              \item automatic snapshot creation with length \(T\);
              \item load-growth factors attached for later scaling.
          \end{itemize}

    \item[Phase 3 – MILP formulation \& solve]  
          \texttt{optimization.py} hosts two core functions:
          \texttt{dcopf()} builds the model, while
          \texttt{investement\_multi()} (sic) solves, extracts and
          stores all variables.  Key modelling choices:

          \begin{itemize}
              \item \emph{at-most-one-build-per-lifetime} window  
                    \(\Rightarrow\) removes legacy multi-binary logic.
              \item Annualised CAPEX via CRF \(\bigl(\)\texttt{compute\_crf}\(\bigr)\);
                    operational costs weighted by season-weeks.
              \item No slack variables; nodal balances must close.
              \item Storage SoC forced to zero at each season edge,
                    breaking cross-season energy loops.
          \end{itemize}

    \item[Phase 4 – Post-processing]  
          \texttt{post.py} renders a Gantt-like timeline, seasonal
          generation mixes and an implementation plan, whereas
          \texttt{analysis/production\_costs.py} computes MWh and
          \$ flows, re-using the same CRF/discount helpers to stay
          consistent with the objective.
\end{description}

\subsubsection*{C.\ MILP model structure}

Figure~\ref{fig:milp_internal_flow} sketches the internal control
flow inside \texttt{dcopf()}.  It shows how build binaries cascade
into installed status, which then gate capacity and dispatch
variables before everything is funnelled into a single cost
function.

\begin{tikzpicture}[
      font=\small,
      node distance=1.4cm and 1.8cm,
      every node/.style={align=center, rounded corners, minimum height=1.0cm, draw},
      box/.style={fill=blue!10},
      arrow/.style={->, thick}
  ]
  % Nodes ------------------------------------------------
  \node[box] (zbuild)  {Create binaries\\ \(z_{\text{build}}\)};
  \node[box, right=of zbuild] (zinst)  {Link to\\ installed status\\ \(z_{\text{inst}}\)};
  \node[box, right=of zinst]  (cap)   {Capacity\\ constraints};
  \node[box, below=of cap]    (dispatch) {Dispatch vars\\ \((p_{\text{gen}},\,p_{\text{line}},\,p_{\text{stor}})\)};
  \node[box, left=of dispatch] (balance) {Nodal\\ balances};
  \node[box, right=of dispatch]    (cost) {Cost\\ function};
  \node[box, below=of cost]   (solve) {Solve with\\ \textsc{Cplex}};
  \node[box, below=of dispatch]  (extract) {Results\\ (JSON)};
  
  % Arrows ------------------------------------------------
  \draw[arrow] (zbuild) -- (zinst);
  \draw[arrow] (zinst) -- (cap);
  \draw[arrow] (cap) -- (dispatch.north);
  \draw[arrow] (balance) -- (dispatch.west);
  \draw[arrow] (zbuild.north) |- ++(0,0.5) -| (cost.north);      % CAPEX
  \draw[arrow] (dispatch.east) -- (cost.west);                   % OPEX
  \draw[arrow] (cost) -- (solve);
  \draw[arrow] (solve) -- (extract);
  
  % Background frame
  \begin{pgfonlayer}{background}
      \node[draw=orange, thick, rounded corners, inner sep=1cm, 
            fit=(zbuild) (zinst) (cap) (dispatch) (balance) (cost) (solve) (extract), 
            label=above:{\textcolor{orange}{\textbf{\texttt{Optimization workflow}}}}] {};
      \node[draw=grey!20, thick, rounded corners, inner sep=0.2cm, 
            fit=(zbuild) (zinst) (cap) (dispatch) (balance) (extract), 
            label=below:{\textcolor{gray}{\textbf{\texttt{IntegratedNetwork()}}}}] {};
  \end{pgfonlayer}
  \end{tikzpicture}

\bigskip
The resulting MILP stack therefore mirrors the earlier
forecasting-pipeline layout while obeying power-system specific
lifetimes, energy conservation and investment logic.