\newpage
\section{Linear to Mixed-Integer Programming Transition}
\label{sec:MILP_transition}
Building on last semester's investment model work~\cite{vierui2024vt1}, we now let the
model decide when to build or replace each asset.  To simply put it, we add binary variables 
to capture theassets availability and usage and define a new objective function 
balancing both investment and operational costs.

The previous linear (LP) formulation fixed the asset list ex-ante; any capacity
study meant running dozens of “what-if” scenarios offline.  In the MILP,
those scenarios collapse into one optimisation that jointly chooses
dispatch \emph{and} build schedule for the lowest net present cost. Running this logic over
multiple years, generates the possibility to assess when in a horizon, should an asset be built 
given its lifetime and associated cost.

\subsection{Key modelling ideas}
\label{ssec:MILP_methodology}

\begin{enumerate}[label=\textbf{\roman*}.]
  \item \textbf{Two binary variables per asset and year}\footnote{Upper-case sets:
        $\mathcal{A}$ assets, $\mathcal{Y}$ years, $\mathcal{B}$ buses.
        Lower-case indices: $a,\,y,\,b$ etc.}
        \begin{align*}
            z^{\text{build}}_{a,y} &= 
              \begin{cases}
                1 &\text{if asset $a$ is \emph{commissioned} in year $y$}\\
                0 &\text{otherwise}
              \end{cases},
            &
            z^{\text{on}}_{a,y} &= 
              \begin{cases}
                1 &\text{if asset $a$ is \emph{operational} in year $y$}\\
                0 &\text{otherwise.}
              \end{cases}
        \end{align*}

        Two binaries per asset per year are present in the model, but:
        \begin{itemize}
            \item Only the $z^{\text{build}}$ variables are truly “free” (decision variables).
            \item The $z^{\text{on}}$ variables are “dependent” binaries: each year’s installed status is determined (by constraint) as the sum of all unexpired builds.
        \end{itemize}

  \item \textbf{Lifetime-aware coupling}\\
        Each asset lives $L_a$ (lifetime) years.
        \begin{equation}
            \setlength{\arraycolsep}{2em}
            \begin{array}{rlrl}
                z^{\text{on}}_{a,y} = \displaystyle \sum_{\substack{y' \le y \\ y - y' < L_a}} z^{\text{build}}_{a,y'} 
                & \displaystyle \sum_{\substack{y' \le y \\ y - y' < L_a}} z^{\text{build}}_{a,y'} \le 1
            \end{array}
            \tag{2a, 2b}
            \label{eq:lifetime_link}
        \end{equation}
        \addtocounter{equation}{1}  % skips the unused (1) makes the next auto one (2)
        \noindent
        \begin{itemize}
            \item (a) says an asset is “on” if it was built in the last $L_a$ years
            \item (b) forbids more than one build : "at-most-one-build-per-lifetime"
            \item $y'$ is the year of the last build
        \end{itemize}

  \item \textbf{Capacity gating}\\
       Dispatch variables (Generator : $P^{\mathrm{gen}}$, storage power : $(P^{\mathrm{ch}},
       \text{and} P^{\mathrm{dis}})$ and state of charge : $E$) are multiplied by the installation flag $i_{a,y}$ .
       If an asset is not built, its capacity is mathematically zero.
       \[ 0 \leq X^{\text{asset}}_{a,y} \leq X^{\text{max}}_{a} \cdot z^{\text{on}}_{a,y} \]


  \item \textbf{Annualised CAPEX via Capital Recovery Factor (CRF)}\\
      Capital expenditure (CAPEX) is annualized using the Capital Recovery Factor (CRF).  It is an asset-specific financial metric used to calculate the annualized cost of an investment over its lifetime.
      The goal is to make investment and operational costs comparable in a single linear objective. It is defined as:
      \begin{equation}
      \text{CRF}_a \;=\; \frac{i\;(1+i)^{L_a}}{(1+i)^{L_a}-1},
      \qquad
      A_a \;=\; \text{CRF}_a \, \cdot\, C_a.
      \end{equation}
      \begin{itemize}
            \item $A_a$ is the \emph{annual} cost of owning one unit of asset $a$.
            \item $C_a$ is the \emph{capital cost} of asset $a$.
            \item $i$ is the \emph{interest rate} (depreciation rate between 5--10\%) 
      \end{itemize}
      
      The annulized cost brings a strong advantage in the context of a multi-year analysis.
      Instead of charging the full investment (CAPEX) upfront or at the end of an asset's lifetime, 
      the total cost is distributed evenly across each year that the asset is operational within the 
      optimization horizon.
      This approach has several technical and economic advantages:

      \begin{itemize}
            \item Each asset pays a fixed annual cost while installed, simplifying the objective.
            \item Only pay for years the asset is in use, avoiding end-of-horizon "distortions".
            \item Removes cost jumps at build/retire dates.
            \item The annuity is linear and easy to implement, unlike full discounted cash-flow tracking.
            \item Slight approximation error from annualizing if the asset is not used for all its lifetime, 
            but negligible in practice.
      \end{itemize}

      This is implemented in code as follows:
      \begin{lstlisting}[language=Python, numbers=none]
      def compute_crf(lifetime, discount_rate):
            # Capital Recovery Factor (CRF)
            if lifetime is None or lifetime <= 0:
                  return 1.0
            i, n = discount_rate, lifetime
            return (i * (1 + i)**n) / ((1 + i)**n - 1)
      
      # Annualized cost (annuity)
      annual_asset_cost = npv * compute_crf(lifetime, discount_rate)
      \end{lstlisting}

      In the optimization: The annualized cost is added for every year the asset is installed, for both 
      generators and storage units:
      \begin{lstlisting}[language=Python, caption=Objective function cost term]
      for y in years:
      total_cost += annual_asset_cost * gen_installed[(g, y)]
      \end{lstlisting}
      
      This modeling choice is implemented in \texttt{dcopf()} and post-processed in the cost analysis script. It ensures that costs are spread proportionally and that the cost function remains robust, regardless of asset timing or lifetime relative to the analysis horizon.

  \item \textbf{Load growth factor}\\
        To mimic a time-varying load through the years, the loads are scaled within the nodal-balance by a defined 
        factor $\gamma_y$. This induces a growing demand in the model without increasing the generation. 
        This induces a diversification of the generation mix through the years.
        \[
        \tilde D_{y} = \gamma_y\,D_{y}.
        \]
        It was decided to model the load growth factor as a global factor for all buses, but it could be extended to a 
        bus-specific factor.
\end{enumerate}

\newpage
\subsection{MILP formulation}
\label{ssec:MILP_implementation}

Only the constraints that changed vs.\ the LP are listed. For readability, the indices set such as $s$ (season) and $t$ 
(hour) were are omitted since per season or per hour concept are implied. Other non-relevant indices may be missing too.

\paragraph{A.~Investment constraints}
\begin{itemize}
  \item Lifetime link \eqref{eq:lifetime_link} (see above).
      \item Generator, storage power and energy limits as mentionned in the
        capacity gating bullet:
        \begin{itemize}
        \item Generator Output Limits:\\
            \begin{equation}
                  0 \leq P^{\text{gen}}_{a,y} 
                  \leq P^{\text{max}}_{a} \cdot z^{\text{on}}_{a,y}
            \end{equation}
    
        \item Storage Power Limits:\\
            \begin{equation}
                  \begin{aligned}
                        0 &\leq P^{\text{ch}}_{a,y} \leq P^{\text{max}}_{a} \cdot z^{\text{on}}_{a,y} \\
                        0 &\leq P^{\text{dis}}_{a,y} \leq P^{\text{max}}_{a} \cdot z^{\text{on}}_{a,y}
                  \end{aligned}
            \end{equation}
    
        \item Energy Capacity Limit:\\
            \begin{equation}
                  - E^{\text{max}}_{a} \cdot z^{\text{on}}_{a,y} \leq E_{a,y} \leq E^{\text{max}}_{a} \cdot z^{\text{on}}_{a,y}
            \end{equation}
      \end{itemize}
\end{itemize}

\paragraph{B.~Operational constraints (modified)}
\begin{itemize}

      
      \item Nodal balance with load scaling\\
            For each bus $b$, season $s$, year $y$, hour $t$:
            \begin{equation}
                  \sum_{a \in \mathcal{A}_b} P^{\text{gen}}_{a,y} +
                  \sum_{a \in \mathcal{A}_b} (P^{\text{dis}}_{a,y} - P^{\text{ch}}_{a,y}) +
                  \sum_{\ell \in \text{in}(b)} F_{\ell,y}
                  = 
                  \gamma_y \cdot D_{b} +
                  \sum_{\ell \in \text{out}(b)} F_{\ell,y}
                  \label{eq:nodal_new}
            \end{equation}
      
      The only change is the growth factor $\gamma_y$ on the demand term.
\end{itemize}

\paragraph{C.~Objective function}
\begin{equation}
      \min \quad
      \underbrace{\sum_{s \in \Sigma} W_s \sum_{y}\Bigl(\sum_{a} c_a \cdot P^{\text{gen}}_{a}\cdot z^{\text{on}}_{a}\Bigr )}_{\substack{\text{operating costs} \\ \text{OPEX}}}
      +
      \underbrace{\sum_{y}\Bigl(
      \sum_{a\in\mathcal{A}}
          A_a \;\cdot z^{\text{on}}_{a}\Bigr)}_{\substack{\text{investment costs} \\ \text{CAPEX}}}
  \end{equation}
  With:
  \begin{itemize}
      \item $W_s$: Number of calendar weeks represented by each season $s$ (e.g., winter = 13)
      \item Annualised CapEx per asset $a$ : $A_a = \mathrm{CRF}_a \cdot C_a$ 
  \end{itemize}  
Hence the MILP simultaneously finds the least-cost \emph{dispatch} and the cheapest \emph{build / replace} schedule over the planning horizon. It eliminates the need for external spreadsheets computing NPV and manual scenarios comparison.

\subsection{Implementation pipeline}
Figures \ref{fig:milp_internal_flow} and~\ref{fig:scripts_block}
show how CSV inputs and time-series flow through the four main Python
modules (\texttt{pre.py}, \texttt{network.py}, \texttt{optimization.py},
\texttt{post.py}) and finally into \texttt{cost.py} for the cost analysis. The code matches the
math 1:1.

%----------------------------------------------------------
\subsection{Investment analysis pipeline}
\label{ssec:milp_impl}
%----------------------------------------------------------

\subsubsection*{A.\ Overview and roles}

\begin{figure}[h!]
      \centering
      \begin{adjustbox}{width=0.8\textwidth}
      \begin{tikzpicture}[
          node distance=1.2cm and 1.6cm,
          every node/.style={font=\small, rounded corners},
          phase/.style={draw, fill=white!40, minimum height=1cm, minimum width=2.6cm, align=center},
          io/.style={draw, fill=blue!20, minimum height=1cm, minimum width=2.6cm, align=center},
          main/.style={draw, fill=green!20, minimum height=1cm, minimum width=2.6cm, align=center},
          opti/.style={draw, fill=orange!20, minimum height=1cm, minimum width=2.6cm, align=center},
          output/.style={draw, fill=purple!20, minimum height=1cm, minimum width=2.6cm, align=center},
          data/.style={draw, fill=gray!10, minimum height=1cm, minimum width=2.6cm, align=center},
          arrow/.style={->, thick}
      ]
  
      % Nodes
      \node[phase] (cli) {CLI};
      \node[main, right=of cli] (main) {main.py\\\scriptsize orchestration};
      \node[io, below=1.2cm of main] (grid) {grid/\\\scriptsize grid config};
      \node[opti, right=of main, text=red] (opti) {scripts/\\\scriptsize Optimization framework};
      \node[data, below=1.2cm of opti] (data) {data/\\\scriptsize generation and loads profiles};
      \node[output, right=of opti] (out) {results/\\\scriptsize metrics, plots};
  
      % Arrows
      \draw[arrow] (cli) -- (main);
      \draw[arrow] (grid) -- (main);
      \draw[arrow] (data.north) -- (opti.south);
      \draw[arrow] (main) -- (opti);
      \draw[arrow] (opti) -- (out);
  
      \end{tikzpicture}
      \end{adjustbox}
      \caption{Compact overview of the forecasting pipeline.}
      \label{fig:forecast_flow_compact}
  \end{figure}

\begin{itemize}
    \item \texttt{main.py} –\- driver script accepting CLI flags, steering the four logical stages:
          preprocessing, network assembly, optimisation, and post-processing/logging.  
    \item \texttt{pre.py} –\- slices three \mbox{168-h} representative weeks, matches
          profiles to assets, and attaches analysis meta-data (years, season weights, load-growth).
    \item \texttt{optimization.py} –\- builds the "DC-OPF MILP" optimization problem with annualised CAPEX,
          solves it via \textsc{CPLEX}, then serialises all variables back into the
          \texttt{IntegratedNetwork}.
    \item \texttt{post.py} –\- turns the raw decision variables into human-readable
          implementation plans, generation-mix graphics and asset timelines.
    \item \texttt{analysis/costs.py} –\- converts dispatch into MWh,
          adds annuitised investment streams, and prints/plots per-asset
          cost breakdowns.
    \item \texttt{results/} –\- output directory for logs, results (.json) and figures.
\end{itemize}

\subsubsection*{B.\ Implementation}
The figure below shows the deeper call graph inside the \texttt{scripts/} directory, highlighting the \emph{three}
execution phases:

\begin{figure}[h!]
    \centering
    \begin{adjustbox}{width=0.8\textwidth}
    \begin{tikzpicture}[
        node distance=1.2cm and 1.2cm,
        every node/.style={font=\small, rounded corners},
        io/.style={draw, fill=green!20, minimum height=1cm, minimum width=2cm, align=center},
        phase/.style={draw, fill=orange!40, minimum height=1cm, minimum width=2.8cm, align=center},
        sub/.style={draw, fill=pink!10, minimum height=1cm, minimum width=3.0cm, align=center},
        helper/.style={draw, fill=orange!5,  minimum height=1cm, minimum width=3.0cm, align=center},
        output/.style={draw, fill=purple!20, minimum height=1cm, minimum width=2.6cm, align=center},
        data/.style={draw, fill=gray!10, minimum height=1cm, minimum width=2.6cm, align=center},
        arrow/.style={->, thick},
        looplabel/.style={font=\scriptsize\itshape}
    ]

    % Nodes -----------------------------------------------
    \node[phase] (prep)      {pre.py\\\scriptsize \#1 Pre-process};
    \node[io, below=of prep, yshift=-4.4cm] (raw) {main.py};
    \node[phase, right=of prep] (net)   {network.py\\\scriptsize \#2 Build objects};
    \node[phase, right=of net]  (milp)  {optimization.py\\\scriptsize \#3 Solve MILP};
    \node[sub, below=of milp] (integ) {IntegratedNetwork()};
    \node[phase, right=of integ] (post)  {post.py\\\scriptsize \#5 Post-proc};
    \node[phase, below=of integ] (analysis) {costs.py \\\scriptsize \#4 Costs analysis};
    \node[coordinate, below=0.3cm of analysis] (mergepoint) {};
    \node[output, below=of analysis] (out) {results/};
    \node[data, above=of prep] (data) {data/};

      % data inputs
      \draw[arrow, dashed] (raw) -- (prep);
      \draw[arrow, dashed] (raw) |- (analysis.west);
      \draw[arrow, dashed] (data) -- (prep);

      % build & optimisation flow
      \draw[arrow] (prep) -- (net);
      \draw[arrow] (net)  -- (milp);
      \draw[arrow] (integ) -- (post);

      \draw[arrow] (milp) -- ++(0,-1.5) node[midway, left] {\scriptsize writes} -- (integ.north);
      \draw[arrow] (net.south) -- ++(0,-1.7) node[midway, left] {\scriptsize builds} -- (integ.west);
      \draw[arrow] (integ.south) -- (analysis.north);


      % new: join arrows from post and analysis
      \draw[arrow] (post.south) |- (mergepoint);
      \draw[arrow] (analysis.south) -- (mergepoint);
      \draw[arrow, dashed] (mergepoint) -- (out.north);

    % Envelope
    \begin{pgfonlayer}{background}
        \node[draw=orange, thick, rounded corners, inner sep=0.4cm, 
              fit=(prep) (net) (milp) (post) (integ) (analysis), 
              label=above:{\textbf{\texttt{scripts/}}}] {};
    \end{pgfonlayer}
    \end{tikzpicture}
    \end{adjustbox}
    \caption{Detailed flow inside \texttt{scripts/}: preprocessing \(\rightarrow\) object construction \(\rightarrow\) MILP solution \(\rightarrow\) reporting.}
    \label{fig:scripts_block}
\end{figure}


\begin{enumerate}
      \item \textbf{Pre-processing}\\
            \texttt{pre.py} converts raw CSV/time-series into \texttt{grid\_data} \(+\) \texttt{seasons\_profiles}.  
            A light sanity-check now ensures the \texttt{representative\_weeks} sum to 52.

      \item \textbf{Object assembly}\\
            \texttt{network.py} wraps every season in a
            \texttt{Network} (data-only) and records them inside a
          global \texttt{IntegratedNetwork}.  Tweaks that proved
          essential:
          \begin{itemize}
              \item bus-ID matching (string vs.\ integer)
              \item automatic snapshot creation with length \(T\)
              \item load-growth factors attached for later scaling
          \end{itemize}

      \item \textbf{Optimization formulation and solve}\\
            \texttt{optimization.py} hosts two core functions:
            \texttt{dcopf()} builds the model, while
            \texttt{investement\_multi()} solves, extracts and stores all variables.

            Key modelling choices:
            \begin{itemize}
                \item \emph{at-most-one-build-per-lifetime} window  
                      \(\rightarrow\) The “installed” variable at year y is the sum of active builds not yet expired:
      \begin{lstlisting}[language=Python]
      for g in generators:
            lifetime = int(first_network.generators.at[g, 'lifetime_years'])
            for y_idx, y in enumerate(years):
                  window_builds = [gen_build[(g, yb)]
                  for yb_idx, yb in enumerate(years)
                  if (y_idx - yb_idx) < lifetime and y_idx >= yb_idx]
                       global_constraints.append(cp.sum(window_builds) <= 1)

            # Installed status = sum of "active" build binaries in
            for y_idx, y in enumerate(years):
                  window_builds = [gen_build[(g, yb)]
                                    for yb_idx, yb in enumerate(years)
                                    if (y_idx - yb_idx) < lifetime and y_idx >= yb_idx]
                  global_constraints.append(
                  gen_installed[(g, y)] == cp.sum(window_builds))

      \end{lstlisting}
                      
                \item Annualised CAPEX via CRF \(\bigl(\)\texttt{compute\_crf}\(\bigr)\)
                      operational costs weighted by season-weeks
                \item No slack variables; nodal balances must close (same as in the LP)
                \item Storage SoC forced to zero at each season edge,
                      \(\Rightarrow\) breaking cross-season energy loops (same as in the LP)
            \end{itemize}

      \item \textbf{Post-processing}\\  
            \texttt{post.py} renders a Gantt-like timeline, seasonal
                  generation mixes, and an implementation plan, while
            \texttt{analysis/costs.py} computes MWh and
                  \$ flows using the same annuity (CRF/discount) logic as
                  in the objective, ensuring consistency.
\end{enumerate}

\subsubsection*{C.\ MILP model structure}
\begin{figure}[h!]
      \centering
      \begin{adjustbox}{width=0.8\textwidth}
\begin{tikzpicture}[
      font=\small,
      node distance=1.4cm and 1.8cm,
      every node/.style={align=center, rounded corners, minimum height=1.0cm, minimum width=2cm, draw},
      box/.style={fill=blue!10},
      arrow/.style={->, thick}
  ]
  % Nodes ------------------------------------------------
  \node[box] (zbuild)  {Create binaries\\ \(z_{\text{build}}\)};
  \node[box, right=of zbuild] (zinst)  {Link to\\ installed status\\ \(z_{\text{inst}}\)};
  \node[box, right=of zinst]  (cap)   {Capacity\\ constraints};
  \node[box, below=of cap]    (dispatch) {Dispatch vars\\ \((p_{\text{gen}},\,p_{\text{line}},\,p_{\text{stor}})\)};
  \node[box, left=of dispatch] (balance) {Nodal\\ balances};
  \node[box, right=of dispatch]    (cost) {Cost\\ function};
  \node[box, below=of cost]   (solve) {Solve with\\ \textsc{Cplex}};
  \node[box, below=of dispatch]  (extract) {Results\\ (JSON)};
  
  % Arrows ------------------------------------------------
  \draw[arrow] (zbuild) -- (zinst);
  \draw[arrow] (zinst) -- (cap);
  \draw[arrow] (cap) -- (dispatch.north);
  \draw[arrow] (balance) -- (dispatch.west);
  \draw[arrow] (zbuild.north) |- ++(0,0.5) -| (cost.north);      % CAPEX
  \draw[arrow] (dispatch.east) -- (cost.west);                   % OPEX
  \draw[arrow] (cost) -- (solve);
  \draw[arrow] (solve) -- (extract);
  
  % Background frame
  \begin{pgfonlayer}{background}
      \node[draw=orange, thick, rounded corners, inner sep=1cm, 
            fit=(zbuild) (zinst) (cap) (dispatch) (balance) (cost) (solve) (extract), 
            label=above:{\textcolor{orange}{\textbf{\texttt{Optimization workflow}} (optimization.py)}}] {};
      \node[draw=gray!50, thick, rounded corners, inner sep=0.2cm, 
            fit=(zbuild) (zinst) (cap) (dispatch) (balance) (extract), 
            label=below:{\textcolor{gray}{\textbf{\texttt{IntegratedNetwork()}}}}]{};
  \end{pgfonlayer}
  \end{tikzpicture}
  \end{adjustbox}
  \caption{Internal control flow inside \texttt{dcopf()}}
  \label{fig:milp_internal_flow}
\end{figure}

Figure~\ref{fig:milp_internal_flow} sketches the internal control
flow inside \texttt{dcopf()}. 

A major bottleneck in large-scale optimization (e.g., MILP) is constraint assembly: explicit Python for-loops 
to create constraints individually causes excessive overhead in both \texttt{CVXPY} (matrix stuffing) and the 
solver’s pre-processing. By leveraging vectorized expressions and a modern solver interface (CPLEX), we significantly 
reduced this overhead.

\begin{itemize}
\item \textbf{Old (CBC/PuLP):} Constraints are added one-by-one in explicit Python for-loops:
      \begin{lstlisting}[language=Python]
            for t in T:
            for i in buses:
            DCOPF += (
            gen_sum - pd_val + flow_in - flow_out == 0
            ), f"Power_Balance_Bus_{i}Time{t}"
      \end{lstlisting}
      This approach is highly inefficient: each constraint is parsed and processed individually in Python, resulting in excessive pre-solve times as problem size grows.
\item \textbf{New (CVXPY/CPLEX):} Constraints are constructed in bulk using array operations:
      \begin{lstlisting}[language=Python]
            for s in seasons:
            for y in years:
            for b in buses:
            flat_constraints.append(
            gen_sum + st_net + flow_in == load_vec + flow_out
            )
      \end{lstlisting}
      Or, for a fully vectorized version, all buses/time steps at once:
      \begin{lstlisting}[language=Python]
            flat_constraints.append(
            cp.sum(p_gen_matrix, axis=0) + cp.sum(p_discharge_matrix, axis=0)
            - cp.sum(p_charge_matrix, axis=0) + cp.sum(flow_in_matrix, axis=0)
            == load_matrix + cp.sum(flow_out_matrix, axis=0)
            )
      \end{lstlisting}
This eliminates per-constraint Python overhead and allows CVXPY to construct and pass large sparse matrices directly to CPLEX.
\end{itemize}

In the old model, assembly times scale poorly ($\mathcal{O}(n_{\text{vars}} \cdot n_{\text{time}})$ constraints 
created in Python; pre-processing could take minutes for moderate $n$).
In the new model, vectorized construction reduces overhead by $50\times$ to $100\times$; model build is near-
instantaneous even for $10^5$+ constraints. Main constraints (capacities, flows, balances) are fully vectorized.

For a multi-year, seasonal, multi-asset problem, model assembly dropped from \textasciitilde 5 minutes (old) 
to $<$10 seconds (new), with no loss of model fidelity. The main bottleneck is now the mathematical difficulty
itself (increased number of constraints and variables).