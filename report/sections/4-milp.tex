\newpage
\section{Linear to Mixed-Integer Programming Transition}
\label{sec:MILP_transition}

In this second semester, the \texttt{vt2} code base upgrades the annual DC-OPF from 
a pure linear program (LP) to a multi-year mixed-integer linear program (MILP).  
This shift keeps the hour-by-hour physical realism of the original formulation while 
moving all lumpy investment choices (build, replacement, decommission) inside the 
optimisation model instead of analysing them in a post-processing spreadsheet.

\subsection{Methodological Rationale}
\label{ssec:MILP_methodology}

\begin{itemize}
    \item \textbf{Endogenised investment.}  
          Candidate generators $g\in\mathcal{G}$ and storage units $s\in\mathcal{S}$ 
          are now accompanied by binary variables
          \(
              \text{build}_{a,y}\in\{0,1\},\;
              \text{installed}_{a,y}\in\{0,1\}
          \)
          for every planning year $y\in\mathcal{Y}$, turning asset-mix selection from 
          an external scenario parameter into an optimisation decision.

    \item \textbf{Lifetime-aware replacement.}  
          A sliding-window logic links first commissioning to subsequent availability, 
          capturing the fact that each asset lives exactly $L_a$ years before a 
          replacement choice arises.

    \item \textbf{Integrated cost metric.}  
          Variable production cost and annualised capital cost (via the capital-recovery 
          factor, CRF) are minimised in a \emph{single} linear objective, removing the 
          need for an external NPV calculator.

    \item \textbf{Multi-year demand growth.}  
          Deterministic load-growth factors $\gamma_y$ scale hourly demand, permitting a 
          least-cost build path that is robust against rising consumption.

    \item \textbf{Strict feasibility.}  
          The former slack variables for load shedding are discarded; unmet demand renders 
          the model infeasible, highlighting missing capacity early.
\end{itemize}

\subsection{Mathematical Implementation}
\label{ssec:MILP_implementation}

The classical DC-OPF constraints (\emph{power balance, line limits, generator bounds, 
storage dynamics}) are retained\,\cite{wood2013power}; only the differences are shown 
below.

\paragraph{(i) Build–Installed Coupling (new).}
For every asset $a$ (generator or storage) with lifetime $L_a$ and for every year 
$y\in\mathcal{Y}$:

\begin{subequations}
\label{eq:build_install}
\begin{align}
\text{installed}_{a,y} &= 
      \sum_{\substack{y'\le y\\ y-y'<L_a}}
      \text{build}_{a,y'} ,
      \qquad\text{(NEW equality)} \label{eq:installed_def}\\[2pt]
\sum_{\substack{y'\le y\\ y-y'<L_a}}
      \text{build}_{a,y'} &\le 1 ,
      \qquad\text{(NEW ``at-most-one'' inequality)} \label{eq:one_build}
\end{align}
\end{subequations}

Eq.~\eqref{eq:installed_def} defines availability as a sliding sum of past builds;  
Eq.~\eqref{eq:one_build} prevents overlapping rebuilds within the asset’s lifetime.

\paragraph{(ii) Capacity Linking (replaces LP bound).}
Dispatch variables inherit a binary toggle:

\begin{align}
0\;\le\;p^{\text{gen}}_{g,y,s,t}
      &\le
      P^{\text{nom}}_{g}\,
      \alpha^{\text{profile}}_{g,s,t}\;
      \text{installed}_{g,y},
      &&\forall g,y,s,t \label{eq:gen_cap}\\
0\;\le\;p^{\text{charge}}_{s,y,s,t}
      &\le
      P^{\text{nom}}_{s}\,
      \text{installed}_{s,y},            &&\forall s,y,s,t \nonumber\\
0\;\le\;p^{\text{dis}}_{s,y,s,t}
      &\le
      P^{\text{nom}}_{s}\,
      \text{installed}_{s,y},            &&\forall s,y,s,t \nonumber\\
0\;\le\;\text{soc}_{s,y,s,t}
      &\le
      E^{\text{nom}}_{s}\,
      \text{installed}_{s,y},            &&\forall s,y,s,t \label{eq:soc_cap}
\end{align}

Equation~\eqref{eq:gen_cap} generalises the old LP upper bound by multiplying nominal 
capacity with both weather availability $\alpha$ (for wind/solar) \emph{and} the binary 
installation flag.  
Bounds \eqref{eq:soc_cap} do the same for storage power and energy.

\paragraph{(iii) Seasonal SoC Boundary (replaces cyclical constraint).}
Instead of enforcing $\text{soc}_{T}= \text{soc}_{0}$, vt2 starts empty and limits the final state:

\begin{equation}
\text{soc}_{s,y,s,0}=0,\qquad
0\le\text{soc}_{s,y,s,T}\le 0.1\,E^{\text{nom}}_{s}\;
\text{installed}_{s,y}.
\label{eq:soc_boundary}
\end{equation}

This relaxation eliminates infeasibility caused by single-week net-export/-import imbalances.

\paragraph{(iv) Nodal Power Balance (replaces LP version).}
Load scaling enters as a multiplier $\gamma_y$:

\begin{equation}
\sum_{g\in\mathcal{G}_b} p^{\text{gen}}_{g,y,s,t}
\;+\;
\sum_{s\in\mathcal{S}_b}\!\bigl(p^{\text{dis}}_{s,y,s,t}-p^{\text{charge}}_{s,y,s,t}\bigr)
\;+\;
\sum_{\ell\in\text{in}(b)}\!\!f_{\ell,y,s,t}
=
\gamma_y\,d_{b,s,t}
+
\sum_{\ell\in\text{out}(b)}\!\!f_{\ell,y,s,t}.
\label{eq:nodal_balance}
\end{equation}

\paragraph{(v) Objective with Annualised CapEx (new component).}
The total cost minimised is

\begin{equation}
\min\;
\underbrace{\sum_{s\in\Sigma}\!W_s\sum_{y,t,g} c_g\,p^{\text{gen}}_{g,y,s,t}}_{\text{operational}}
+
\underbrace{\sum_{y}\!\bigl(
  \sum_{g}\! \text{CRF}_g\,C^{\text{cap}}_g\,\text{installed}_{g,y}
  +
  \sum_{s}\! \text{CRF}_s\,C^{\text{cap}}_s\,\text{installed}_{s,y}\bigr)}_{\text{annualised capital}}
\label{eq:objective}
\end{equation}

where $\text{CRF}_a=\dfrac{i(1+i)^{L_a}}{(1+i)^{L_a}-1}$ converts net-present cost into an equivalent yearly annuity.

\vspace{1ex}
\noindent
Together, constraints \eqref{eq:build_install}–\eqref{eq:nodal_balance} and objective \eqref{eq:objective} turn the original LP into a tractable MILP that simultaneously schedules dispatch \emph{and} chooses the optimal build/replace timetable for each asset.