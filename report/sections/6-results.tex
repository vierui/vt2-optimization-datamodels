\newpage
\section{Results}

\subsection{Investment module using MILP optimization}





\subsection{Forecasting module}
A series of models were developed and compared for operational forecasting of electricity generation. 
Performance was evaluated over a 7-day period starting 2024-01-01. The principal metrics considered are RMSE, 
MAE, and $R^2$, computed both overall and per day. Table~\ref{tab:forecast-metrics} summarizes the test results 
for all model variants.

\begin{table}[h!]
    \centering
    \begin{tabular}{lccc}
        \textbf{Model} & \textbf{RMSE} & \textbf{MAE} & $\mathbf{R^2}$ \\
        \hline
        A. Time only & 0.130 & 0.070 & 0.095 \\
        B. Time + Lag (Bayesian opt.) & 0.022 & 0.011 & 0.973 \\
        C. Bayes Opt. + CV (Time + Lag) & 0.022 & 0.013 & 0.973 \\
        D. Recursive (not used) & 0.119 & 0.053 & -0.281 \\
        E. D + POA Clear-Sky feature & 0.024 & 0.010 & 0.970 \\
    \end{tabular}
    \caption{Forecasting model performance on test set (7 days).}
    \label{tab:forecast-metrics}
\end{table}

%------------------------------------------
\subsubsection*{A. Time Features Only}
The baseline model (A), relying solely on time features, achieved poor predictive performance ($R^2=0.095$), 
with substantial errors for certain days (see daily breakdowns). 

We began with a minimal model using only time-related features (hour and day). This 
provided a simple benchmark and captured regular daily and weekly patterns but did not 
account for weather effects or recent historical trends. It achieved poor performance. 

\begin{figure}[H]
    \centering
    \includegraphics[width=\linewidth]{images/set_time.png}
    \caption{Set 1 - Time features only}
    \label{fig:set1-forecast-profile}
\end{figure}

\begin{table}[H]
    \centering
    \begin{tabular}{lccc}
        Date        & MAE    & RMSE   & R\textsuperscript{2} \\
        \hline
        2024-01-01  & 0.045  & 0.076  & 0.693 \\
        2024-01-02  & 0.098  & 0.169  & -15.138 \\
        2024-01-03  & 0.036  & 0.064  & 0.922 \\
    \end{tabular}
    \caption{Set A - Daily Performance Metrics}
\end{table}

%------------------------------------------
\subsubsection*{B. Time + Lagged Features}
Recognizing the autocorrelated nature of PV output, we next included lagged values of 
electricity generation (e.g., previous hour, previous day, previous week). Incorporating 
these lags enabled the model to better understand night/time patterns (no sub-zero values). However, 
significant errors remained on some days, indicating the model's limited ability to 
capture complex temporal dependencies.

\begin{figure}[H]
    \centering
    \includegraphics[width=\linewidth]{images/set_lags.png}
    \caption{Set B - Time + time/cyclical features (before feature selection)}
\end{figure}

\begin{table}[H]
    \centering
    \begin{tabular}{lccc}
        Date        & MAE    & RMSE   & R\textsuperscript{2} \\
        \hline
        2024-01-01  & 0.043  & 0.078  & 0.677 \\
        2024-01-02  & 0.087  & 0.156  & -12.658 \\
        2024-01-03  & 0.037  & 0.069  & 0.908 \\
    \end{tabular}
    \caption{Set 2 - Daily Performance Metrics}
\end{table}

\textbf{Feature Importance Analysis} Understanding which features 
most significantly impact the model's predictions is crucial for 
interpretability and further model refinement. We conducted a feature 
importance analysis by keeping only the most important features in the 
training set.

We conducted  the feature optimization number by using the Bayesian optimization search.
It learns from previous trials, improving efficiency over brute-force methods sucha grid 
search or back-/front- ward elimination which are more expensive.

\begin{figure}[h!]
    \centering
    \includegraphics[width=0.75\linewidth]{images/feature-importance-cyclic.png}
    \caption{Feature Importance derived from the Bayesian optimization search}
    \label{fig:feature-importance}
\end{figure}

The decrease of the error in this step is highly significant. Not only does it 
show that over 95\% of the model's predictive power was explained by just three 
features: \texttt{electricity\_lag1}, \texttt{electricity\_lag24}, and \texttt{hour\_sin} 
but it also highlights the relevance of removing noise (excessive features) in the prediction
to avoid overfitting. 

\begin{figure}[H]
    \centering
    \includegraphics[width=\linewidth]{images/set_lags2.png}
    \caption{Set B - Time + time/cyclical features (after feature selection)}
    \label{fig:set2-forecast-profile}
\end{figure}

\begin{table}[H]
    \centering
    \begin{tabular}{lccc}
        Date        & MAE    & RMSE   & R\textsuperscript{2} \\
        \hline
        2024-01-01  & 0.013  & 0.025  & 0.967 \\
        2024-01-02  & 0.013  & 0.023  & 0.693 \\
        2024-01-03  & 0.012  & 0.019  & 0.993 \\
    \end{tabular}
    \caption{Set 2 - Daily Performance Metrics}
\end{table}

%------------------------------------------
\subsubsection*{C. Parameter tuning and cross-validation}
To evaluate how well your model is likely to perform on unseen data, we performed 
cross validation (evaluation mechanism). It splits the data into several parts (folds), 
trains on some, and tests on others. It should prevent overfitting. It requires a peticular
attention when it comes to time series data so the splits are not randomly picked but rather 
sequential in time.

Hyperparameter tuning is the search process to find the best hyperparameters for the model.
Its robustness is improved. Again, we used a Bayesian optimization search to find the 
best hyperparameters.

\begin{figure}[H]
    \centering
    \includegraphics[width=\linewidth]{images/set_bayesian1.png}
    \caption{Set C - Features selected by Bayesian optimization}
    \label{fig:set3-forecast-profile}
\end{figure}

\begin{table}[H]
    \centering
    \caption{Set 3 - Daily Performance Metrics}
    \begin{tabular}{lccc}
        Date        & MAE    & RMSE   & R\textsuperscript{2} \\
        \hline
        2024-01-01  & 0.009  & 0.023  & 0.972 \\
        2024-01-02  & 0.010  & 0.022  & 0.732 \\
        2024-01-03  & 0.011  & 0.019  & 0.993 \\
    \end{tabular}
\end{table}

Ironically, these steps could lead to overfitting and poor generalization. 
Simultaneous optimization of hyperparameters with limited evaluation 
calls may not fully explore the searchspace, leading to suboptimal solutions. 
These methods are computationally expensive and using too wide ranges or too 
little cv folds may lead to poor results. It's a trade-off between exploration and 
exploitation. 

We see here, that despite the (expensive) cross-validation and hyperparameter tuning, 
the model does not generalize much better than the one with feature selection only. 

%------------------------------------------
\subsubsection*{D. Recursive Prediction}
In operational settings, true future lag values are unknown; predictions must be generated recursively, 
using model outputs as future lag inputs. This recursive prediction (Model D) more accurately reflects 
real-world constraints, but error accumulation can occur, leading to degraded performance.

\begin{figure}[H]
    \centering
    \includegraphics[width=\linewidth]{images/set_recursive.png}
    \caption{Set 4 - Recursive prediction}
    \label{fig:set4-forecast-profile}
\end{figure}

\begin{table}[H]
    \centering
    \begin{tabular}{lccc}
        Date        & MAE    & RMSE   & R\textsuperscript{2} \\
        \hline
        2024-01-01  & 0.080  & 0.160  & -0.353 \\
        2024-01-02  & 0.027  & 0.052  & -0.499 \\
        2024-01-03  & \textit{NaN}    & \textit{NaN}    & \textit{NaN} \\
    \end{tabular}
    \caption{Set 4 - Daily Performance Metrics}
\end{table}

The model seemed successfully implemented including the preprocessing of time-series 
data (removing nighttime hours), aligning input-output sequences, and setting up the loop 
logic to feed previous predictions into future steps. However, the model did not yield 
stable or reliable results. 

It remained flat. This behavior is most likely due error accumulation and feedback saturation, 
where early incorrect low predictions suppress the entire sequence. Broken temporal continuity 
from removing nighttime data, represents a clear challenge in the modeling too. We tried to 
implement the method but not extensive research was done.  

%------------------------------------------
\subsubsection*{E. Enhanced Features with POA Clear-Sky}
Finally, we extended the feature set to include physics-based drivers—specifically, 
plane-of-array (POA) clear-sky irradiance. By combining these weather-driven variables 
with the selected lags and time features, the model could account for both physical 
potential and recent variability, resulting in the best overall performance.

\begin{figure}[H]
    \centering
    \includegraphics[width=\linewidth]{images/set_poa.png}
    \caption{Set 5 - Enhanced features with POA clear-sky}
    \label{fig:set5-forecast-profile}
\end{figure}

\begin{table}[H]
    \centering
    \begin{tabular}{lccc}
        Date        & MAE    & RMSE   & R\textsuperscript{2} \\
        \hline
        2024-01-01  & 0.008  & 0.026  & 0.970 \\
        2024-01-02  & 0.009  & 0.022  & 0.725 \\
        2024-01-03  & 0.015  & 0.028  & 0.985 \\
    \end{tabular}
    \caption{Set 5 - Daily Performance Metrics}
\end{table}

Error and accuracy metrics are improved. However, the model is still far from being perfect 
and computational expense is high for a sole 0.1\% improvement of the MAE-error for the day ahead 
prediction horizon. 

\begin{figure}[H]
    \centering
    \includegraphics[width=0.75\linewidth]{images/feature-importance-weather.png}
    \caption{Feature Importance including weather-features}
    \label{fig:feature-importance-weather}
\end{figure}

We performed a feature selection with the new weather-features based to understand their importance 
compare to the previous time-only-lags. We notice 7 additional "POA"-features in the top 20 and we see 
that within out parameter array seach we manage to quantify the the sum of their importance which 
improves the prediction performance by 0.2\% of the MAE-error. In average POA-features impact is +0.003.


% \begin{table}[h]
% \centering
% \caption{Validation MAE by model (2022–2023 split).}
% \label{tab:model-comp}
% \begin{tabular}{lcc}
% \hline
% Model & MAE [kW] & Rel. $\Delta$ vs. SARIMA \\
% \hline
% Naïve Seasonal      & 0.142 & +34\% \\
% SARIMA baseline     & 0.106 & — \\
% MLP (2×128)         & 0.565 & +433\% \\
% TCN (64f, 4 blk)    & 0.128 & +21\% \\
% GBDT (XGBoost)      & 0.090 & $-$15\% \\
% SARIMA + GBDT (ours)& 0.085 & $-$20\% \\
% \hline
% \end{tabular}
% \end{table}
% % Table of MAE / RMSE for each model; plots (actual vs forecast).
% % TODO: Write about:
% % - Comparison table of MAE/RMSE for different models
% % - Actual vs forecast plots and visualizations
% % - Model performance across different seasons
% % - Accuracy metrics for different forecast horizons
% % - Statistical significance of improvements

% \subsection{MILP vs Legacy LP}
% % Cost reduction, unit-commitment realism, run-time overhead.
% % TODO: Write about:
% % - Total system cost comparisons
% % - Unit-commitment decision realism improvements
% % - Investment decision quality analysis
% % - Run-time overhead assessment
% % - Solution quality and optimality gaps

% \subsection{Solver Impact (CPLEX vs GLPK)}
% % Solve time, mipgap convergence, memory footprint.
% % TODO: Write about:
% % - Solve time performance comparisons
% % - MIP gap convergence analysis
% % - Memory footprint differences
% % - Scalability improvements
% % - Robustness and reliability comparisons

% \subsection{Sensitivity and Scenario Analysis}
% % Congestion pricing, Lifetime sensitivity, discount-rate sweep.
% % TODO: Write about:
% % - Congestion pricing impact analysis
% % - Asset lifetime sensitivity studies
% % - Discount rate parameter sweeps
% % - Load growth scenario comparisons
% % - Renewable penetration sensitivity

% \subsection{Integrated Workflow Demo}
% % End-to-end run + gantt like result.
% % TODO: Write about:
% % - Complete workflow demonstration
% % - End-to-end execution results
% % - Timeline and scheduling visualization
% % - Integration performance metrics
% % - Workflow automation benefits

% \newpage 