% !TEX program = pdflatex
% !TEX root = main.tex

%----------------------------------------------------------------------------------------
\documentclass[a4paper,10pt]{article}

% Basic packages
\usepackage[utf8]{inputenc}
\usepackage{graphicx}
\usepackage{float}
\usepackage{amsmath}
\usepackage{amssymb}
\usepackage{amsthm}
\usepackage{multicol}
\usepackage{enumitem}
\usepackage{subcaption}
\usepackage{listings}
\usepackage{xcolor}
\usepackage{tikz}
\usetikzlibrary{positioning}
\usepackage[export]{adjustbox}
\usepackage[margin=2.5cm]{geometry}
\graphicspath{{images/}}
\usepackage{multirow}

% Load custom thesis style from config directory
\usepackage{config/thesis}

% Thesis information
\thesistype{Specialisation Project (VT1)}
\thesistitle{Platform for Investment Analysis}
\thesiscontext{Linear Programming Optimization Model for Integrated Energy Systems in Python}
\thesisdate{\today}
\thesissemester{HS2024}
\keywords{linear programming, quantitative modeling, python, strategic planning, optimization, asset valuation, power-flow, platform}

% Author information
\author{Rui Vieira}
\authorinstitute{Institute of Product Development and Production Technologies (IPP)}
\degree{Master of Science in Engineering}
\studyprogram{Business Engineering, MSc in Engineering}
\studyprogramlink{https://www.msengineering.ch/profiles/engineering-and-it/business-engineering}

% Supervisor information
\supervisorA{Dr. Andrea Giovanni Beccuti}
\supervisorAmail{giovanni.beccuti@zhaw.ch}
\supervisorAweb{https://www.zhaw.ch/en/about-us/person/becc}
\supervisorAinstitute{IEFE Model Based Process Optimisation}
\supervisorAinfo{%
    \supnameA\\
    \supinstituteA\\
    Email: \href{mailto:\supmailA}{\supmailA}\\
}

%----------------------------------------------------------------------------------------
\begin{document}

% Front
\begin{titlepage}

\begin{center}

\textup{\small {\bf Specialisation Project (VT) } \\  HS2024}\\[3.0in]

% Title
\fontsize{20}{24}\selectfont
\textbf {Platform for Investment Analysis}\\
\normalsize Optimization Framework for Energy Asset Management using Linear Programming in Python\\[3.0in]


       

% Submitted by
\normalsize Submitted by \\[0.2in]
\textbf{Rui Vieira}\\
Business Engineering Profile\\
IPP Institute of Product Development and Production Technologies\\

\vspace{.2in}




\vspace{.3in}

% Bottom of the page
\includegraphics[width=0.4 \textwidth]{images/mse.png}\\[0.1in]

January 2025

\end{center}

\end{titlepage}
% !TEX root = ../main.tex

%----------------------------------------------------------------------------------------
% IMPRINT
%----------------------------------------------------------------------------------------

\thispagestyle{empty}
\vspace*{\fill}

\noindent
{\bfseries  \Large Imprint}
\vspace{0.75cm}

\begin{footnotesize}

% Project information
\begin{flushleft} 
\begin{tabular}{ @{}lp{0.64\textwidth}@{} } 
    \emph{Project:}  & \ttype\\ 
    \emph{Title}:    & \ttitle\\
    \emph{Author}:   & \authorname\\
    \emph{Date}:     & \tdate\\
    \emph{Keywords}: & \keywordnames\\
\end{tabular}
\end{flushleft}

\vspace{0.75cm}

% University information
\noindent
\begin{minipage}[t]{0.95\textwidth}
\begin{flushleft} 
\emph{Study program:}\\
\href{\studyproglink}{\studyprog}\\
\href{\univlink}{\univname}
\end{flushleft}
\end{minipage}

\vspace{1.1cm}

\noindent
\begin{minipage}[t]{0.50\textwidth}
\begin{flushleft} 
\emph{Supervisor:}\\
\supinfoA
\end{flushleft}
\end{minipage}
\begin{minipage}[t]{0.45\textwidth}

    \begin{flushleft} 
\ifdefempty{\supnameB}
{}
{
    \emph{Supervisor 2:}\\
    \supinfoB
}
\end{flushleft}
\end{minipage}


\end{footnotesize}

\newpage
% !TEX root = ../main.tex

%----------------------------------------------------------------------------------------
% ABSTRACT PAGE
%----------------------------------------------------------------------------------------

\begin{abstract}
This work presents an integrated investment framework for energy systems, focusing on 
optimal technology selection and placement of electrical generation, conversion, and storage assets. The core 
engine combines DC Optimal Power Flow (DC-OPF) simulations with linear programming (using the PuLP solver) to 
evaluate both technical feasibility and economic viability across a multi-scenario analysis. A 9-bus test network 
provides the backdrop for a reduced ten distinct cases, each featuring a unique mix of conventional (nuclear, gas) and 
renewable (solar, wind) power plants, supplemented by battery storage of varying capacities.

To balance computational efficiency with seasonal realism, the annual horizon is divided into three representative 
weeks (summer, winter, and spring/autumn), whose costs and operations are subsequently scaled to form a full-year 
analysis. This approach reveals significant seasonal differences in storage utilization even enabling clean assets to
compensate for the cost of conventional generation.

\begin{figure}[H]
    \centering
    \includegraphics[width=0.6\textwidth]{images/soc_5.png}
    \caption{Summer/Winter battery SoC profile comparison -- Scenario 5}
    \label{fig:soc5}
\end{figure}

In scenarios featuring abundant solar generation, the SoC frequently reaches its upper limits, highlighting the 
potential for upsizing or more flexible operational strategies—such as battery leasing or modular additions—to 
capture peak renewable output.

An economic sensitivity analysis underscores the strong influence of high-cost resources during extreme load 
conditions, causing a disproportionate rise in total costs when reliance on expensive generation escalates. 
Meanwhile, scenarios with nuclear-dominated baseload exhibit lower operational cost volatility but may still 
benefit from targeted storage deployment to manage residual demand swings. AI-assisted reporting consolidates 
these findings by identifying cost drivers, optimal technology mixes, and operational bottlenecks across all 
scenarios. Notably, Scenario7’s balanced blend of nuclear, solar, and wind with moderate battery support emerges 
as the most cost-effective configuration, while Scenario4, featuring gas-fired generation and multiple storage 
units, proves the least favorable in terms of net present value (NPV).

Overall, the proposed framework bridges technical dispatch simulation and investment analysis, guiding stakeholders 
in designing resilient, economically viable energy systems. Future enhancements include broader maintenance modeling, 
real-time price integration for advanced arbitrage strategies, and further prompt-engineering improvements to refine 
AI-driven reporting and decision support.

\textbf{Keywords:} \keywordnames

%DC-OPF, Investment Analysis, Linear Programming, Renewable Integration, 
%Power Systems Planning, Asset Placement Optimization
\end{abstract}
\newpage

% Table of contents
\setcounter{tocdepth}{2}
\tableofcontents
\newpage

% Main content
\newpage
\section{Introduction}


% Context and motivation
% Problem statement
% Project objectives
% Brief overview of existing solutions/literature review
% Brief introduction to energy investment optimization and the need for automated analysis tools.

\subsection{Project Context}

\subsection{Objectives}
\begin{itemize}
    \item Develop a Python-based platform for energy investment analysis
    \item Implement linear programming optimization for asset management
    \item Investment analysis 
\end{itemize}
\newpage 

\subsection{DC Optimal Power Flow}
The DC Optimal Power Flow (DCOPF) is a simplified version of the AC power flow problem, making several key assumptions to linearize the problem while maintaining sufficient accuracy for investment analysis.

\subsubsection{Key Assumptions}
\begin{itemize}
    \item Voltage magnitudes are approximately 1.0 per unit
    \item Voltage angle differences are small ($\sin \theta \approx \theta$)
    \item Line resistance is negligible compared to reactance (R « X)
    \item Reactive power flow is ignored
\end{itemize}

\subsubsection{Mathematical Formulation}
The DCOPF problem is formulated as a linear programming optimization:

\begin{equation}
    \min_{P_g, \theta} \sum_{i \in G} c_i P_{g,i}
\end{equation}

Subject to the following constraints:

\textbf{Power Balance Constraints:}
\begin{equation}
    \sum_{i \in G} P_{g,i} - \sum_{i \in D} P_{d,i} = \sum_{(i,j) \in L} P_{ij} \quad \forall i \in N
\end{equation}

\textbf{Line Flow Constraints:}
\begin{equation}
    P_{ij} = \frac{\theta_i - \theta_j}{x_{ij}} \quad \forall (i,j) \in L
\end{equation}

\textbf{Generation Limits:}
\begin{equation}
    P_{g,i}^{min} \leq P_{g,i} \leq P_{g,i}^{max} \quad \forall i \in G
\end{equation}

\textbf{Line Capacity Limits:}
\begin{equation}
    -P_{ij}^{max} \leq P_{ij} \leq P_{ij}^{max} \quad \forall (i,j) \in L
\end{equation}

Where:
\begin{itemize}
    \item $P_{g,i}$: Power generation at bus $i$
    \item $P_{d,i}$: Power demand at bus $i$
    \item $\theta_i$: Voltage angle at bus $i$
    \item $x_{ij}$: Line reactance between buses $i$ and $j$
    \item $P_{ij}$: Power flow on line between buses $i$ and $j$
    \item $c_i$: Generation cost coefficient at bus $i$
\end{itemize}

\subsubsection{Additional Constraints for Investment Analysis}
For our investment optimization, we add the following constraints:

\textbf{Investment Budget Constraint:}
\begin{equation}
    \sum_{i \in G} I_i x_i \leq B_{max}
\end{equation}

\textbf{Capacity Factor Constraints:}
\begin{equation}
    \frac{1}{T} \sum_{t=1}^T \frac{P_{g,i,t}}{P_{g,i}^{max}} \geq CF_{min,i} \quad \forall i \in G
\end{equation}

Where:
\begin{itemize}
    \item $I_i$: Investment cost for generator $i$
    \item $x_i$: Binary decision variable for investment in generator $i$
    \item $B_{max}$: Maximum investment budget
    \item $CF_{min,i}$: Minimum required capacity factor for generator $i$
    \item $T$: Number of time periods
\end{itemize}

\subsubsection{Implementation Considerations}
\begin{itemize}
    \item The problem is implemented using Python with Pyomo optimization framework
    \item Seasonal variations are handled through multiple time periods
    \item Generator availability is modeled using capacity factors
    \item Investment decisions are binary variables
\end{itemize} 
\newpage
\section{Methodology and Implementation}
% The system architecture is designed to keep data handling, scenario management, and optimization separate yet 
% interconnected. This structure simplifies extending the model—whether adding more generation types, changing load 
% profiles, or altering cost assumptions. The next section (\S 3.2) will focus on the core components of the 
% mathematical formulation and the integration of financial metrics, such as Net Present Value (NPV) and annuity 
% calculations.

% •	Data Architecture: Clearly separated CSV files for network topology, time-series data, and scenario definitions.
% •	Software Components: A Python-based DCOPF solver with scenario management via main.py, plus optional
% AI-powered summaries.
% •	Flowcharts & Snippets: Recommended visuals and short code excerpts to illustrate data flow (inputs → solver 
% → outputs) and to demonstrate the script’s orchestration logic.
% ensures flexibility in adding or modifying new assets, adapting to different load profiles, and 
% rapidly generating comparative analyses across multiple scenarios.


%---------------------------------------------------------------------------------------
\subsection{System Architecture}
\label{sec:system_architecture}

\subsubsection{Data structure and Input Flow}
\label{sec:input_flow}
The architecture relies on a set of CSV files that define (1) the physical network, (2) the time-series input data, 
and (3) the scenario configurations for analysis. All of these files reside in the \texttt{data/working} directory 
and are ultimately passed to the \texttt{main.py} script, which processes each scenario in turn.

\begin{enumerate}
  \item \textbf{Physical Network Files}: \texttt{branch.csv} and \texttt{bus.csv} \\
  Define the perimeter and physical layout of the power grid in a format inspired by MATPOWER~\cite{matpower2024}. 
  For example, \texttt{branch.csv} provides line impedances and flow limits, while \texttt{bus.csv} specifies bus 
  voltage information, types, and identifiers.
  
  \item \textbf{Time-Series Input Data}: \texttt{master\_gen.csv} and \texttt{master\_load.csv} \\
  One holds the asset-level generation profiles (nuclear, wind, solar, etc.), including capacity 
  limits, per-unit costs, and operational constraints. 
  
  The other one, contains bus-level load profiles, mapped to specific seasonal segments (winter, summer, 
  spring/autumn). These files reflect the availability or demand data relevant to a particular study-case.
  
  \item \textbf{Scenario Configurations}: \texttt{scenarios\_parameters.csv} \\
  Centralizes the definitions of each scenario: which generators or storage units are placed at which buses, along 
  with any load scaling factors (e.g., $\pm20\%$ demand). This dataset dictates how the solver will allocate and 
  dispatch resources in different configurations.
\end{enumerate}

The main driver script, \texttt{multi\_scenario.py}, loads and combines these CSVs to set up each scenario’s 
DCOPF problem. By separating network topologies, time-series inputs, and scenario definitions, the framework 
allows new assets, bus layouts, or experimental conditions to be tested without significant changes to the 
core codebase.

\begin{figure}[H]
    \centering
    \includegraphics[width=0.75\textwidth]{images/input-flow.jpeg}
    \caption{Overview of the Data Flow into the Optimization Process} \label{fig:input-flow}
\end{figure}

%---
% flowchart LR
%     subgraph Inputs
%       A["Physical Network <br> (branch.csv, bus.csv)"]
%       R["Raw Data"]
%       I{"Ingestion Data"}
%       T["Time-Series <br> (master_gen.csv, master_load.csv)"]
%       C["Scenario Configurations <br> (scenarios_parameters.csv)"]
%     end

%     %% Chain for time-series ingestion
%     R --> I
%     I --> T

%     %% Connect inputs to main processing (main.py)
%     A --> multiScenario[main.py]
%     T --> multiScenario
%     C --> multiScenario

%     %% LP Solver and hidden endpoint
%     multiScenario --> D[Execution]
%     D --> E[Output]

    % %% Define a custom class with a light fill color
    % classDef lightBox fill:#f3f3f3, stroke:#333, stroke-width:2px;

    % %% Apply the custom class to nodes D and E
    % class D,E, lightBox;
%---

\subsubsection{Execution}
\label{sec:execution}
The \texttt{main.py} script serves as the central orchestrator of the optimization workflow.
Its key responsibilities include:

\begin{itemize}
    \item \textbf{Scenario Management}\\
    Loads scenario definitions from \texttt{scenarios\_parameters.csv}, which specify generator placements, storage 
    configurations, and load scaling factors for each test case.
    
    \item \textbf{Data Integration}\\
    Combines network topology data (\texttt{bus.csv}, \texttt{branch.csv}) with time-series inputs 
    (\texttt{master\_gen.csv}, \texttt{master\_load.csv}) to construct complete optimization problems.
    
    \item \textbf{Sensitivity Analysis}\\
    For each base scenario, optionally generates variants with modified load factors (e.g., $\pm20\%$) to test 
    system robustness under different demand conditions.
    
    \item \textbf{Results Collection}\\
    Aggregates solver outputs, calculates key metrics (e.g., capacity factors, annual costs), and stores results 
    in standardized formats for further analysis.
    
    \item \textbf{Investment Analysis}\\
    Interfaces with \texttt{create\_master\_invest.py} to compute financial metrics like NPV and annualized costs 
    across different time horizons.
\end{itemize}

They can be vizualized in the following flowchart:

\begin{figure}[H]
    \centering
    \includegraphics[width=0.75\textwidth]{images/execution-flow.jpeg}
    \caption{Execution and interaction of the main.py script with the other modules}
    \label{fig:execution-flow}
\end{figure}

%---
% flowchart LR

%     subgraph execution [Execution]
%         E
%         G
%         I
%         F
      
%     end

%     D[main.py]
%     %% LP Solver and Annual Costs Calculation
%     D --> E{"LP Solver"}
%     E --> F["Annual Costs <br> (scenario_results.csv)"]

%     %% Investment Analysis
%     D --> G{"Invest. Metrics <br> Calculations"}
%     F --> G
%     G --> K
%     %% Scenario Critic and Reporting
%     D --> I{"AI <br> Analysis"}
%     K --> I
%     I --> J

%     %% Output Files/Structure
%     subgraph Outputs [Outputs]
%         J["Markdown Summary and <br> AI-Generated Reports"]
%         K["Final Scenario Results <br>(scenario_results_with_investment.csv)"]
%         END:::hidden  
%     end
%---

The \texttt{main.py} script also coordinates with auxiliary modules for visualization (\texttt{summary\_plots.py}), 
and documentation updates (\texttt{update\_readme.py}), ensuring up-to-date visualizations and online-documentation.

\subsubsection{Output Flow}
\label{sec:output_flow}
After \texttt{main.py} coordinates the execution of all scenarios through \texttt{dcopf.py}, 
\texttt{create\_master\_invest.py}, and \texttt{summary\_plots.py}, it generates three key outputs:

\begin{itemize}
    \item \textbf{Global Summary Report}\\
    A comprehensive \texttt{summary.md} file that ranks scenarios by annuity value, 
    provides AI-generated insights on overall trends, and includes comparative visualizations across scenarios.

    \item \textbf{Individual Scenario Reports}\\
    For each scenario (e.g., \texttt{scenario\_1\_analysis.md}), it generates a 
    detailed report with: dispatch plots and generation mix charts, financial metrics breakdown, and AI-generated 
    commentary on specific operational patterns.

    \item \textbf{Consolidated Results}\\
    A \texttt{scenario\_results.csv} within \texttt{data/results} containing raw operational
    data (generator dispatch, line flows), investment metrics (NPV, annual costs, annuities), and sensitivity 
    analysis results (if enabled).
\end{itemize}

\noindent
It illustrates the principal output files after interaction with the main.py script and the other modules.

\begin{figure}[H]
    \centering
    \includegraphics[width=0.75\textwidth]{images/output-flow.jpeg}
    \caption{Output results and reports}
    \label{fig:output-flow}
\end{figure}

%---
% flowchart LR
%     %% LP Solver and Annual Costs Calculation
%     A[Inputs]-->main
%     main[main.py] --> Execution
%     Execution --> Outputs
%     %% Output Files/Structure
%     subgraph Outputs [Outputs]
%       K["Scenario Results <br>(scenario_results_with_investment.csv)"]
%       L["Global Comparison Report<br>(global_comparison_report.md)"]
%       M["Individual Scenario Reports<br>(scenario_x folders with markdown & figures)"]
%     end
%     style Execution fill:#f3f3f3, stroke:#333;
%     style A         fill:#f3f3f3, stroke:#333;
%---

This multi-step workflow ensures all relevant data—raw operational outputs, investment metrics, and optional AI 
insights—remain easily accessible for post-processing or stakeholder review. As a result, users can quickly compare 
scenarios under different configurations, load sensitivities, or asset placements without altering the core solver routines.

% \subsubsection{Illustrative Code Excerpt}
% % Directive: Provide a code snippet from main.py focusing on scenario orchestration.

% The following code snippet from \texttt{main.py} illustrates how scenario parameters are loaded and solved:

% \begin{lstlisting}[
%     language=Python,
%     label={lst:main},
%     caption={Excerpt from main.py illustrating how scenario parameters are loaded and solved},
%     float=htbp
% ]
% # Load scenario definitions
% scenarios_df = pd.read_csv(scenarios_params_file)

% for _, row in scenarios_df.iterrows():
%     scenario_name = row["scenario_name"]
%     gen_positions = parse_positions(row["gen_positions"], data_context['type_to_id'])
%     storage_positions = parse_positions(row["storage_units"], data_context['type_to_id'])
%     base_load_factor = float(row["load_factor"])

%     # Optionally run sensitivity variants: nominal, high (+20%), low (-20%)
%     variants_to_run = [("nominal", base_load_factor)]
%     if run_sensitivity:
%         variants_to_run += [("high", base_load_factor * 1.2),
%                             ("low", base_load_factor * 0.8)]

%     for variant_name, load_factor in variants_to_run:
%         result = run_scenario_variant(
%             scenario_name=scenario_name,
%             gen_positions=gen_positions,
%             storage_positions=storage_positions,
%             load_factor=load_factor,
%             variant=variant_name,
%             data_context=data_context
%         )
%         # Aggregate seasonal results, compute investment metrics, store final
%         ...
        
% \end{lstlisting}

%---------------------------------------------------------------------------------------
\subsection{Core Components}
We detail the four principal building blocks of the project. Their interaction is illustrated in the system 
architecture presented in Section~\ref{sec:system_architecture} by losanges.

\begin{multicols}{2}
\begin{itemize}
  \item LP Optimization Solver (DCOPF)
  \item Data Ingestion \& Preprocessing
\end{itemize}
\columnbreak
\begin{itemize}
  \item Investment Metrics calculation
  \item AI-based reporting
\end{itemize}
\end{multicols}
Each element addresses a distinct requirement, from solving the DC power flow problem to 
generating final scenario reports with optional AI-driven commentary.


\subsubsection{Data Ingestion \& Preprocessing}
\label{sec:data_preprocessing}

\textbf{Data Handling} \\
Data handling follows a three-tiered structure:

\begin{enumerate}
  \item \textbf{\texttt{data/raw}}: Unaltered sources such as annual wind/solar profiles from public 
  databases or raw load curves provided by our supervising professor.

  \item \textbf{\texttt{data/processed}}: Intermediate files that have undergone partial cleaning 
  (e.g., timestamps alignment, filtering outliers). As part of this step, we also perform a 
  \emph{Seasonal and Trend decomposition using Loess (STL)} to identify anomalies in the load profile, following 
  methods described in \cite{predictive_modeling_notes}. The seasonal median week was picked for each season.

  \item \textbf{\texttt{data/working}}: used .csv files directly by the solver and scenario 
  scripts (\texttt{master\_gen.csv}, \texttt{master\_load.csv}). These files are concise, 
  containing only the time-series data and parameters needed for each scenario run.

\end{enumerate}

The STL decomposition applied during the preprocessing step to validate our \emph{typical-week} selection for each 
season (Figure~\ref{fig:stl_load_decomposition}). The data showed a broad U-shaped trend (lower demand in warmer months) and 
strong daily/weekly seasonality. Residuals exhibited higher variance at the start and end of the year, suggesting 
possible holiday or extreme-weather anomalies. Excluding those outlier weeks helped ensure our final “median” week 
captures typical load patterns.

\begin{figure}[H]
    \centering
    \includegraphics[width=0.6\textwidth]{images/stl-load.png}
    \caption{STL decomposition of the annual load data}
    \label{fig:stl_load_decomposition}
\end{figure}

\textbf{Scripts for Data Preprocessing} \\
Two Python scripts form the backbone of our data preprocessing:

\begin{itemize}
  \item \texttt{create\_master\_gen.py}:
  Consolidates multiple raw generation sources (wind, solar, 
  nuclear, etc.) into a unified time series, selecting a \emph{median week} per season to reduce 
  hours from 8,760 to just $7 \times 24$.

  \item \texttt{create\_master\_load.py}: 
  Builds coherent load profiles for each season, optionally shifting one bus’s demand by a week if needed.
  Buses~5 and~6 serve as the primary load centers in our example network.
\end{itemize}

These scripts output \texttt{master\_gen.csv} and \texttt{master\_load.csv} in \texttt{data/working}.
By scaling each typical week by 13 (winter/summer) or 26 (spring/autumn), the method preserves core 
weekly cycles—consistent with the STL findings—while excluding outlier periods. This provides a 
robust yet computationally efficient basis for annual cost estimation.

\textbf{Network Modeling} \\
A simplified network is defined via \texttt{branch.csv} and \texttt{bus.csv} in \texttt{data/working}. 
Key characteristics of this grid (e.g., line impedances, bus voltage levels) are presented in 
Section~\ref{sec:results}. While our example focuses on a small test system, larger networks 
(e.g., IEEE 30-bus or 118-bus) can be incorporated using the same file structure.


\subsubsection{LP Optimization Solver}
\label{sec:lp_solver}
In this module, the \texttt{dcopf.py} script translates the DC power flow problem and associated operational 
constraints into a linear program (LP). Although the mathematical formulation (objective function and constraints)
has been fully described in Section~\ref{sec:theoretical_background}, we outline here how these elements are 
\emph{implemented} at the code level, focusing on the handling of hourly dispatch and storage dynamics.

\textbf{Time-Step Stacking}\\
For each hour $t$ of the representative dataset (e.g., 24 hours $\times$ 7 days per season), the solver creates:

\begin{itemize}
  \item \textbf{Generation Variables}\\
  It ensures the solver respects asset-level capacity limits in every hour.
  In \texttt{dcopf.py}, each dispatchable asset $g$ is assigned a variable
  \lstinline|GEN[g, t]| that ranges between \lstinline|pmin| and \lstinline|pmax|.
  For each time step $t$, we create:
  \begin{lstlisting}[language=python]
  for g in G:
    for t in T:
        pmin = ...
        pmax = ...
        GEN[g, t] = pulp.LpVariable(
            f"GEN_{g}_{t}_var",
            lowBound=pmin, upBound=pmax)
  \end{lstlisting}

  \item \textbf{Voltage Angles} (Phase Angles) at Each Bus \\
  The DC power flow constraints then enforce line flows based on the angle difference between connected buses.
  A dictionary \lstinline|THETA[i, t]| stores the phase angle at bus $i$ and time $t$:
  \begin{lstlisting}[language=python]
  for i in N:     # set of buses
    for t in T:   # set of time steps
        THETA[i, t] = pulp.LpVariable(
            f"THETA_{i}_{t}_var", lowBound=None, upBound=None)
    \end{lstlisting}
    
  \item \textbf{Line Flow Variables} \\
  For each branch $(i, j)$ in \texttt{branch.csv}, we create \lstinline|FLOW[i, j, t]|,
  constrained by the line's thermal limit (\lstinline|rateA|). The relevant code snippet
  looks like:
  \begin{lstlisting}[language=python]
  for idx, row in branch.iterrows():
    i = int(row['fbus'])
    j = int(row['tbus'])
    limit = row['ratea']
    for t in T:
        FLOW[i, j, t] = pulp.LpVariable(
            f"FLOW_{i}_{j}_{t}_var",
            lowBound=-limit, upBound=limit)
    \end{lstlisting}
    Here, negative values indicate flow in the opposite direction, respecting the
    bidirectional capacity of AC transmission lines (under the DC approximation).
\end{itemize}

These variables are repeated across all time steps, thereby stacking the corresponding constraints to capture the
evolution of dispatch decisions over the week.

\textbf{Storage Modeling}\\
A key feature in \texttt{dcopf.py} is the \emph{storage} handling. The code introduces:

\begin{itemize}
    \item \textbf{Charge/Discharge Variables} for each storage asset, bounded by $\pm P_{\max}$.
    Each storage asset $s$ has two distinct variables: 
    \lstinline|P_charge[s, t]| (charge rate) and \lstinline|P_discharge[s, t]| (discharge rate),
    both bounded by $\pm P_{\max}$. Below is a simplified Python excerpt:

    \begin{lstlisting}[language=python]
for s in S:  # S is the set of storage IDs
    # Retrieve max power from CSV or data context
    P_max = storage_info[s]['pmax']
    for t in T:
        P_charge[s, t] = pulp.LpVariable(
            f"P_charge_{s}_{t}",
            lowBound=0,   # cannot be negative
            upBound=P_max)
        P_discharge[s, t] = pulp.LpVariable(
            f"P_discharge_{s}_{t}",
            lowBound=0,   # cannot be negative
            upBound=P_max)
    \end{lstlisting}
    Note that charge and discharge are individually constrained to be non-negative, ensuring 
    the net power from storage ($P_{\text{discharge}} - P_{\text{charge}}$) stays within $\pm P_{\max}$.

    \item \textbf{State of Charge (SoC) Variables:}\\
    The SoC at each time step $t+1$ depends on the SoC at time $t$, as well as any charge or discharge volumes. 
    To ensure continuity, we define an \emph{extended time index} such that $t_0 = t_{24 \times 7 + 1}$, 
    creating a cyclical boundary condition where the end-of-horizon SoC equals the initial one.

  \end{itemize}

\textbf{Objective Function \& Solver Execution.}\\
The objective function \emph{sums the dispatch costs} of all generators over each time step. After constructing 
these LP constraints, the script invokes PuLP’s  interface to the CBC solver to:
\begin{enumerate}
    \item \textbf{Assemble} the problem (variables, constraints, and objective) in standard form:
    \begin{equation*}
    \begin{aligned}
    \min_{\mathbf{x}} \quad & \mathbf{c}^T\mathbf{x} \\
    \text{s.t.} \quad & \mathbf{A}\mathbf{x} = \mathbf{b} \\
    & \mathbf{l} \leq \mathbf{x} \leq \mathbf{u}
    \end{aligned}
    \end{equation*}
    where $\mathbf{x}$ contains all decision variables (generation, flows, storage), $\mathbf{c}$ represents costs, 
    $\mathbf{A}$ encodes network constraints, and $\mathbf{l},\mathbf{u}$ are variable bounds.
    \item \textbf{Solve} for an \textit{hourly dispatch schedule} that minimizes total cost while respecting line 
    flows and operational limits.
    \item \textbf{Extract} the final solutions (e.g., \texttt{gen}, \texttt{flow}, \texttt{storage} states), storing
    them in data frames.
\end{enumerate}

\textbf{Scaling to Annual Results.}\\
Because each run typically focuses on a single “typical week” per season, the corresponding cost is subsequently 
scaled (see Section~\ref{sec:data_preprocessing}) to approximate annual figures. While this reduces the 
computational load by omitting all annual 8,760 hours, computing 3 seperaed weeks of 504 hours (24 hours $\times$ 
7 days) instead. It preserves the essential behavior of generation and storage dispatch.


\subsubsection{Financial Metrics Module}
Investment decisions in infrastructure oftens rely on 3 key aspects : numbers (metrics), interest rates $i$ , 
and time horizon $T$. The calculations are performed in \texttt{create\_master\_invest.py}.

Interest rates for electrical infrastructure typically range from 5\% to 10\% \cite{irena2023costs} Its value has 
significant impact on the calculations and, consequently, the investment decision. Time horizon often depends
on the project's expected lifetime, but can also be subject to regulatory constraints. For metrics, we can rely on 
the Net Present Value (NPV) and the annuity (a).

\begin{figure}[h!]
    \centering
    \includegraphics[width=0.3\textwidth]{images/investment-decision.png}
    \caption{The three judgement dimensions influencing investment decisions by T. Herrmann}
    \label{fig:investment_decision}
\end{figure}

\textbf{Net Present Value (NPV)} \\
The NPV sums the present value of all cash flows over the asset lifetime. If positive, it is profitable.

\begin{equation}
\label{eq:npv_equation}
\text{NPV} = Z_0 + \sum_{t=0}^{T} \frac{Z_t}{(1 + i)^t}
\end{equation}

\noindent
where $Z_0$ represents the initial investment cost at $t=0$ and $Z_t$ are the cash flows (costs or revenues) 
in period $t$.

By repeating the NPV calculation for each scenario, we can compare the financial viability of different 
investment options. The scenario with the highest positive NPV is the most profitable one.

\textbf{Annuity} \\
An other dynamyic decison making metric is the \emph{annuity} which permits to assess decison making on assets
with different lifetimes. It denotes an equivalent constant monetary input over the period under consideration.

Is calculated calculated from the NPV by multiplying it by the annuity factor (Capital Recovery Factor, CRF) :

\begin{equation}
\label{eq:annuity_equation}
a = NPV \cdot \frac{i (1 + i)^T}{(1 + i)^T - 1} = NPV \cdot CRF
\end{equation}

\textbf{Sensitivity Analysis} \\
While sensitivity analyses typically focus on varying discount rates, we instead analyze load variations 
($\pm20\%$) since interest rates were already discussed in Section~\ref{sec:data_preprocessing}. This helps 
assess both investment robustness, under demand uncertainty and identify potential risk of network constraints 
requiring additional capacity.


\subsubsection{Automatic AI-Based Reporting}
\label{sec:ai_reporting}

% A final, optional feature involves generating textual summaries via \textbf{OpenAI API}~\cite{openai_api_docs}
% calls in \texttt{scenario\_critic.py}. After each scenario's DCOPF run and financial calculations:
% \begin{itemize}
%   \item \textbf{Scenario Data Collection}: The script compiles key metrics (annual cost, NPV, etc.) and 
%   sends them to an LLM endpoint.
%   \item \textbf{Markdown Report Augmentation}: The LLM returns a short descriptive or analytic paragraph 
%   (e.g., identifying dominant cost drivers or storage benefits), which is appended to \texttt{.md} reports.
% \end{itemize}

% Moreover, the script automatically generates individual Markdown reports for each scenario including related 
% metrics and creates a global summary report that synthesizes insights across all scenarios, enabling quick 
% comparison of key findings and trade-offs.

% \paragraph{AI-Based Reporting and Automatic Reports: }
This module, \texttt{scenario\_critic.py}, integrates scenario data (annual costs, generator dispatch, net
present values, etc.) with an AI-based module to automatically produce Markdown reports for each scenario. 
Concretely, it performs:

\begin{enumerate}
    \item \textbf{API Integration} \\
    A class (\lstinline|ScenarioCritic|) initializes an OpenAI client using an API key~\cite{openai_api_docs}. We 
    define a "context prompt" describing the energy system mix (nuclear, gas, wind, etc.), relevant 
    cost metrics, and storage assets. 

    \item \textbf{Generating a Critical Analysis} \\
    The script collects scenario outputs (e.g. annual\_cost, generation by asset, etc.) into 
    a short "user" prompt. It requests an AI-generated critique. We chose to focus on the economic efficiency, 
    strengths/weaknesses, and possible improvements. This feedback is concise (up to 200 words) and helps users 
    quickly assess each scenario.

    \item \textbf{Automatic Markdown Reports.} \\
    Using both values from the optimization/investment metrics module and the AI critique, 
    the script compiles a scenario-specific \texttt{.md} file (e.g., 
    \lstinline|scenario_1_analysis.md|). 
    An example of the reports can be found in the Appendix.
    The final report includes:
    \begin{itemize}
        \item \emph{Key Financial Metrics}: Initial investment, annual cost, various NPVs.
        \item \emph{Generation Statistics}: Generation costs by asset, capacity factors, seasonal trends.
        \item \emph{AI Critical Analysis}: The generated text is appended to the report’s conclusion, 
        providing a quick executive-level summary.
        \item \emph{Plots and Figures}: If configured, the script references or embeds seasonal 
        generation comparisons and annual summaries.
    \end{itemize}
\end{enumerate}

% \begin{figure}[H]
%   \centering
%   \begin{tikzpicture}[%
%       every node/.style={draw, rectangle, rounded corners=2pt, font=\small, align=center},
%       node distance=4mm,
%       on grid,
%   ]
  
%   % Title box
%   \node[fill=gray!20, text width=7cm, minimum height=0.8cm] (title) {%
%       \textbf{Scenario Analysis Report: scenario\_X}\\
%       \small{Generated on: YYYY-MM-DD HH:MM}
%   };
  
%   % Plots row
%   \node[fill=blue!10, below=of title, text width=3.5cm, minimum height=2cm] (plot1) {Plot 1\\(Season Comparison)};
%   \node[fill=blue!10, right=3mm of plot1, text width=3.5cm, minimum height=2cm] (plot2) {Plot 2\\(Annual Summary)};
%   \node[fill=blue!10, below=3mm of plot2, text width=3.5cm, minimum height=2cm] (plot3) {Plot 3\\(Scenario Comparison)};
%   \node[draw=none, below=1mm of plot1] (dummy) {};
  
%   % Key metrics box
%   \node[fill=green!20, below=3mm of dummy, text width=7.2cm, minimum height=2cm] (metrics) {%
%       \textbf{Key Metrics}\\
%       \footnotesize \begin{tabular}{ll}
%       Initial Investment: & €X,XXX,XXX \\
%       Annual Cost: & €X,XXX,XXX \\
%       NPV (10\,y): & €X,XXX,XXX
%       \end{tabular}
%   };
  
%   % Generation stats
%   \node[fill=orange!10, below=3mm of metrics, text width=7.2cm, minimum height=2.2cm] (genstats) {%
%       \textbf{Generation / Costs / Capacity Factors}\\
%       \footnotesize \begin{verbatim}
% nuclear: 12345 MWh
% gas: 6789 MWh
% solar: 1357 MWh
%       \end{verbatim}
%   };
  
%   % AI summary box
%   \node[fill=red!10, below=3mm of genstats, text width=7.2cm, minimum height=1.8cm] (ai) {%
%       \textbf{AI Critical Analysis}\\
%       \footnotesize
%       "This scenario effectively balances X and Y,
%       but shows some reliance on gas. We recommend
%       additional storage to mitigate..."
%   };
  
%   % Title at top
%   \draw (title.south) -- (plot1.north);
%   \end{tikzpicture}
%   \caption{A schematic mockup of the generated scenario report layout.}
%   \label{fig:sketch-report}
% \end{figure}

\textbf{Cost Analysis} \\
It is important to note that the OpenAI API has a cost -- $0.00015$ USD per token. As seen in the Figure below on 
the 29th of January 2025, for our 12 scenarios plus 1 global (summary) report, it costed us \$0.0035 USD 
($\pm$ \$0.00027 per scenario). While negligible compared to the investment decisions analyzed sums, this is not 
a free feature. Other open-source LLMs usage could be evaluated.

\begin{figure}[H]
    \centering
    \includegraphics[width=0.75\textwidth]{images/api-cost.png}
    \caption{January 2025 project cost chart of OpenAI API Usage (Report Generation only)}
    \label{fig:api-cost}
\end{figure}

The AI analysis feature was implemented as an optional component that can be enabled or disabled according to user 
preference prior to running the optimization process. Hence guaranteeing that the platform can be used by users 
completely free of costs.


%---------------------------------------------------------------------------------------
\subsection{Technical Implementation}
\label{sec:technical_implementation}

This section covers the practical aspects of using Python and related libraries for solving the 
DCOPF and generating scenario results. It also addresses performance considerations.

\subsubsection{Programming Language \& Libraries}
\label{sec:programming_libraries}

All scripts are written in Python 3.10, leveraging the following key libraries:
\begin{itemize}
  \item \texttt{pulp} for linear programming and interfacing with \texttt{CBC} \cite{forrest2018cbc}. The verssion used is 2.9.0.
  \item \texttt{pandas}, \texttt{numpy}, and \texttt{matplotlib}/\texttt{seaborn} for data manipulation and
  visualization.
  \item \texttt{networkx} for optional graph-based analyses (e.g., if we extend to network exploration).
  \item \texttt{openai} for LLM access and generation of reports.
\end{itemize}


It was decided to use Poetry for dependency management and packaging, ensuring consistent versions across 
different environments. A detail .toml listing all dependencies is available if needed. Poetry facilitates 
the containerization of the platform.

While parts of the project are designed for containerization, certain components like the OpenAI API key require 
secure handling of personal credentials. Currently, the project runs locally without containerization, 
though its architecture was developed with it in mind.

\subsubsection{Performance}
\label{sec:performance_scalability}

While the solver has successfully handled a handful of scenarios (e.g., 20--40), it has yet to be benchmarked 
against large-scale or more complex networks. As order of magnitude, we computed different scenarios 
including sensitivity analysis and scenario plot generation enabled. No AI report generation was enabled.
We yield a linear correlation :

\begin{figure}[h]
    \centering
    \includegraphics[width=0.5\textwidth]{images/time-comput.png}
    \caption{Computation time vs number of scenarios (correlation for limited number of scenarios only)}
    \label{fig:computation-time}
\end{figure}

As this was my first major programming project beyond small scripts, I prioritized a flexible, 
readable codebase over computational efficiency. Nonetheless, handling hourly time-series data for an 
entire year inherently creates a large number of constraints (over 8,700). Real-world systems even adopt 
finer temporal resolutions (e.g., 15-minute intervals). 

To keep run times feasible, we selected representative weeks per season—an approach that cuts computing by 
roughly 17x. (With 52 weeks in a year reduced to just 3 representative weeks -- 52/3 = 17).

% \subsubsection{Validation Checks (Optional)}
% \label{sec:validation_checks}

% Internally, the DCOPF solver in \texttt{dcopf.py} outputs:
% \begin{itemize}
%   \item \textbf{Solver Status}: Confirms if the final status is \texttt{Optimal} (code = 1). Otherwise, 
%   the script prints diagnostic messages.
%   \item \textbf{Power Balance Consistency}: Each bus has an equality constraint ensuring generation plus 
%   inflow equals demand plus outflow.
%   \item \textbf{Line Flow Limits}: Each line’s power flow is constrained by \texttt{rateA}, preventing 
%   unrealistic transfers.
% \end{itemize}
% When executed, the script logs each step, making it relatively straightforward to trace and debug any 
% infeasibilities or unexpected cost spikes. Although no separate “checksum” routine is implemented at 
% present, these logs provide a practical first-level validation of each scenario run.

% \bigskip
% \noindent
% \textbf{In summary, the technical choices---a Python stack, Poetry-driven package management, and a DCOPF 
% solver in \texttt{pulp}---offer a clean, reproducible workflow.} While current tests focus on a limited 
% number of scenarios and relatively small networks, the methodology can be scaled with additional 
% optimization or ML-based strategies for high-volume scenario analyses.
\newpage
\section{Results and Validation}
\label{sec:results}

\subsection{Test Cases (Scenario Definitions)}
\label{sec:testcases_scenarios}

\subsubsection{Configuration}
We used a simple 9-bus network topology, as illustrated below. 

While the initial design considered multiple load buses (at least two), technical implementations limitations in 
integrating storage units with multiple load centers led us to simplify the model to use only bus N°5 as a load bus. 
The remaining buses are available for scenario-specific generation placement. The network consists of 
standardized transmission elements with the following characteristics:
\begin{itemize}
    \item Uniform line parameters: r = 0.00281 p.u., x = 0.0281 p.u., b = 0.00712 p.u.
    \item Consistent thermal limits: 60 MW for all branches
    \item Voltage bounds: 0.9 $\leq$ V $\leq$ 1.1 p.u.
    \item Base voltage level: 230 kV
\end{itemize}

\begin{figure}[h]
    \centering
    \includegraphics[width=0.4\textwidth]{images/network.png}
    \caption{Base topology network}
    \label{fig:base_network}
\end{figure}

To systematically analyze different network configurations, we defined a simple population of 10 
scenarios for the sake of clarity in this report. These scenarios explore various combinations of generation 
and storage placements within our pre-defined network. Each scenario specifies:

\begin{enumerate}
    \item The location and type of generation units (nuclear, solar, wind, or gas) at different buses
    \item The placement of storage units (Battery1 or Battery2) if any
    \item A load factor to scale the demand - in our case kept by default at 1.0 for our analysis.
\end{enumerate}

\begin{table}[H]
\centering
\begin{tabular}{l l l l}
\hline
\textbf{Scen.} & \textbf{Generation Positions} & \textbf{Storage Units} \\
\hline
\centering{1} & \{Bus 1: Nuclear, Bus 4: Solar\} & \textit{None} \\
\centering{2} & \{Bus 1: Nuclear, Bus 4: Solar\} & \{Bus 2: Bat1, Bus 7: Bat1\} \\
\centering{3} & \{Bus 1: Nuclear, Bus 4: Solar, Bus 2: Wind\} & \{Bus 7: Bat1\} \\
\centering{4} & \{Bus 2: Nuclear, Bus 4: Wind, Bus 8: Gas\} & \{Bus 1: Bat1, Bus 7: Bat2\} \\
\centering{5} & \{Bus 1: Wind, Bus 2: Solar, Bus 3: Nuclear\} & \{Bus 8: Bat1, Bus 4: Bat2\} \\
\centering{6} & \{Bus 2: Nuclear, Bus 4: Wind, Bus 7: Solar\} & \textit{None} \\
\centering{7} & \{Bus 3: Solar, Bus 4: Nuclear, Bus 8: Wind\} & \{Bus 1: Bat2\} \\
\centering{8} & \{Bus 1: Gas, Bus 4: Nuclear, Bus 7: Solar\} & \{Bus 9: Bat2, Bus 2: Bat2\} \\
\centering{9} & \{Bus 2: Wind, Bus 4: Nuclear, Bus 7: Solar, Bus 1: Solar\} & \textit{None} \\
\centering{10} & \{Bus 2: Wind, Bus 4: Nuclear, Bus 7: Solar, Bus 1: Solar\} & \{Bus 3: Bat2, Bus 9: Bat2\} \\
\hline
\end{tabular}
\caption{10 scenarios case-study from \texttt{scenarios\_parameters.csv}}
\label{tab:scenario_definitions}
\end{table}


\subsubsection{Data}
The generation units have distinct characteristics -- static limits and variable generation profiles.
Constant limits such as nuclear power plants and gas turbines have a fixed power output. Their costs and max
power were defined with dummy variables. Solar and wind are variable and depend on weather patterns hence have
a variable availability profile.

\begin{table}[h]
\centering
\begin{tabular}{l|c|c}
\hline
Type & P\textsubscript{max} (MW) & Cost (\$/MWh) \\
\hline
Nuclear & 800 & 5.0  \\
Gas & 250 & 8.0  \\
Wind & Variable & 0  \\
Solar & Variable & 0  \\
\hline
\end{tabular}
\caption{Generation Unit Specifications}
\label{tab:gen_specs}
\end{table}

\begin{table}[h]
\centering
\begin{tabular}{l|c|c|c}
\hline
Type & Power (MW) & Capacity (MWh) & Efficiency \\
\hline
Battery1 & $\pm$30 & 60 & 99\% \\
Battery2 & $\pm$55 & 110 & 99\% \\
\hline
\end{tabular}
\caption{Storage Unit Specifications}
\label{tab:storage_specs}
\end{table}


The load who can be interpreted as the demand of a little town (max. 100MW) also is variable. This hourly
fluctuating data was obtained from different sources : 
\begin{itemize}
    \item The load profile was provided by the supervising professor. An STL decomposition was applied 
    to confirm the recurring weekly patterns and seasonalities.
    \item The renewable generation data was obtained 
    from renewables.ninja \cite{renewables_ninja} for a location in Sion, Switzerland (46.231°N, 7.359°E).
    Both solar PV and wind configurations were then processed and scaled.
\end{itemize}

Their profiles can be observed below in their respective seasonal weeks. As discussed in 
Section~\ref{sec:data_preprocessing}, these week choices were made based on the mean load demand per week in 
their respective season. We aimed for the residuals to be normally distributed in these chosen weeks.

\begin{figure}[h]
  \centering
  \begin{subfigure}[b]{0.32\linewidth}
     \includegraphics[width=0.8\linewidth]{images/winter_season_plot.png}
     
     \caption{Winter (WK 02)}
     \label{fig:winter_season}
  \end{subfigure}
  \hfill
  \begin{subfigure}[b]{0.32\linewidth}
     \includegraphics[width=0.8\linewidth]{images/summer_season_plot.png}
     \caption{Summer (WK 31)}
     \label{fig:summer_season}
  \end{subfigure}
  \hfill
  \begin{subfigure}[b]{0.32\linewidth}
     \includegraphics[width=0.8\linewidth]{images/autumn_spring_season_plot.png}
     \caption{Spring/Autumn (WK 43)}
     \label{fig:autumn_spring_season}
  \end{subfigure}
  \caption{Weekly seasonal load and generation availability profiles}
  \label{fig:weekly_seasonal_profiles}
\end{figure}


%---------------------------------------------------------------------------------------
\subsection{Operational Results}
\label{sec:operational_results}

\subsubsection{Hourly Dispatch}
Below is an example of an hourly dispatch during a summer week. As expected it shows some 24-hour seasonality in 
both the generation and demand. The stacked area plot reveals a clear daily pattern where solar generation (dashed 
orange line) peaks during midday hours while wind generation (dashed dark blue line) provides more variable output 
throughout the day. The total generation profile closely follows but slightly exceeds the demand curve (black line), 
with the excess being stored in batteries for later use.

We notice that during peak solar hours, when photovoltaic output exceeds demand, the surplus free of cost energy is 
stored in the battery. This stored energy is then discharged during evening hours when solar generation declines but 
demand remains high -- idem with the wind generation. While more difficult to notice in this configuration, the 
batteries discharge at the end of the calculation cycle to meet the condition $SOC_{0} = SOC_{final}$.

\begin{figure}[h]
    \centering
    \includegraphics[width=0.8\textwidth]{images/gen_vs_demand-summer.png}
    \caption{Hourly dispatch for Scenario~5}
    \label{fig:scenario_5_dispatch}
\end{figure}

On day 2, total generation exceeds demand likely due to the battery model allowing simultaneous charging and discharging. 
Without constraints preventing concurrent charge/discharge operations, the solver can exceed single-direction power 
limits by setting both P\_charge and P\_discharge nonzero in the same hour.

\subsubsection{Feasibility \& Technical Observations}
To validate the model's behavior, particularly after simplifying from multiple loads to a single load bus, 
we analyzed the binding constraints in two contrasting scenarios (4 and 5). A review of the solver logs in 
\texttt{dcopf.py} revealed several critical binding constraints that help verify proper system operation:

\begin{figure}[h]
    \centering
    \begin{subfigure}[b]{0.48\textwidth}
        \includegraphics[width=\textwidth]{images/soc_4.png}
        \caption{Scenario~4, SoC left: Summer, right: Winter}
        \label{fig:soc_4}
    \end{subfigure}
    \hfill
    \begin{subfigure}[b]{0.48\textwidth}
        \includegraphics[width=\textwidth]{images/soc_5.png}
        \caption{Scenario~5, SoC left: Summer, right: Winter}
        \label{fig:soc_5}
    \end{subfigure}
    \caption{State of Charge comparison between Scenarios 4 and 5}
    \label{fig:soc_comparison}
\end{figure}

The State of Charge (SOC) comparison between Scenarios 4 and 5 (Fig.~\ref{fig:soc_comparison}) reveals striking 
differences in storage utilization patterns, despite both scenarios having identical generation and storage 
capacities. The key distinction lies in the generation type at Bus 2:

\begin{itemize}
    \item In Scenario 4, with nuclear generation at Bus 2, we observe relatively modest SOC variations. This 
    reflects the steady, baseload of nuclear power,(consistent output regardless of time of day). We can assume 
    that the batteries primarily serve to optimize power flow rather than accommodate large generation swings.
    They peaked on their only first usage day suggesting an emerging need for storage (60 MWh x2)
    
    \item Scenario 5, featuring solar generation at Bus 2, shows much more SoC fluctuations. The 
    pronounced midday solar peaks drive rapid battery charging-- seeing them reaching their limit every day 
    suggesting a size increase. While evening hours see significant discharge as stored energy supplements 
    the diminished solar output.On the other hand, in summer weeks the batteries are not used at all.
\end{itemize}

The network topology, particularly the lines connecting Bus 2 to buses 5, 7, 8, and 9, plays a key role in 
distributing generation and enabling storage utilization. The placement of generation units impacts power flows 
and storage patterns throughout the network.

\subsubsection{Generation dispatch}
Line flow analysis during peak hours would likely show congestion on transmission corridors connecting major generation 
sources to Bus 5. While detailed hourly patterns weren't examined, the seasonal-week models explored. In the context
of "investment" analysis, stakeholders would likely be interested in the global generation dispatch and their associated
trends.

As expected, nuclear generation increases significantly during winter periods to meet higher demand. 
Most notably in scenario 5, solar generation reaches impressive levels that even exceed nuclear output during peak 
periods, suggesting the potential for solar to serve as a primary generation source.

The annual generation profiles shown below illustrate the key seasonal patterns and generation mix across these two 
scenarios:

\begin{figure}[h]
    \centering
    \begin{subfigure}[b]{0.9\textwidth}
        \includegraphics[width=\textwidth]{images/annual_4.png}
        % \caption{Scenario~4}
        \label{fig:annual_4}
    \end{subfigure}
    \\[\baselineskip]
    \begin{subfigure}[b]{0.9\textwidth}
        \includegraphics[width=\textwidth]{images/annual_5.png}
        %\caption{b) Scenario~5}
        \label{fig:annual_5}
    \end{subfigure}
    \caption{Summary plots of annual generation profiles for investment reports}
    \label{fig:annual_comparison}
\end{figure}

In Scenario 5, solar generation reaches impressive levels that even exceed nuclear output during peak 
periods. This scenario demonstrates how zero-cost renewable generation can create significant economic advantages.
Scenario 5 ranks second-best in overall costs whereas Scenario 4's conventional generation mix shows the worst 
cost performance, highlighting the financial benefits of integrating renewable resources into the power system.

\subsection{Investment \& Economic Analysis}
\label{sec:investment_econ}

The economic assessment focuses exclusively on renewable energy and storage assets, as these represent the key investment 
decisions. While nuclear and gas generation costs are modeled in the optimization to determine optimal dispatch, they are 
excluded from the investment analysis since our primary goal is to evaluate the financial viability of green infrastructure 
additions to an existing conventional generation fleet.

The following parameters were used: 
 \begin{table}[h]
    \centering
    \begin{tabular}{l l r}
    \hline
    \textbf{Parameter Type} & \textbf{Technology} & \textbf{Value}  \\
    \hline
    \multirow{4}{*}{CAPEX (CHF/MW)} 
        & Wind & 1'400'000 \\
        & Solar & 900'000 \\
        & Battery Type 1 & 250'000 \\
        & Battery Type 2 & 450'000 \\
    \hline
    \multirow{4}{*}{Technical Lifetime (years)}
        & Wind & 19 \\
        & Solar & 25 \\
        & Battery Type 1 & 6 \\
        & Battery Type 2 & 8 \\
    \hline
    \multirow{4}{*}{Annual OPEX (\% of CAPEX)}
        & Wind & 4\% \\
        & Solar & 2\% \\
        & Battery Type 1 & 3\% \\
        & Battery Type 2 & 4\% \\
    \hline
    \end{tabular}
    \caption{Updated financial and technical parameters used in the analysis}
    \label{tab:updated_parameters}
    \end{table}

The results illustrate a versatile platform that integrates various generation and storage technologies,
each contributing unique cost and performance profiles. Renewable assets like wind and solar offer 
attractive CAPEX values, while their associated OPEX and technical lifetimes shape long-term economics,
particularly when combined with battery systems that require more frequent replacement.

Among the scenarios analyzed, the mix in Scenario 7, which includes solar, nuclear, and wind generation 
with a single Battery Type 2, delivers the lowest annualized cost and most favorable net present values
over 10 and 30 years. This example highlights the potential of combining diverse assets to achieve a 
balanced and economically sustainable energy mix.
    
Conversely, the configuration in Scenario 4, featuring nuclear, wind, and gas with two types of battery
storage, demonstrates how additional storage and more complex asset mixes can elevate both upfront and 
recurring costs. This underlines the importance of optimizing asset selection and mix based on specific
case study requirements.
    
Overall, these results serve as a foundational example, showcasing the platform's capability to compare
different asset combinations. The insights gained here can be further refined for tailored applications
in specific case studies, emphasizing the critical role of both capital investment and long-term 
operational considerations in energy system planning.

\begin{table}[h]
\centering
\begin{tabular}{l r r r r r}
\hline
\textbf{Scen.} & \textbf{Initial Inv.} & \textbf{Annual Cost} & \textbf{10y NPV} & \textbf{30y NPV} & \textbf{Annuity} \\
\hline
7 &  2'750'000 &  942'766 &  -9'918'474 &  -15'233'659 &  1'353'167 \\
3 &  2'550'000 &  980'376 &  -9'821'150 &  -15'258'653 &  1'355'387 \\
6 &  2'300'000 &  1'048'043 &  -9'829'002 &  -15'353'770 &  1'363'836 \\
5 &  3'000'000 &  905'287 &  -10'113'189 &  -15'478'398 &  1'374'906 \\
9 &  3'200'000 &  1'048'043 &  -10'849'783 &  -16'578'091 &  1'472'589 \\
1 &  900'000 &  1'380'923 &  -10'286'888 &  -16'770'455 &  1'489'676 \\
10 &  4'100'000 &  887'122 &  -11'361'773 &  -16'896'633 &  1'500'885 \\
2 &  1'400'000 &  1'300'109 &  -10'637'014 &  -17'193'991 &  1'527'298 \\
8 &  1'800'000 &  1'275'673 &  -11'172'440 &  -17'715'738 &  1'573'644 \\
4 &  2'100'000 &  1'482'721 &  -12'967'033 &  -20'754'695 &  1'843'586 \\
\hline
\end{tabular}
\caption{Ranked by annuity, financial comparison across scenarios}
\label{tab:financial_comparison}
\end{table}


%---------------------------------------------------------------------------------------
\subsubsection{Sensitivity Analysis}

We evaluated how a \(\pm20\%\) variation in load affects each scenario's NPV over time. 
The table shows that scenarios with high gas dependency (e.g., Scenario 4) exhibit greater NPV 
volatility compared to renewable-heavy configurations. A 20\% load increase causes Scenario 4's 30-year NPV to deteriorate 
by 58\%, while Scenario 5's more diverse mix limits the impact to 75\%:

\begin{table}[ht]
\centering
\caption{NPV Analysis for Scenarios 4--5 under Load Variations (10, 20, 30-year horizons)}
\label{tab:npv_sensitivity}
\begin{tabular}{lccccc}
\hline
\textbf{Scenario} & \textbf{Load Factor} & \textbf{NPV (10yr)} & \textbf{NPV (20yr)} & \textbf{NPV (30yr)} \\
\hline
Scenario 4 & 0.8 & -11'172'440 & -15'548'394 & -17'715'738 \\
           & 1.0 & -12'967'033 & -18'400'575 & -20'754'695 \\
           & 1.2 & -14'761'626 & -21'252'756 & -23'793'652 \\
\hline
Scenario 5 & 0.8 & -8'090'551 & -11'046'381 & -12'382'718 \\
           & 1.0 & -10'113'189 & -13'807'976 & -15'478'398 \\
           & 1.2 & -12'135'827 & -16'569'571 & -18'574'078 \\
\hline
\end{tabular}
\end{table}

We can conclude that the sensitivity analysis provides insights into resilience of the scenarios.


%---------------------------------------------------------------------------------------
\subsection{Report Generation and Insights}
The AI report generates a summary of the results of individuals scenarios focus on three aspects for the individual 
scenario : 
\begin{itemize}
  \item Economic Efficiency of the Generation Mix
  \item System Composition Strengths/Weaknesses
  \item Key Recommendations for Improvement
\end{itemize}

while the global report provides a comprehensive overview of all scenarios, highlighting the 
optimal and suboptimal scenarios. It focus on the following aspects : 
\begin{itemize}
  \item Overall Trends in Cost Effectiveness
  \item Trade-offs Between Different Generation Mixes
  \item Key Success Factors in Better Performing Scenarios
  \item Recommendations for Future Scenario Design
\end{itemize}

In the individual report, while the AI effectively identifies scenarios and references analytical data, its descriptions 
remain somewhat generic and lack the depth of expert analysis. In its current configuration, it serves as a useful complement to, 
rather than replacement for experts advice. The strength lies in providing a context-based overview that 
helps investors quickly grasp the key implications of each scenario.

Also, no specific prompt engineering was performed to optimize the handling of metrics. With a more tailored 
prompt and detailed context and objectives, such as battery dimensioning or investment thresholds, the AI could generate
more targeted and insightful reports that better serve decision-making purposes.

However, the AI's global summary report effectively identifies optimal and suboptimal scenarios while providing 
comprehensive comparisons across multiple evaluation criteria -- on a superior level than the scenario-based report. The AI 
integration proves particularly valuable in generating regression analyses and uncovering relationships between predictor 
variables, leading to meaningful insights of assets usage and their associated costs. The complete global summary 
is available in the appendix.




\newpage
\section{Discussion}
\label{sec:discussion}

%---------------------------------------------------------------------------------------
\subsection{Limitations}
\label{sec:limitations }

\paragraph{Multiple Load Profiles.}
Although the load profiles used in the test case are static (their buses is pre-defined) by platform design it 
is often the case in real life. However, we did not handle multiple loads at once, and the platform is not yet 
designed to handle them. This is a limitation that should be addressed in future work. It suggests is yet designed 
as a single demand per bus-time pair.

\paragraph{Limited Benchmarking.}
The main limitation of the platform is that it lacks comprehensive benchmarking against established 
standards or similar systems. While the platform demonstrates functionality, there is no systematic comparison 
(e.g. IEEE test cases). This makes it difficult to objectively assess the platform's effectiveness compared 
to existing solutions.


\subsection{Methodological Insights}
\label{sec:methodological_insights}

\paragraph{Technical-Economic Integration.}
The platform's key methodological contribution lies in establishing a direct link between technical system design
and economic valuation. By integrating technical dispatch optimization with comprehensive financial analysis, it
enables stakeholders to evaluate both the operational feasibility and economic viability simultaneously during
the design phase. This represents a significant advancement over traditional approaches that often treat technical
and economic assessments as separate processes.

\paragraph{Renewable Asset Valuation.}
The optimization framework enables assessment of renewable energy assets by quantifying the costs of 
conventional "dirty" generation within the entire system. This cost metric, when combined with investment 
parameters, provides a clear basis for comparing different scenarios and determining the profitability of 
renewable alternatives. Rather than evaluating green assets in isolation, this approach reveals their true 
economic value by measuring their impact on overall system costs.

\paragraph{Storage Optimization.}
The platform's analysis of battery state-of-charge patterns reveals opportunities for optimizing 
storage deployment timing and sizing. By examining partial-year installation scenarios, as shown in the 
state-of-charge comparison figures, we can better determine not just the optimal storage capacity but also 
when during the year new storage should be commissioned. This temporal dimension of storage deployment 
represents an important area for future framework development.

\paragraph{Maintenance Management.}
The platform could be enhanced by incorporating a maintenance re-routing system that creates temporary 
alternative paths during asset downtime. This would allow for seamless transitions during maintenance periods 
by automatically redirecting power flows through equivalent backup systems. Such functionality would improve 
system reliability and provide more realistic operational scenarios that account for planned and unplanned 
maintenance events.


%---------------------------------------------------------------------------------------
\subsection{Future Directions}
\label{sec:future_directions}

\paragraph{Benchmarking and Validation.}
A critical next step is establishing comprehensive benchmarking 
against industry standards and IEEE test cases. This would validate the platform's performance and provide 
objective comparisons with existing solutions, building confidence in its results and highlighting areas 
for improvement.

\paragraph{Real-Time Price Integration.}
Incorporating real-time electricity pricing mechanisms would significantly enhance the platform's practical utility. This could enable:

\begin{itemize}
    \item Analysis of arbitrage opportunities between peak and off-peak periods
    \item Evaluation of green energy resale potential, particularly for private owners 
    \item More accurate modeling of revenue streams from grid services
    \item Dynamic optimization of storage charging/discharging based on market conditions
\end{itemize}

\paragraph{Code Optimization.}
While computational efficiency is not an immediate concern, for any serious 
use of the platform the code reliability and maintainability should be improved. Same goes for the interactions 
between the different components.

\paragraph{Enhanced Prompt Engineering.}
Further prompt engineering work could improve report readability, 
aiming at facilitating stakeholder engagement. This would ensure key metrics and findings are presented in clear, 
actionable formats aligned with industry standards.


\subsection{Concluding Remarks}
This platform successfully bridges technical dispatch simulations with investment analysis in energy systems. 
The test cases validate the core functionality while demonstrating how sensitivity analyses, scenario comparisons, 
and AI-enhanced reporting can generate actionable insights. Future development priorities should be chosen based 
on specific use cases, whether utility-scale planning, microgrid optimization, or renewable integration. However,
benchmarking against industry standards and IEEE test cases is a must.
\newpage
\section{Conclusion}

This project has successfully developed and demonstrated a comprehensive investment model for optimizing 
technology asset deployment in integrated energy systems. The platform combines DC Optimal Power Flow (DCOPF) 
simulations with detailed investment analysis, providing valuable insights for decision-makers. Key 
achievements and conclusions include:

\begin{itemize}
    \item \textbf{Investment-Focused Optimization:} The platform effectively evaluates different technology 
    combinations through both technical and economic lenses. As demonstrated in the 9-bus case study, it can 
    simultaneously assess multiple aspects:
    \begin{itemize}
        \item Capital expenditure impacts on long-term profitability
        \item Operational costs and their influence on Net Present Value
        \item Storage sizing and its role in system economics
        \item Trade-offs between conventional and renewable technologies
    \end{itemize}
    
    \item \textbf{Economic Analysis Capabilities:} The model provides robust financial metrics for decision-making:
    \begin{itemize}
        \item Net Present Value (NPV) calculations across different time horizons (10, 20, 30 years)
        \item Annuity comparisons for assets with different lifetimes
        \item Sensitivity analysis to evaluate investment robustness and simulate load increase
    \end{itemize}

    \begin{figure}[H]
        \centering
        \includegraphics[width=0.6\textwidth]{images/global.png}
        \caption{Annuity comparison across scenarios}
        \label{fig:global_annuities}
    \end{figure}

    As shown in Figure~\ref{fig:global_annuities}, scenarios with balanced technology mixes (e.g., Scenarios 7 
    and 5) achieved the lowest annuities, around 1.35M CHF/year. These configurations with renewable generation 
    and appropriate storage, demonstrated the value of diversified technology portfolios. 
    
    In contrast, scenarios heavily dependent on gas generation or oversized storage (e.g., Scenario 4) showed 
    significantly higher annuities, reaching 1.84M CHF/year.
  \end{itemize} 

From a technical standpoint, the platform integrates DCOPF network constraints, multiple energy 
carriers, storage dynamics, and renewable generation variability. And, from a decision support standpoint, 
the system enables rapid scenario comparison, identifies cost drivers, assesses investment risks, and 
provides AI-enhanced analysis.

While the case study utilized a simplified 9-bus system, the methodology and platform have demonstrated their 
capability to handle the core requirements of investment decision support in energy systems. The results show 
that optimal technology selection depends on multiple factors, including:
\begin{itemize}
    \item Initial investment constraints
    \item Operational cost considerations
    \item Technology lifetime and replacement cycles
    \item System reliability requirements
\end{itemize}

Future development should be guided by use case requirements first. For utility-scale planning, priorities 
may include multi-carrier integration (gas, hydrogen) and sophisticated environmental assessments. For microgrid 
optimization, the focus should be on real-time pricing and enhanced storage modeling. For optimal energy system 
dimensioning, improving the framework to better determine optimal generator sizes based on demand profiles would 
be valuable. For renewable integration studies, improving sensitivity analysis capabilities and AI-driven scenario evaluation 
would be most valuable. 

Rather than pursuing all improvements simultaneously, development efforts should align with the intended 
application to ensure appropriate depth and accuracy where it matters most.The platform provides a solid foundation for investment decision-making in energy systems, balancing technical 
feasibility with economic viability. Its modular architecture allows for future extensions to address more 
complex scenarios and additional optimization objectives, such as emissions reduction, while maintaining its 
core strength in economic assessment and technology selection.

\newpage
\section*{Acknowledgements}

For the redaction of this report, I would like to acknowledge the use of artificial intelligence to improve
the clarity and structure of my sentences. The core observations, analyses, and personal reflections
are entirely my own, drawn from my experiences during the field trip and subsequent research. The
LLM usage was employed primarily for language refinement, code formatting, and orthographic corrections.
Its integration helped communicate complex concepts clearly and effectively.


\newpage


% Appendices
\appendix
%\section{Technical Documentation}
%\subsection{Installation Guide}
%\subsection{User Manual}
%\subsection{API Reference}

\section{Code Model}
\subsection{Optimization Model}
\subsection{Solver}
\subsection{Economic Calculations} 
\subsection{AI Integration}
\subsection{Report Generation}

% References
\newpage
\addcontentsline{toc}{section}{References}
\bibliographystyle{plain}
\bibliography{sections/refs}

\end{document}
%----------------------------------------------------------------------------------------

% GUIDELINES
% technical explanation structure
    % start, high-level concepts before diving into details
    % pattern:
        % 1. Purpose/Goal
        % 2. Input requirements
        % 3. Process/Algorithm
        % 4. Output format
        % 5. Example usage
    % key technical areas to focus on:
        % Scenario definition system
        % DCOPF solver integration
        % Multi-scenario analysis process
        % AI integration for analysis
        % Report generation system

% Professor Interests ?
    % Mathematical rigor in DCOPF implementation
    % Scalability of the solution
    % Validation of results
    % Innovation in combining:
        % Power systems
        % Investment analysis
        % AI-powered insights
        % Real-world applicability
        % Code quality and structure

% Focus order
%     2 system architecture and data flow
%     1 DCOPF implementation
%     3 investment analysis layer
%     4 AI integration
%     5 complete workflow with examples

%----------------------------------------------------------------------------------------