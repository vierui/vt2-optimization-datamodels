% !TEX program = pdflatex
% !TEX root = main.tex

%----------------------------------------------------------------------------------------
\documentclass[a4paper,10pt]{article}

% Basic packages
\usepackage[utf8]{inputenc}
\usepackage{graphicx}
\usepackage{float}
\usepackage{amsmath}
\usepackage{amssymb}
\usepackage{amsthm}
\usepackage{multicol}
\usepackage{enumitem}
\usepackage{subcaption}
\usepackage{listings}
\usepackage{xcolor}
\usepackage{tikz}
\usepackage{verbatim}
\usetikzlibrary{positioning}
\usepackage[export]{adjustbox}
\usepackage[margin=2.5cm]{geometry}
\graphicspath{{images/}}
\usepackage{multirow}

% Load custom thesis style from config directory
\usepackage{config/thesis}

% Thesis information
\thesistype{Specialisation Project (VT2)}
\thesistitle{Enhanced Platform for Investment Analysis}
\thesiscontext{Mixed-Integer Linear Programming Optimization Model for Integrated Energy Systems in Python}
\thesisdate{\today}
\thesissemester{HS2024}
\keywords{mixed-integer linear programming, MILP, quantitative modeling, python, strategic planning, optimization, asset valuation, power-flow, platform, forecasting, energy trading}

% Author information
\author{Rui Vieira}
\authorinstitute{Institute of Product Development and Production Technologies (IPP)}
\degree{Master of Science in Engineering}
\studyprogram{Business Engineering, MSc in Engineering}
\studyprogramlink{https://www.msengineering.ch/profiles/engineering-and-it/business-engineering}

% Supervisor information
\supervisorA{Dr. Andrea Giovanni Beccuti}
\supervisorAmail{giovanni.beccuti@zhaw.ch}
\supervisorAweb{https://www.zhaw.ch/en/about-us/person/becc}
\supervisorAinstitute{IEFE Model Based Process Optimisation}
\supervisorAinfo{%
    \supnameA\\
    \supinstituteA\\
    Email: \href{mailto:\supmailA}{\supmailA}\\
}

%----------------------------------------------------------------------------------------
\begin{document}

% Front
\input{front/title}
\input{front/imprint}

% Abstract
\input{front/abstract}

% Table of contents
\setcounter{tocdepth}{2}
\tableofcontents
\newpage

% Main content
\newpage
\section{Introduction}

\subsection{Context \& Motivation}
% Rising share of renewables → need for integrated planning and short-term forecasting
% TODO: Write about:
% - Energy-system planning challenges
% - Why multi-year, multi-season optimization matters
% - Rising renewable penetration requiring integrated planning
% - Need for short-term forecasting capabilities

\subsection{VT1 Recaps}
% DC-power-flow LP prototype; static demand; manual scenario handling, solver, key data flows
% TODO: Write about:
% - DC-power-flow LP prototype characteristics
% - Static demand assumptions
% - Manual scenario handling approach
% - Solver and key data flows
% - System architecture and limitations

\subsubsection{Bottlenecks}
% TODO: Write about:
% - Performance bottlenecks in VT1
% - Areas where VT1 was slow
% - Investment logic misrepresentations
% - Specific problems that motivated VT2 development

\subsection{Goals}
% (i) migrate optimisation core from LP to MILP with realistic unit-commitment and investment decisions
% (ii) add data-driven PV/Wind forecasting to feed the optimiser
% (iii) consolidate codebase (Poetry + tests) -> Not a goal but genuinely improved
% TODO: Write about:
% - Goal (i): Migrate optimization core from LP to MILP
% - Realistic unit-commitment and investment decisions
% - Goal (ii): Add data-driven PV/Wind forecasting
% - Integration of forecasting to feed the optimizer
% - Goal (iii): Consolidate codebase (Poetry + tests)
% - Code quality improvements and testing infrastructure

\newpage

% --- Literature and Toolchain Review ---
\section{Literature and Toolchain Review}

\subsection{Mixed-Integer Programming in Power-System Planning}
Mixed-Integer Programming (MILP) is an improved version of Linear Programming (LP) 
that allows to solve problems with discrete variables for example whether a unit is on or off
(unit commitment). Such binary decisions can not be modelled with pure linear programming.
This allows to integrate investment (capital expenditure) and operational (dispatch, or operating cost)
decisions into one optimization framework. One can co-optimize the true least-cost solution 
that considers both capex and opex together. \cite{andersson2004power, wood2013power}.

Formulating generation expansion planning (GEP) as an MILP captures the binary nature of building 
decisions (build vs. not build) and can include operational details like unit commitment. Despite being possible to include operation 
details such as minimum on/off time, rampe rates, etc. in the model, it was not done in this project.
Not only is it more realistic but it also avoids suboptimal decisions that could arise from 
treating planning and operation separately.

\subsection{Short-term PV/Wind Forecasting Methods}
Short-term forecasting of photovoltaic (PV) or wind power is vital for grid operations. 
Approaches include:
\begin{itemize}
    \item \textbf{Persistence/Empirical}: Simple baselines, e.g., assuming tomorrow equals today.
    \item \textbf{Physical/NWP}: Use weather forecasts and physical models for power prediction.
    \item \textbf{Statistical}: ARIMA/SARIMA and ARIMAX models learn from historical data and 
    exogenous variables \cite{predictive_modeling_notes}. They are interpretable and 
    data-efficient but limited for nonlinearities.
    \item \textbf{Machine Learning}: Methods like neural networks, SVR, and especially gradient 
    boosting (e.g., XGBoost) capture complex patterns and often outperform statistical models 
    when sufficient data is available \cite{grzebyk2021xgboost, zhong2020xgboost, scikit-learn}. 
    Gradient boosting is noted for its accuracy and speed in PV/wind forecasting.
    \item \textbf{Hybrid/Ensemble}: Combine models (e.g., ARIMA+ANN) for improved robustness, 
    though added complexity may not always yield better results.
\end{itemize}
Given these findings, we used gradient boosting (XGBoost) for forecasting, with SARIMA as a 
baseline.

\subsection{Solver Landscape and Selection}
Solving large MILPs requires robust solvers. Commercial options (CPLEX, Gurobi, Xpress) are 
state-of-the-art, offering fast solve times and reliability \cite{mitchell2011pulp, forrest2018cbc}. 
Open-source solvers (CBC, GLPK, SCIP, HiGHS) are free but generally slower and less robust. 
Benchmarks show Gurobi and CPLEX are typically 12--100x faster than CBC, and commercial solvers 
solve more instances to optimality \cite{mittelmann2023benchmarks}.

For this project, CPLEX was chosen for the sake of understanding the solver and for having a 
academic license. Additionally, CPLEX offers a Python API (docplex), which made integration with 
our Python-based workflow straightforward

\subsection{Python Ecosystem for Optimisation and ML}
This interoperability and abundance of libraries is a major reason Python is so dominant in these 
fields today. Our literature review also confirmed that many recent research works in energy systems 
adopt Python for similar tasks, citing its balance of user-friendliness and powerful capabilities

Below is a list of the most relevant libraries for optimization and machine learning:

\begin{itemize}
    \item For optimization, libraries like PuLP and Pyomo allow flexible model formulation and 
    solver switching \cite{mitchell2011pulp}. We used the CPLEX Python API (docplex) for direct 
    integration. 
    \item For ML, scikit-learn and XGBoost provide powerful tools for data processing and 
    forecasting. Others librairies such as PyTorch, TensorFlow, and Keras are the reference ones 
    for deep learning and neural networks \cite{scikit-learn}. 
    \item Statsmodels was used for time-series (SARIMA) modeling as a baseline model.
\end{itemize}

\newpage

\newpage
\section{Problem Definition and Scope}

\subsection{Planning Horizon and System Boundaries}
% TODO: Write about:
% - Multi-year planning horizon definition
% - System boundaries and geographical scope
% - Temporal resolution and seasonal representation
% - Which aspects are excluded (AC power flow, real-time trading, etc.)
% - Justification for exclusions and scope limitations

\subsection{Decision Variables and Constraints}
% Investment binaries, UC binaries, storage SOC, DC flows
% TODO: Write about:
% - Investment binary variables for capacity expansion
% - Unit commitment binary variables for operational decisions
% - Storage state of charge (SOC) variables
% - DC power flow variables
% - Constraint formulations and mathematical relationships

\subsection{Forecast Horizon and Accuracy Targets}
% TODO: Write about:
% - Short-term forecasting horizon definition
% - Accuracy targets for different forecast types
% - Integration of forecasts with optimization model
% - Uncertainty handling and scenario generation
% - Forecast update frequency and rolling horizon approach

\subsection{KPIs}
% Total cost, MILP solve time, forecast MAE/RMSE.
% TODO: Write about:
% - Total system cost metrics
% - MILP solve time performance indicators
% - Forecast accuracy metrics (MAE/RMSE)
% - Investment decision quality measures
% - Operational efficiency indicators

\newpage

\newpage
\section{Methodology}

as mentioned before, the code base has been upgraded to a multi-year MILP and a new
forecasting module. The codebase relies on that strucutre hence the methodology used.

The change on the code base happen mostly between these three blocks:
% \begin{enumerate}
% \item \textbf{Data layer} – \texttt{data/} – Reads network configuration and generation/load time-series
% \item \textbf{Investment model} – \texttt{scripts/} – Solves and optimises the investment problem for a given grid
% \item \textbf{Forecasting module} – \texttt{forecast/} – Builds a day-ahead prediction based on statistical and machine learning models
% \end{enumerate}
\begin{figure}[h!]
  \centering
    \begin{verbatim}
        investment-model/
          +-- data/                       # data layer
          |   \-- ...
          |
          +-- forecast/                   # day-ahead forecasting module
          |   \-- ...
          |
          \-- scripts/                    # MILP investment model
          |   \-- ...
          ...
    \end{verbatim}
  \caption{Code architecture - showing the main components: data layer for grid and time-series inputs, forecasting module, and optimization scripts.}
  \label{fig:code-arch}
\end{figure}

\subsection{Data layer}
\label{sec:data-layer}

The data layer looks similar to the previous by splitting static part (network topology and the assets specifications)
and asset profiles. (defining profiles for the demand and assets capabilities) 
However, switching to a multi-year MILP removed the need for per-scenario network files !
hence creation of a new configuration file \texttt{analysis.json}, which defines the parameters of the investment problem (optimization) 
such as the planning horizon, annual load growths, and representative weeks selection.

In the new version the static grid data files have been simplified, still replicating the
metadata of the previous version using the matlab method (SEARCH FOR REF) but only the necessary 
columns were kept -> made use to clean 
Representative-week slicing (three 168h blocks) is carried over from the LP version and therefore not
described here again.

\begin{figure}[h!]
  \centering
    \begin{verbatim}
        data/         
          +-- grid/                       
          |   +-- analysis.json           # configuration file : horizon, growth, ...
          |   +-- buses.csv               # network topology
          |   +-- generators.csv          # assets specifications, lifetime and CAPEX
          |   +-- lines.csv               # network topology
          |   +-- loads.csv               # demand profile
          |   \-- storages.csv            # assets specifications, lifetime and CAPEX
          +-- processed/                  
              +-- load-2023.csv           # load time-series
              +-- solar-2023.csv          # solar time-series
              \-- wind-2023.csv           # wind time-series
    \end{verbatim}
  \caption{Data layer - showing the main components: static grid data and generation and load profiles.}
  \label{fig:data-layer}
\end{figure}

\subsection{Linear to Mixed-Integer Programming Transition}
\label{subsec:lin_to_milp}

The original \texttt{vt1} framework formulated each seasonal DC--OPF as a \emph{pure linear program} (LP).  
Asset capacities, locations and lifetimes were fixed \emph{exogenously} in a scenario file; the optimiser merely 
scheduled hourly dispatch.  
While convenient, this linear setting could not answer strategic questions such as
\emph{“When should we build or replace a 200\,MW CCGT?”} or
\emph{“Is storage cheaper than new wind under a 10-year horizon?”}.  
To embed such build-versus-dispatch trade-offs directly inside the optimiser, \texttt{vt2} elevates the model to a
\emph{mixed-integer linear program} (MILP).

\subsubsection{New decision layers}
Let $G$ be the set of candidate generators, $S$ the set of storage units,  
$\mathcal{Y}=\{1,\dots,Y\}$ the planning years, and $\Sigma$ the representative seasons.
For every asset $a\!\in\!G\cup S$ and year $y\!\in\!\mathcal{Y}$ we introduce binary variables
\[
\boxed{\;b_{a,y}\in\{0,1\}: \text{ ``build (or replace) asset $a$ in year $y$''}\;}
\qquad
\boxed{\;z_{a,y}\in\{0,1\}: \text{ ``asset $a$ is operational in year $y$''}\;}
\]
Dispatch, power-flow and storage variables remain continuous exactly as in the LP.

\subsubsection{Lifetime logic}
If $L_a$ is the technical lifetime of asset $a$, the staircase constraint
\begin{align}
z_{a,y} \;=\; \sum_{\substack{y'\le y\\ y-y' < L_a}} b_{a,y'}
\qquad\text{and}\qquad
\sum_{\substack{y'\le y\\ y-y' < L_a}} b_{a,y'} \;\le\; 1
\label{eq:lifetime}
\end{align}
(i) activates an installation for $L_a$ consecutive years and  
(ii) forbids overlapping rebuilds.  
Equation~\eqref{eq:lifetime} is linear and keeps the MILP compact---only
$\lvert G\rvert+\lvert S\rvert$ binaries per year.

\subsubsection{Unified objective}
Operational and capital costs are now minimised \emph{simultaneously}:
\begin{equation}
\min \;
\underbrace{\sum_{s\in\Sigma} w_s
           \sum_{y\in\mathcal{Y}}\sum_{g\in G}\sum_{t} c_g\,p_{g,y,s,t}}
_{\text{fuel \& variable}\;(\text{continuous})}
\;+\;
\underbrace{\sum_{y\in\mathcal{Y}}\bigl(
            \sum_{g\in G} \text{CRF}_g\,\text{CapEx}_g \, z_{g,y} +
            \sum_{s\in S} \text{CRF}_s\,\text{CapEx}_s \, z_{s,y}\bigr)}
_{\text{annualised capex}\;(\text{binary})}.
\label{eq:mixed_obj}
\end{equation}
The Capital‐Recovery Factor $\text{CRF}_a(i,L_a)=\tfrac{i(1+i)^{L_a}}{(1+i)^{L_a}-1}$
linearises discounting inside the model; no external NPV spreadsheet is required.

\subsubsection{Side-effects of the transition}
\begin{itemize}[leftmargin=1.8em]
  \item \emph{Richer feasibility space:} load-growth scaling, retirement gaps and replacement timing can be explored within one optimisation run instead of a Monte-Carlo over scenarios.
  \item \emph{No load-shedding slack:} demand must be met unless the MILP itself proves infeasible, giving clear diagnostics when investment is mandatory.
  \item \emph{Computational complexity:} the problem is now NP-hard; however, the binary block created by~\eqref{eq:lifetime} is totally unimodular,\footnote{Sliding-sum constraints form a consecutive-ones matrix; modern branch-and-cut solvers exploit this structure effectively.} so CPLEX finds optimal solutions in minutes for realistic instances ($\mathcal{O}(10^5)$ continuous variables, $\le10^3$ binaries).
\end{itemize}

\subsubsection{Outcome}
By migrating from LP to MILP, \texttt{vt2} integrates capacity expansion, replacement planning and hourly dispatch
in a \emph{single, coherent} optimisation layer, enabling least-cost road-maps rather than ex-post scenario scoring.


\subsection{Forecasting module}
\label{sec:forecasting-module}

The forecasting module is new and is based on a statistical and machine learning models 
that predicts the profile of a given generation asset.






\newpage
\newpage
\section{Implementation}
\label{sec:implementation}

\subsection{Code Walk-through}
% key classes, CLI entry points, config files
% TODO: Write about:
% - Key classes and their responsibilities
% - CLI entry points and command-line interface
% - Configuration file structure and management
% - Module dependencies and interactions
% - Code organization and design patterns

\subsection{Data Structures \& File Formats}
% CSV → Parquet, YAML configs, MLflow logs and pipelines
% TODO: Write about:
% - Migration from CSV to Parquet format
% - YAML configuration file structures
% - MLflow integration for experiment tracking
% - Data pipeline logging and monitoring
% - File format optimization for performance

\subsection{Performance Profiling and Optimisation}
% Matrix stuffing timings, CPLEX solve benchmarks.
% TODO: Write about:
% - Matrix construction timing analysis
% - CPLEX solver performance benchmarks
% - Memory usage optimization strategies
% - Computational bottleneck identification
% - Performance improvement techniques

\subsection{Testing \& Validation Strategy}
% Pytest fixtures, 5-bus toy grid, CI results.
% TODO: Write about:
% - Pytest framework implementation
% - Test fixtures and mock data generation
% - 5-bus toy grid for testing
% - Continuous integration setup and results
% - Validation strategies and coverage metrics

\newpage
\newpage
\section{Results}

\subsection{Investment module using MILP optimization}





\subsection{Forecasting module}
A series of models were developed and compared for operational forecasting of electricity generation. 
Performance was evaluated over a 7-day period starting 2024-01-01. The principal metrics considered are RMSE, 
MAE, and $R^2$, computed both overall and per day. Table~\ref{tab:forecast-metrics} summarizes the test results 
for all model variants.

\begin{table}[h!]
    \centering
    \begin{tabular}{lccc}
        \textbf{Model} & \textbf{RMSE} & \textbf{MAE} & $\mathbf{R^2}$ \\
        \hline
        A. Time only & 0.130 & 0.070 & 0.095 \\
        B. Time + Lag (Bayesian opt.) & 0.022 & 0.011 & 0.973 \\
        C. Bayes Opt. + CV (Time + Lag) & 0.022 & 0.013 & 0.973 \\
        D. Recursive (not used) & 0.119 & 0.053 & -0.281 \\
        E. D + POA Clear-Sky feature & 0.024 & 0.010 & 0.970 \\
    \end{tabular}
    \caption{Forecasting model performance on test set (7 days).}
    \label{tab:forecast-metrics}
\end{table}

%------------------------------------------
\subsubsection*{A. Time Features Only}
The baseline model (A), relying solely on time features, achieved poor predictive performance ($R^2=0.095$), 
with substantial errors for certain days (see daily breakdowns). 

We began with a minimal model using only time-related features (hour and day). This 
provided a simple benchmark and captured regular daily and weekly patterns but did not 
account for weather effects or recent historical trends. It achieved poor performance. 

\begin{figure}[H]
    \centering
    \includegraphics[width=\linewidth]{images/set_time.png}
    \caption{Set 1 - Time features only}
    \label{fig:set1-forecast-profile}
\end{figure}

\begin{table}[H]
    \centering
    \begin{tabular}{lccc}
        Date        & MAE    & RMSE   & R\textsuperscript{2} \\
        \hline
        2024-01-01  & 0.045  & 0.076  & 0.693 \\
        2024-01-02  & 0.098  & 0.169  & -15.138 \\
        2024-01-03  & 0.036  & 0.064  & 0.922 \\
    \end{tabular}
    \caption{Set A - Daily Performance Metrics}
\end{table}

%------------------------------------------
\subsubsection*{B. Time + Lagged Features}
Recognizing the autocorrelated nature of PV output, we next included lagged values of 
electricity generation (e.g., previous hour, previous day, previous week). Incorporating 
these lags enabled the model to better understand night/time patterns (no sub-zero values). However, 
significant errors remained on some days, indicating the model's limited ability to 
capture complex temporal dependencies.

\begin{figure}[H]
    \centering
    \includegraphics[width=\linewidth]{images/set_lags.png}
    \caption{Set B - Time + time/cyclical features (before feature selection)}
\end{figure}

\begin{table}[H]
    \centering
    \begin{tabular}{lccc}
        Date        & MAE    & RMSE   & R\textsuperscript{2} \\
        \hline
        2024-01-01  & 0.043  & 0.078  & 0.677 \\
        2024-01-02  & 0.087  & 0.156  & -12.658 \\
        2024-01-03  & 0.037  & 0.069  & 0.908 \\
    \end{tabular}
    \caption{Set 2 - Daily Performance Metrics}
\end{table}

\textbf{Feature Importance Analysis} Understanding which features 
most significantly impact the model's predictions is crucial for 
interpretability and further model refinement. We conducted a feature 
importance analysis by keeping only the most important features in the 
training set.

We conducted  the feature optimization number by using the Bayesian optimization search.
It learns from previous trials, improving efficiency over brute-force methods sucha grid 
search or back-/front- ward elimination which are more expensive.

\begin{figure}[h!]
    \centering
    \includegraphics[width=0.75\linewidth]{images/feature-importance-cyclic.png}
    \caption{Feature Importance derived from the Bayesian optimization search}
    \label{fig:feature-importance}
\end{figure}

The decrease of the error in this step is highly significant. Not only does it 
show that over 95\% of the model's predictive power was explained by just three 
features: \texttt{electricity\_lag1}, \texttt{electricity\_lag24}, and \texttt{hour\_sin} 
but it also highlights the relevance of removing noise (excessive features) in the prediction
to avoid overfitting. 

\begin{figure}[H]
    \centering
    \includegraphics[width=\linewidth]{images/set_lags2.png}
    \caption{Set B - Time + time/cyclical features (after feature selection)}
    \label{fig:set2-forecast-profile}
\end{figure}

\begin{table}[H]
    \centering
    \begin{tabular}{lccc}
        Date        & MAE    & RMSE   & R\textsuperscript{2} \\
        \hline
        2024-01-01  & 0.013  & 0.025  & 0.967 \\
        2024-01-02  & 0.013  & 0.023  & 0.693 \\
        2024-01-03  & 0.012  & 0.019  & 0.993 \\
    \end{tabular}
    \caption{Set 2 - Daily Performance Metrics}
\end{table}

%------------------------------------------
\subsubsection*{C. Parameter tuning and cross-validation}
To evaluate how well your model is likely to perform on unseen data, we performed 
cross validation (evaluation mechanism). It splits the data into several parts (folds), 
trains on some, and tests on others. It should prevent overfitting. It requires a peticular
attention when it comes to time series data so the splits are not randomly picked but rather 
sequential in time.

Hyperparameter tuning is the search process to find the best hyperparameters for the model.
Its robustness is improved. Again, we used a Bayesian optimization search to find the 
best hyperparameters.

\begin{figure}[H]
    \centering
    \includegraphics[width=\linewidth]{images/set_bayesian1.png}
    \caption{Set C - Features selected by Bayesian optimization}
    \label{fig:set3-forecast-profile}
\end{figure}

\begin{table}[H]
    \centering
    \caption{Set 3 - Daily Performance Metrics}
    \begin{tabular}{lccc}
        Date        & MAE    & RMSE   & R\textsuperscript{2} \\
        \hline
        2024-01-01  & 0.009  & 0.023  & 0.972 \\
        2024-01-02  & 0.010  & 0.022  & 0.732 \\
        2024-01-03  & 0.011  & 0.019  & 0.993 \\
    \end{tabular}
\end{table}

Ironically, these steps could lead to overfitting and poor generalization. 
Simultaneous optimization of hyperparameters with limited evaluation 
calls may not fully explore the searchspace, leading to suboptimal solutions. 
These methods are computationally expensive and using too wide ranges or too 
little cv folds may lead to poor results. It's a trade-off between exploration and 
exploitation. 

We see here, that despite the (expensive) cross-validation and hyperparameter tuning, 
the model does not generalize much better than the one with feature selection only. 

%------------------------------------------
\subsubsection*{D. Recursive Prediction}
In operational settings, true future lag values are unknown; predictions must be generated recursively, 
using model outputs as future lag inputs. This recursive prediction (Model D) more accurately reflects 
real-world constraints, but error accumulation can occur, leading to degraded performance.

\begin{figure}[H]
    \centering
    \includegraphics[width=\linewidth]{images/set_recursive.png}
    \caption{Set 4 - Recursive prediction}
    \label{fig:set4-forecast-profile}
\end{figure}

\begin{table}[H]
    \centering
    \begin{tabular}{lccc}
        Date        & MAE    & RMSE   & R\textsuperscript{2} \\
        \hline
        2024-01-01  & 0.080  & 0.160  & -0.353 \\
        2024-01-02  & 0.027  & 0.052  & -0.499 \\
        2024-01-03  & \textit{NaN}    & \textit{NaN}    & \textit{NaN} \\
    \end{tabular}
    \caption{Set 4 - Daily Performance Metrics}
\end{table}

The model seemed successfully implemented including the preprocessing of time-series 
data (removing nighttime hours), aligning input-output sequences, and setting up the loop 
logic to feed previous predictions into future steps. However, the model did not yield 
stable or reliable results. 

It remained flat. This behavior is most likely due error accumulation and feedback saturation, 
where early incorrect low predictions suppress the entire sequence. Broken temporal continuity 
from removing nighttime data, represents a clear challenge in the modeling too. We tried to 
implement the method but not extensive research was done.  

%------------------------------------------
\subsubsection*{E. Enhanced Features with POA Clear-Sky}
Finally, we extended the feature set to include physics-based drivers—specifically, 
plane-of-array (POA) clear-sky irradiance. By combining these weather-driven variables 
with the selected lags and time features, the model could account for both physical 
potential and recent variability, resulting in the best overall performance.

\begin{figure}[H]
    \centering
    \includegraphics[width=\linewidth]{images/set_poa.png}
    \caption{Set 5 - Enhanced features with POA clear-sky}
    \label{fig:set5-forecast-profile}
\end{figure}

\begin{table}[H]
    \centering
    \begin{tabular}{lccc}
        Date        & MAE    & RMSE   & R\textsuperscript{2} \\
        \hline
        2024-01-01  & 0.008  & 0.026  & 0.970 \\
        2024-01-02  & 0.009  & 0.022  & 0.725 \\
        2024-01-03  & 0.015  & 0.028  & 0.985 \\
    \end{tabular}
    \caption{Set 5 - Daily Performance Metrics}
\end{table}

Error and accuracy metrics are improved. However, the model is still far from being perfect 
and computational expense is high for a sole 0.1\% improvement of the MAE-error for the day ahead 
prediction horizon. 

\begin{figure}[H]
    \centering
    \includegraphics[width=0.75\linewidth]{images/feature-importance-weather.png}
    \caption{Feature Importance including weather-features}
    \label{fig:feature-importance-weather}
\end{figure}

We performed a feature selection with the new weather-features based to understand their importance 
compare to the previous time-only-lags. We notice 7 additional "POA"-features in the top 20 and we see 
that within out parameter array seach we manage to quantify the the sum of their importance which 
improves the prediction performance by 0.2\% of the MAE-error. In average POA-features impact is +0.003.


% \begin{table}[h]
% \centering
% \caption{Validation MAE by model (2022–2023 split).}
% \label{tab:model-comp}
% \begin{tabular}{lcc}
% \hline
% Model & MAE [kW] & Rel. $\Delta$ vs. SARIMA \\
% \hline
% Naïve Seasonal      & 0.142 & +34\% \\
% SARIMA baseline     & 0.106 & — \\
% MLP (2×128)         & 0.565 & +433\% \\
% TCN (64f, 4 blk)    & 0.128 & +21\% \\
% GBDT (XGBoost)      & 0.090 & $-$15\% \\
% SARIMA + GBDT (ours)& 0.085 & $-$20\% \\
% \hline
% \end{tabular}
% \end{table}
% % Table of MAE / RMSE for each model; plots (actual vs forecast).
% % TODO: Write about:
% % - Comparison table of MAE/RMSE for different models
% % - Actual vs forecast plots and visualizations
% % - Model performance across different seasons
% % - Accuracy metrics for different forecast horizons
% % - Statistical significance of improvements

% \subsection{MILP vs Legacy LP}
% % Cost reduction, unit-commitment realism, run-time overhead.
% % TODO: Write about:
% % - Total system cost comparisons
% % - Unit-commitment decision realism improvements
% % - Investment decision quality analysis
% % - Run-time overhead assessment
% % - Solution quality and optimality gaps

% \subsection{Solver Impact (CPLEX vs GLPK)}
% % Solve time, mipgap convergence, memory footprint.
% % TODO: Write about:
% % - Solve time performance comparisons
% % - MIP gap convergence analysis
% % - Memory footprint differences
% % - Scalability improvements
% % - Robustness and reliability comparisons

% \subsection{Sensitivity and Scenario Analysis}
% % Congestion pricing, Lifetime sensitivity, discount-rate sweep.
% % TODO: Write about:
% % - Congestion pricing impact analysis
% % - Asset lifetime sensitivity studies
% % - Discount rate parameter sweeps
% % - Load growth scenario comparisons
% % - Renewable penetration sensitivity

% \subsection{Integrated Workflow Demo}
% % End-to-end run + gantt like result.
% % TODO: Write about:
% % - Complete workflow demonstration
% % - End-to-end execution results
% % - Timeline and scheduling visualization
% % - Integration performance metrics
% % - Workflow automation benefits

% \newpage 
% \newpage
% \section{Discussion}

% \subsection{Interpretation of Key Findings}
% Interpretation of key findings and changes/advantages LP vs MILP
% TODO: Write about:
% - Key findings from the LP to MILP transition
% - Advantages and improvements achieved
% - Investment decision quality improvements
% - Unit commitment realism benefits
% - Overall system performance gains

% \subsection{Trade-offs Analysis}
% Trade-offs: model fidelity vs computation; ML accuracy vs complexity.
% TODO: Write about:
% - Model fidelity vs computational complexity trade-offs
% - Machine learning accuracy vs model complexity
% - Solver performance vs solution quality
% - Forecast accuracy vs computational requirements
% - Real-time vs planning horizon considerations

\subsection{Limitations}
% Limitations (data quality, solver scalability, stochastic coupling still WIP, time, tuning of tuning).
% TODO: Write about:
% - Data quality limitations and impacts
% - Solver scalability constraints
% - Stochastic coupling work in progress
% - Time constraints and development limitations
% - Hyperparameter tuning challenges
% - Model validation limitations

\newpage
\section{Conclusions and Outlook}

\subsection{Achievements Relative to Goals}
% Achievements relative to goals.
% TODO: Write about:
% - Goal achievement assessment
% - LP to MILP migration success
% - Forecasting module implementation
% - Code consolidation and testing improvements
% - Performance improvements quantified
% - Deliverables completed

\subsection{Near-term Tasks}
% Near-term tasks (complete NN robustness tests, finish trading prototype).
% TODO: Write about:
% - Complete neural network robustness tests
% - Finish trading prototype development
% - Model validation and testing completion
% - Documentation and deployment tasks
% - Performance optimization opportunities

\subsection{Long-term Vision}
% Long-term vision (possible integration of forecasts/trading with MILP).
% TODO: Write about:
% - Integration possibilities of forecasting with MILP
% - Integration of trading module with main optimization
% - Future research directions and opportunities
% - Scalability improvements and extensions
% - Real-world deployment considerations

\newpage
\section*{Acknowledgements}

For the redaction of this report, I would like to acknowledge the use of artificial intelligence to improve
the clarity and structure of my sentences. The core observations, analyses, and personal reflections
are entirely my own, drawn from my experiences during the field trip and subsequent research. The
LLM usage was employed primarily for language refinement, code formatting, and orthographic corrections.
Its integration helped communicate complex concepts clearly and effectively.

\newpage


% Appendices
%\appendix

\section{Data Files \& Directory Layout}
% TODO: Document:
% - Data file structure and organization
% - Directory layout explanation
% - File naming conventions
% - Data sources and formats
% - Raw data vs processed data organization

\section{Major Code Listings}
% TODO: Include relevant code sections

\subsection{Optimization Model}
\lstinputlisting[
    language=Python,
    caption={DCOPF Implementation},
    label={lst:dcopf},
]{../scripts/optimization.py}

\subsection{Solver}
\lstinputlisting[
    language=Python,
    caption={Main Solver Implementation},
    label={lst:main},
]{../scripts/main.py}

\subsection{Forecasting Module}
% TODO: Add forecasting code listings
% \lstinputlisting[
%     language=Python,
%     caption={Forecasting Implementation},
%     label={lst:forecast},
% ]{../forecast2/src/}

\section{Mathematical Formulations}
% TODO: Include detailed mathematical formulations:
% - Detailed constraint sets
% - CRF (Capital Recovery Factor) derivation
% - MILP formulation differences from LP
% - Objective function formulations
% - Binary variable definitions
% - Lifetime-linking constraints

\subsection{MILP Formulation}
% TODO: Write detailed MILP mathematical formulation

\subsection{Capital Recovery Factor (CRF) Derivation}
% TODO: Provide CRF mathematical derivation

\subsection{Constraint Sets}
% TODO: Document all constraint sets used in the model

\section{Config/YAML Samples}
% TODO: Include sample configuration files
% - Main optimization config
% - Forecasting config
% - Data pipeline config
% - Solver settings

\section{Extra Plots \& Test Outputs}
% TODO: Include additional figures and test results
% - Extended performance plots
% - Sensitivity analysis results
% - Test coverage reports
% - Benchmark comparisons

\section{Symbols Glossary}
% TODO: Define all symbols used in mathematical formulations
% - Decision variables
% - Parameters
% - Sets and indices
% - Mathematical operators


% References
\newpage
\addcontentsline{toc}{section}{References}
\bibliographystyle{plain}
\bibliography{sections/refs}

\end{document}
%----------------------------------------------------------------------------------------

% GUIDELINES FOR VT2
% Technical explanation structure
    % start, high-level concepts before diving into details
    % pattern:
        % 1. Purpose/Goal
        % 2. Input requirements
        % 3. Process/Algorithm
        % 4. Output format
        % 5. Example usage
    % key technical areas to focus on:
        % MILP formulation and implementation
        % Solver migration and performance improvements
        % Constraint restructuring and vectorization
        % Forecasting module development
        % Energy trading add-on module
        % Multi-scenario analysis process
        % AI integration for analysis
        % Report generation system

% Professor Interests ?
    % Mathematical rigor in MILP implementation
    % Performance improvements over VT1
    % Scalability of the solution
    % Validation of results
    % Innovation in combining:
        % Power systems optimization
        % Investment analysis
        % Forecasting capabilities
        % Energy trading strategies
        % AI-powered insights
        % Real-world applicability
        % Code quality and structure

% Focus order for VT2
%     1 VT1 baseline recap and identified bottlenecks
%     2 MILP implementation and solver migration
%     3 Constraint restructuring and performance improvements
%     4 Forecasting module (stand-alone research)
%     5 Results and performance benchmarks
%     6 Energy trading add-on (independent module)
%     7 Comprehensive evaluation and future work

%----------------------------------------------------------------------------------------