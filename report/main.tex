% !TEX program = pdflatex
% !TEX root = main.tex

\documentclass{article}
\usepackage[utf8]{inputenc}
\usepackage{graphicx, float} 
\usepackage{amsmath, amssymb, amsthm}
\graphicspath{{images/}}
\usepackage{enumitem}
\usepackage{subcaption}

\usepackage[letterpaper, top=0.8in, bottom=1.0in, left=1.2in, right=1.2in, heightrounded]{geometry}
\renewcommand{\baselinestretch}{1.15}
\setlength{\parindent}{0pt}
\setlength{\parskip}{0.8em}

\title{Semester Project}
\author{Rui Vieira}
\date{February 2025}

%----------------------------------------------------------------------------------------
\begin{document}

% TITLE PAGE AND IMPRINT
%----------------------------------------------------------------------------------------
\begin{titlepage}

\begin{center}

\textup{\small {\bf Specialisation Project (VT) } \\  HS2024}\\[3.0in]

% Title
\fontsize{20}{24}\selectfont
\textbf {Platform for Investment Analysis}\\
\normalsize Optimization Framework for Energy Asset Management using Linear Programming in Python\\[3.0in]


       

% Submitted by
\normalsize Submitted by \\[0.2in]
\textbf{Rui Vieira}\\
Business Engineering Profile\\
IPP Institute of Product Development and Production Technologies\\

\vspace{.2in}




\vspace{.3in}

% Bottom of the page
\includegraphics[width=0.4 \textwidth]{images/mse.png}\\[0.1in]

January 2025

\end{center}

\end{titlepage}

\tableofcontents
\newpage

\section{Introduction}
\subsection{Project Context}
Brief introduction to energy investment optimization and the need for automated analysis tools.

\subsection{Objectives}
\begin{itemize}
    \item Develop a Python-based platform for energy investment analysis
    \item Implement linear programming optimization for asset management
    \item Create scenario analysis capabilities
    \item Provide AI-powered insights for decision support
\end{itemize}
\newpage

\section{Theoretical Background}
\subsection{Linear Programming in Energy Systems}
Overview of optimization techniques in energy asset management.

\subsection{DC Optimal Power Flow}
The DC Optimal Power Flow (DC-OPF) is a simplified version of the AC power flow problem, commonly used in power systems analysis and optimization. This linearized model makes the following key assumptions:

\begin{itemize}
    \item Voltage magnitudes are assumed to be 1.0 per unit (p.u.)
    \item Line resistances are negligible compared to reactances ($R \ll X$)
    \item Voltage angle differences between connected buses are small
    \item Reactive power flows are ignored
\end{itemize}

\subsubsection{Mathematical Formulation}
The DC-OPF problem can be formulated as follows:

\begin{align}
    \min_{\mathbf{P_g}, \boldsymbol{\theta}} \quad & \sum_{i \in \mathcal{G}} c_i(P_{g,i}) \label{eq:dcopt_obj} \\
    \text{subject to:} \quad & \nonumber \\
    & P_{g,i} - P_{d,i} = \sum_{j \in \mathcal{N}_i} B_{ij}(\theta_i - \theta_j) \quad \forall i \in \mathcal{N} \label{eq:power_balance} \\
    & P_{g,i}^{\min} \leq P_{g,i} \leq P_{g,i}^{\max} \quad \forall i \in \mathcal{G} \label{eq:gen_limits} \\
    & -P_{ij}^{\max} \leq B_{ij}(\theta_i - \theta_j) \leq P_{ij}^{\max} \quad \forall (i,j) \in \mathcal{L} \label{eq:flow_limits} \\
    & \theta_{\text{ref}} = 0 \label{eq:ref_angle}
\end{align}

where:
\begin{itemize}
    \item $\mathcal{N}$ is the set of all buses
    \item $\mathcal{G}$ is the set of generator buses
    \item $\mathcal{L}$ is the set of transmission lines
    \item $\mathcal{N}_i$ is the set of buses connected to bus $i$
    \item $P_{g,i}$ is the active power generation at bus $i$
    \item $P_{d,i}$ is the active power demand at bus $i$
    \item $\theta_i$ is the voltage angle at bus $i$
    \item $B_{ij}$ is the susceptance of the line connecting buses $i$ and $j$
    \item $c_i(P_{g,i})$ is the cost function for generator $i$
    \item $P_{g,i}^{\min}, P_{g,i}^{\max}$ are the generation limits
    \item $P_{ij}^{\max}$ is the transmission line capacity
\end{itemize}

The objective function \eqref{eq:dcopt_obj} minimizes the total generation cost. Constraint \eqref{eq:power_balance} ensures power balance at each bus, \eqref{eq:gen_limits} enforces generator limits, \eqref{eq:flow_limits} ensures transmission line limits are not violated, and \eqref{eq:ref_angle} sets the reference bus angle to zero.

\subsubsection{Line Flow Calculation}
The power flow $P_{ij}$ through a transmission line connecting buses $i$ and $j$ is calculated as:

\begin{equation}
    P_{ij} = B_{ij}(\theta_i - \theta_j)
\end{equation}

This linear relationship between power flow and voltage angles is a key characteristic of the DC power flow approximation.

\subsection{Economic Analysis Framework}
NPV calculations, investment metrics, and risk assessment methods.

\section{Platform Architecture}
\subsection{System Design}
Overall structure and component interaction.

\subsection{Key Components}
\begin{itemize}
    \item Optimization engine
    \item Scenario generator
    \item Results analyzer
    \item AI critique module
\end{itemize}

\section{Implementation}
\subsection{Core Optimization Module}
Description of the linear programming implementation.

\subsection{Scenario Analysis Framework}
How different scenarios are generated and compared.

\subsection{AI Integration}
Implementation of AI-powered analysis features.

\section{Results and Validation}
\subsection{Test Cases}
Description of validation scenarios.

\subsection{Performance Analysis}
Computational efficiency and scalability.

\subsection{Case Studies}
Real-world applications and insights.

\section{Discussion}
\subsection{Platform Capabilities}
Current functionality and limitations.

\subsection{Future Improvements}
Potential enhancements and extensions.

\section{Conclusion}
Summary of achievements and recommendations.

\appendix
\section{Technical Documentation}
\subsection{Installation Guide}
\subsection{User Manual}
\subsection{API Reference}

\section{Mathematical Formulations}
\subsection{Optimization Model}
\subsection{Economic Calculations}

\bibliographystyle{plain}
\bibliography{references}

\end{document}