% !TEX program = pdflatex
% !TEX root = main.tex

\documentclass{article}
\usepackage[utf8]{inputenc}
\usepackage{graphicx, float} 
\usepackage{amsmath, amssymb, amsthm}
\graphicspath{{images/}}
\usepackage{enumitem}
\usepackage{subcaption}

\usepackage[letterpaper, top=0.8in, bottom=1.0in, left=1.2in, right=1.2in, heightrounded]{geometry}
\renewcommand{\baselinestretch}{1.15}
\setlength{\parindent}{0pt}
\setlength{\parskip}{0.8em}

\title{Semester Project}
\author{Rui Vieira}
\date{January 2025}

\begin{document}

\begin{titlepage}

\begin{center}

\textup{\small {\bf Specialisation Project (VT) } \\  HS2024}\\[3.0in]

% Title
\fontsize{20}{24}\selectfont
\textbf {Platform for Investment Analysis}\\
\normalsize Optimization Framework for Energy Asset Management using Linear Programming in Python\\[3.0in]


       

% Submitted by
\normalsize Submitted by \\[0.2in]
\textbf{Rui Vieira}\\
Business Engineering Profile\\
IPP Institute of Product Development and Production Technologies\\

\vspace{.2in}




\vspace{.3in}

% Bottom of the page
\includegraphics[width=0.4 \textwidth]{images/mse.png}\\[0.1in]

January 2025

\end{center}

\end{titlepage}
\tableofcontents
\newpage

\section{Introduction}
\subsection{Project Context}
Brief introduction to energy investment optimization and the need for automated analysis tools.

\subsection{Objectives}
\begin{itemize}
    \item Develop a Python-based platform for energy investment analysis
    \item Implement linear programming optimization for asset management
    \item Create scenario analysis capabilities
    \item Provide AI-powered insights for decision support
\end{itemize}

\section{Theoretical Background}
\subsection{Linear Programming in Energy Systems}
Overview of optimization techniques in energy asset management.

\subsection{DC Optimal Power Flow}
Mathematical formulation and constraints.

\subsection{Economic Analysis Framework}
NPV calculations, investment metrics, and risk assessment methods.

\section{Platform Architecture}
\subsection{System Design}
Overall structure and component interaction.

\subsection{Key Components}
\begin{itemize}
    \item Optimization engine
    \item Scenario generator
    \item Results analyzer
    \item AI critique module
\end{itemize}

\section{Implementation}
\subsection{Core Optimization Module}
Description of the linear programming implementation.

\subsection{Scenario Analysis Framework}
How different scenarios are generated and compared.

\subsection{AI Integration}
Implementation of AI-powered analysis features.

\section{Results and Validation}
\subsection{Test Cases}
Description of validation scenarios.

\subsection{Performance Analysis}
Computational efficiency and scalability.

\subsection{Case Studies}
Real-world applications and insights.

\section{Discussion}
\subsection{Platform Capabilities}
Current functionality and limitations.

\subsection{Future Improvements}
Potential enhancements and extensions.

\section{Conclusion}
Summary of achievements and recommendations.

\appendix
\section{Technical Documentation}
\subsection{Installation Guide}
\subsection{User Manual}
\subsection{API Reference}

\section{Mathematical Formulations}
\subsection{Optimization Model}
\subsection{Economic Calculations}

\bibliographystyle{plain}
\bibliography{references}

\end{document}